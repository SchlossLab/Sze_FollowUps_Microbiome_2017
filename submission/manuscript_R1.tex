\documentclass[12pt,]{article}
\usepackage{lmodern}
\usepackage{amssymb,amsmath}
\usepackage{ifxetex,ifluatex}
\usepackage{fixltx2e} % provides \textsubscript
\ifnum 0\ifxetex 1\fi\ifluatex 1\fi=0 % if pdftex
  \usepackage[T1]{fontenc}
  \usepackage[utf8]{inputenc}
\else % if luatex or xelatex
  \ifxetex
    \usepackage{mathspec}
  \else
    \usepackage{fontspec}
  \fi
  \defaultfontfeatures{Ligatures=TeX,Scale=MatchLowercase}
\fi
% use upquote if available, for straight quotes in verbatim environments
\IfFileExists{upquote.sty}{\usepackage{upquote}}{}
% use microtype if available
\IfFileExists{microtype.sty}{%
\usepackage{microtype}
\UseMicrotypeSet[protrusion]{basicmath} % disable protrusion for tt fonts
}{}
\usepackage[margin=1.0in]{geometry}
\usepackage{hyperref}
\hypersetup{unicode=true,
            pdfborder={0 0 0},
            breaklinks=true}
\urlstyle{same}  % don't use monospace font for urls
\usepackage{longtable,booktabs}
\usepackage{graphicx,grffile}
\makeatletter
\def\maxwidth{\ifdim\Gin@nat@width>\linewidth\linewidth\else\Gin@nat@width\fi}
\def\maxheight{\ifdim\Gin@nat@height>\textheight\textheight\else\Gin@nat@height\fi}
\makeatother
% Scale images if necessary, so that they will not overflow the page
% margins by default, and it is still possible to overwrite the defaults
% using explicit options in \includegraphics[width, height, ...]{}
\setkeys{Gin}{width=\maxwidth,height=\maxheight,keepaspectratio}
\IfFileExists{parskip.sty}{%
\usepackage{parskip}
}{% else
\setlength{\parindent}{0pt}
\setlength{\parskip}{6pt plus 2pt minus 1pt}
}
\setlength{\emergencystretch}{3em}  % prevent overfull lines
\providecommand{\tightlist}{%
  \setlength{\itemsep}{0pt}\setlength{\parskip}{0pt}}
\setcounter{secnumdepth}{0}
% Redefines (sub)paragraphs to behave more like sections
\ifx\paragraph\undefined\else
\let\oldparagraph\paragraph
\renewcommand{\paragraph}[1]{\oldparagraph{#1}\mbox{}}
\fi
\ifx\subparagraph\undefined\else
\let\oldsubparagraph\subparagraph
\renewcommand{\subparagraph}[1]{\oldsubparagraph{#1}\mbox{}}
\fi

%%% Use protect on footnotes to avoid problems with footnotes in titles
\let\rmarkdownfootnote\footnote%
\def\footnote{\protect\rmarkdownfootnote}

%%% Change title format to be more compact
\usepackage{titling}

% Create subtitle command for use in maketitle
\newcommand{\subtitle}[1]{
  \posttitle{
    \begin{center}\large#1\end{center}
    }
}

\setlength{\droptitle}{-2em}
  \title{}
  \pretitle{\vspace{\droptitle}}
  \posttitle{}
  \author{}
  \preauthor{}\postauthor{}
  \date{}
  \predate{}\postdate{}

\usepackage{helvet} % Helvetica font
\renewcommand*\familydefault{\sfdefault} % Use the sans serif version of the font
\usepackage[T1]{fontenc}

\usepackage[none]{hyphenat}

\usepackage{setspace}
\doublespacing
\setlength{\parskip}{1em}

\usepackage{lineno}
\usepackage{microtype}
\usepackage{pdfpages}
\usepackage{comment}
\usepackage{lscape}

\begin{document}

\section{Normalization of the microbiota in patients after treatment for
colonic
lesions}\label{normalization-of-the-microbiota-in-patients-after-treatment-for-colonic-lesions}

\begin{center}
\vspace{25mm}

Marc A Sze${^1}$, Nielson T Baxter${^1}$${^,}$${^2}$, Mack T Ruffin IV${^3}$, Mary AM Rogers${^2}$, and Patrick D Schloss${^1}$${^\dagger}$

\vspace{20mm}

$\dagger$ To whom correspondence should be addressed: pschloss@umich.edu

$1$ Department of Microbiology and Immunology, University of Michigan, Ann Arbor, MI

$2$ Department of Internal Medicine, University of Michigan, Ann Arbor, MI

$3$ Department of Family Medicine and Community Medicine, Penn State Hershey Medical Center, Hershey, PA


\end{center}

Co-author e-mails:

\begin{itemize}
\tightlist
\item
  \href{mailto:marcsze@med.umich.med.edu}{\nolinkurl{marcsze@med.umich.med.edu}}
\item
  \href{mailto:ntbaxter@umich.edu}{\nolinkurl{ntbaxter@umich.edu}}
\item
  \href{mailto:mruffin@pennstatehealth.psu.edu}{\nolinkurl{mruffin@pennstatehealth.psu.edu}}
\item
  \href{mailto:maryroge@med.umich.edu}{\nolinkurl{maryroge@med.umich.edu}}
\end{itemize}

\newpage

\linenumbers

\subsection{Abstract}\label{abstract}

\textbf{Background.} Colorectal cancer is a worldwide health problem.
Despite growing evidence that members of the gut microbiota can drive
tumorigenesis, little is known about what happens to it after treatment
for an adenoma or carcinoma. This study tested the hypothesis that
treatment for adenoma or carcinoma alters the abundance of bacterial
populations associated with disease to those associated with a normal
colon. We tested this hypothesis by sequencing the 16S rRNA genes in the
feces of 67 individuals before and after treatment for adenoma (N = 22),
advanced adenoma (N = 19), and carcinoma (N = 26).

\textbf{Results.} There were small changes to the bacterial community
associated with adenoma or advanced adenoma and large changes associated
with carcinoma. The communities from patients with carcinomas changed
significantly more than those with adenoma following treatment (P-value
\textless{} 0.001). Although treatment was associated with intrapersonal
changes, the change in the abundance of individual OTUs in response to
treatment was not consistent within diagnosis groups (P-value
\textgreater{} 0.05). Because the distribution of OTUs across patients
and diagnosis groups was irregular, we used the Random Forest machine
learning algorithm to identify groups of OTUs that could be used to
classify pre and post-treatment samples for each of the diagnosis
groups. Although the three models could differentiate between the pre
and post-treatment samples, there was little overlap between the OTUs
that were indicative of each treatment. To determine whether individuals
who underwent treatment were more likely to have OTUs associated with
normal colons we used a larger cohort that contained individuals with
normal colons and those with adenomas, advanced adenomas, and
carcinomas. We again built Random Forest models and measured the change
in the positive probability of having one of the three diagnoses to
assess whether the post-treatment samples received the same
classification as the pre-treatment samples. Samples from patients who
had carcinomas changed towards a microbial milieu that resembles the
normal colon after treatment (P-value \textless{} 0.05). Finally, we
were unable to detect any significant differences in the microbiota of
individuals treated with surgery alone and those treated with
chemotherapy or chemotherapy and radiation (P-value \textgreater{}
0.05).

\textbf{Conclusions.} By better understanding the response of the
microbiota to treatment for adenomas and carcinomas, it is likely that
biomarkers will eventually be validated that can be used to quantify the
risk of recurrence and the likelihood of survival. Although it was
difficult to identify significant differences between pre and
post-treatment samples from patients with adenoma and advanced adenoma,
this was not the case for carcinomas. Not only were there large changes
in pre versus post-treatment samples for those with carcinoma, but these
changes were towards a more normal microbiota.

\subsubsection{Keywords}\label{keywords}

microbiota; colorectal cancer; polyps; treatment; risk factor.

\newpage

\subsection{Background}\label{background}

Colorectal cancer (CRC) is the third most common cause of cancer deaths
in the United States {[}1,2{]}. Disease mortality has significantly
decreased, predominately due to improvements in screening {[}2{]}.
Despite these improvements, there are still approximately 50,000
CRC-related deaths per year in the United States {[}1{]}. Current
estimates indicate that 20-30\% of those who undergo treatment will
experience recurrence and 35\% of all patients will die within five
years {[}3--5{]}. Identification of methods to assess patients' risk of
recurrence is of great importance to reduce mortality and healthcare
costs.

There is growing evidence that the gut microbiota is involved in the
progression of CRC. Mouse-based studies have identified populations of
\emph{Bacteroides fragilis}, \emph{Escherichia coli}, and
\emph{Fusobacterium nucleatum} that alter disease progression
{[}6--10{]}. Furthermore, studies that shift the structure of the
microbiota through the use of antibiotics or inoculation of germ free
mice with human feces have shown that varying community compositions can
result in varied tumor burden {[}11--13{]}. Collectively, these studies
support the hypothesis that the microbiota can alter the amount of
inflammation in the colon and with it the rate of tumorigenesis
{[}14{]}.

Building upon this evidence, several human studies have identified
unique signatures of colonic lesions {[}15--20{]}. One line of research
has identified community-level differences between those bacteria that
are found on and adjacent to colonic lesions and have supported a role
for \emph{Bacteroides fragilis}, \emph{Escherichia coli}, and
\emph{Fusobacterium nucleatum} in tumorigenesis {[}21--23{]}. Others
have proposed feces-based biomarkers that could be used to diagnose the
presence of colonic adenomas and carcinomas {[}24--26{]}. These studies
have associated \emph{Fusobacterium nucleatum} and other oral pathogens
with colonic lesions (adenoma, advanced adenoma, and carcinoma). They
have also noted that the loss of bacteria generally thought to produce
short chain fatty acids, which can suppress inflammation, is associated
with colonic lesions. This suggests that gut bacteria have a role in
tumorigenesis with potential as useful biomarkers for aiding in the
early detection of disease {[}21--26{]}.

Despite advances in understanding the role between the gut microbiota
and colonic tumorigenesis, we still do not understand how treatments
including resection, chemotherapy, and/or radiation affect the
composition of the gut microbiota. If the microbial community drives
tumorigenesis then one would hypothesize that treatment to remove a
lesion would affect the microbiota and risk of recurrence. To test this
hypothesis, we addressed two related questions: Does treatment affect
the colonic microbiota in a predictable manner? If so, does the
treatment alter the community to more closely resemble that of
individuals with normal colons?

We answered these questions by sequencing the V4 region of 16S rRNA
genes amplified from fecal samples of individuals with adenoma, advanced
adenoma, and carcinomas pre and post-treatment. We used classical
community analysis to compare the alpha and beta-diversity of
communities pre and post-treatment. Next, we generated Random Forest
models to identify bacterial populations that were indicative of
treatment for each diagnosis group. Finally, we measured the predictive
probabilities to assess whether treatment yielded bacterial communities
similar to those individuals with normal colons. We found that treatment
alters the composition of the gut microbiota and that, for those with
carcinomas, the gut microbiota shifted more towards that of a normal
colon after treatment. In the individuals with carcinomas, no difference
was found by the type of treatment (surgery alone, surgery with
chemotherapy, surgery with chemotherapy and radiation). Understanding
how the community responds to these treatments could be a valuable tool
for identifying biomarkers to quantify the risk of recurrence and the
likelihood of survival.

\newpage

\subsection{Results}\label{results}

\textbf{\emph{Treatment for colonic lesions alters the bacterial
community structure}} Within our 67-person cohort we tested whether the
microbiota of patients with adenoma (N = 22), advanced adenoma (N = 19),
or carcinoma (N = 26) had any broad differences between pre and
post-treatment samples {[}Table 1{]}. None of the individuals in this
study had any recorded antibiotic usage that was not associated with
surgical treatment of their respective lesion. The structure of the
microbial communities of the pre and post-treatment samples differed, as
measured by the \(\theta\)\textsubscript{YC} beta diversity metric
{[}Figure 1A{]}. We found that the communities obtained pre and
post-treatment among the patients with carcinomas changed significantly
more than those patients with adenoma (P-value \textless{} 0.001). There
were no significant differences in the amount of change observed between
the patients with adenoma and advanced adenoma or between the patients
with advanced adenoma and carcinoma (P-value \textgreater{} 0.05). Next,
we tested whether there was a consistent direction in the change in the
community structure between the pre and post-treatment samples for each
of the diagnosis groups {[}Figure 1B-D{]}. We only observed a consistent
shift in community structure for the patients with carcinoma when using
a PERMANOVA test (adenoma P-value = 0.999, advanced adenoma P-value =
0.945, and carcinoma P-value = 0.005). Finally, we measured the number
of observed OTUs, Shannon evenness, and Shannon diversity in the pre and
post-treatment samples and did not observe a significant change for any
of the diagnosis groups (P-value \textgreater{} 0.05) {[}Table S1{]}.

\textbf{\emph{The treatment of lesions are not consistent across
diagnosis groups.}} We used two approaches to identify those bacterial
populations that change between the two samples for each diagnosis
group. First, we sought to identify individual OTUs that could account
for the change in overall community structure. However, using a paired
Wilcoxon test we were unable to identify any OTUs that were
significantly different in the pre and post-treatment groups (P-value
\textgreater{} 0.05). It is likely that high inter-individual variation
and the irregular distribution of OTUs across individuals limited the
statistical power of the test. We attempted to overcome these problems
by using Random Forest models to identify collections of OTUs that would
allow us to differentiate between pre and post-treatment samples from
each of the diagnosis groups. To limit the likelihood that the models
would over-fit the data because of the relatively small number of
subjects in each group, we restricted our models to only incorporate 10
OTUs. Despite this restriction, the models performed well (adenoma AUC
range = 0.69 - 0.92, advanced adenoma AUC range = 0.80 - 1.00, carcinoma
AUC range = 0.82 - 0.98). Interestingly, the 10 OTUs that were used for
each model had little overlap with each other {[}Figure 2{]}. Although
treatment had an impact on the overall community structure, the effect
of treatment was not consistent across patients and diagnosis groups.

\textbf{\emph{Post-treatment samples from patients with carcinoma more
closely resemble those of a normal colon.}} Next, we determined whether
treatment changed the microbiota in a way that the post-treatment
communities resembled that of patients with normal colons. To test this,
we used an expanded cohort of 423 individuals that were diagnosed under
the same protocol as having normal colons or colons with adenoma,
advanced adenoma, or carcinoma {[}Table 2{]}. We then constructed Random
Forest models to classify the study samples, with the 3 diagnosis groups
(adenoma, advanced adenoma, or carcinoma), or having a normal colon. The
models performed well (adenoma AUC range = 0.62 - 0.72, advanced adenoma
AUC range = 0.68 - 0.77, carcinoma AUC range = 0.84 - 0.90; Figure S1).
The OTUs that were incorporated into the adenoma and advanced adenoma
models largely overlapped and those OTUs that were used to classify the
carcinoma samples were largely distinct from those of the other two
models {[}Figure 3A{]}. Among the OTUs that were shared across the three
models were those populations generally considered beneficial to their
host (e.g. \emph{Faecalibacterium}, \emph{Lachnospiraceae},
\emph{Bacteroides}, \emph{Dorea}, \emph{Anaerostipes}, and
\emph{Roseburia}) {[}Figures 3B{]}. Although many of these OTUs were
also included in the model differentiating between patients with normal
colons and those with carcinoma, this model also included OTUs
affiliated with populations that have previously been associated with
carcinoma (\emph{Fusobacterium}, \emph{Porphyromonas},
\emph{Parvimonas}) {[}24--26{]} {[}Figure S2{]} with some individuals
showing a marked decrease in relative abundance {[}Figure S3{]}.
Finally, we applied these three models to the pre and post-treatment
samples for each diagnosis group and quantified the change in the
positive probability of the model. A decrease in the positive
probability would indicate that the microbiota more closely resembled
that of a patient with a normal colon. There was no significant change
in the positive probability for the adenoma or advanced adenoma groups
(P-value \textgreater{} 0.05) {[}Figure 4{]}. The positive probability
for the pre and post-treatment samples from patients diagnosed with
carcinoma significantly decreased with treatment, suggesting a shift
toward a normal microbiota for most individuals (P-value = 0.02). Only,
6 of the 26 patients (23.08\%) who were diagnosed with a carcinoma had a
higher positive probability after treatment; one of those was
re-diagnosed with carcinoma on the follow up visit. These results
indicate that, although there were changes in the microbiota associated
with treatment, those experienced by patients with carcinoma after
treatment yielded gut bacterial communities of greater similarity to
that of a normal colon.

\textbf{\emph{Difficult to identify effects of specific treatments on
the change in the microbiota.}} The type of treatment that the patients
received varied across diagnosis groups. Those with adenomas and
advanced adenomas received surgical resection (adenoma, N=4; advanced
adenoma, N=4) or polyp removal during colonoscopy (adenoma, N=18;
advanced adenoma, N=15) and those with carcinomas received surgical
resection (N=12), surgical resection with chemotherapy (N=9), and
surgical resection with chemotherapy and radiation (N=5). Regardless of
treatment used there was no significant difference in the effect of
these treatments on the number of observed OTUs, Shannon diversity, or
Shannon evenness (P-value \textgreater{} 0.05). Furthermore, there was
not a significant difference in the effect of the treatments on the
amount of change in the community structure (P-value = 0.532). Finally,
the change in the positive probability was not significantly different
between any of the treatment groups (P-value = 0.532). Due to the
relatively small number of samples in each treatment group, it was
difficult to make a definitive statement regarding the specific type of
treatment on the amount of change in the structure of the microbiota.

\newpage

\subsection{Discussion}\label{discussion}

Our study focused on comparing the microbiota of patients diagnosed with
adenoma, advanced adenoma, and carcinoma before and after treatment. For
all three groups of patients, we observed changes in their microbiota.
Some of these changes, specifically for adenoma, may be due to normal
temporal variation, however, those with advanced adenoma and carcinoma
clearly had large microbiota changes. After treatment, the microbiota of
patients with carcinoma changed significantly more than the other
groups. This change resulted in communities that more closely resembled
those of patients with a normal colon. This may suggest that treatment
for carcinoma is not only successful for removing the carcinoma but also
at reducing the associated bacterial communities. Understanding the
effect of treatment on the microbiota of those diagnosed with carcinomas
may have important implications for reducing disease recurrence. It is
intriguing that it may be possible to use microbiome-based biomarkers to
not only predict the presence of lesions but to also assess the risk of
recurrence due to these changes in the microbiota.

Patients diagnosed with adenoma and advanced adenoma, however, did not
experience a shift towards a community structure that resembled those
with normal colons. This may be due to the fundamental differences
between the features of adenomas and advanced adenomas and carcinoma.
Specifically, carcinomas may create an inflammatory milieu that would
impact the structure of the community and removal of that stimulus would
alter said structure. It is possible that the difference between the
microbiota of patients with adenoma and advanced adenoma and those with
normal colons is subtle. This is supported by the reduced ability of our
models to correctly classify patients with adenomas and advanced
adenomas relative to those diagnosed with carcinomas {[}Figure S1{]}.
Given the irregular distribution of microbiota across patients in the
different diagnosis groups, it is possible that we lacked the
statistical power to adequately characterize the change in the
communities following treatment.

There was a subset of patients (6 of the 26 with carcinomas) who
demonstrated an elevated probability of carcinoma after treatment. This
may reflect an elevated risk of recurrence. The 23.08\% prevalence of
increased carcinoma probability from our study is within the expected
rate of recurrence (20-30\% {[}3,4{]}). We hypothesized that these
individuals may have had more severe tumors; however, the tumor severity
of these 6 individuals (3 with Stage II and 3 with Stage III) was
similar to the distribution observed among the other 20 patients. We
also hypothesized that we may have sampled these patients later than the
rest and their communities may have reverted to a carcinoma-associated
state; however, there was not a statistically significant difference in
the length of time between sample collection among those whose
probabilities increased (358 (336 - 458) days) or decreased (334 (256 -
399) days) (Wilcoxon Test; P-value = 0.56) (all days data displayed as
median (IQR)). Finally, it is possible that these patients may not have
responded to treatment as well as the other 20 patients diagnosed with
carcinoma and so the microbiota may not have been impacted the same way.
Again, further studies looking at the role of the microbiota in
recurrence are needed to understand the dynamics following treatment.

Our final hypothesis was that the specific type of treatment altered the
structure of the microbiome. The treatment to remove adenomas and
advanced adenomas was either polyp removal or surgical resection whereas
it was surgical resection alone or in combination with chemotherapy or
with chemotherapy and radiation for individuals with carcinoma. Because
chemotherapy and radiation target rapidly growing cells, these
treatments would be more likely to cause a turnover of the colonic
epithelium driving a more significant change in the structure of the
microbiota. Although, we were able to test for an effect across these
specific types of treatment, the number of patients in each treatment
group was relatively small. Finally, those undergoing surgery would have
received antibiotics and this may be a potential confounder. However,
our pre-treatment stool samples were obtained before the surgery and the
post-treatment samples were obtained long after any effects due to
antibiotic administration on the microbiome would be expected to occur
(344 (266 - 408) days). We also found no difference in the community
structure of those that received surgery and those that did not as a
treatment for adenoma or advanced adenoma.

\subsection{Conclusion}\label{conclusion}

This study expands upon existing research that has established a role
for the microbiota in tumorigenesis and that demonstrated the utility of
microbiome-based biomarkers to predict the presence of colonic lesions.
We were surprised by the lack of a consistent signal that was associated
with treatment of patients with adenomas or advanced adenomas. The lack
of a large effect size may be due to differences in the role of bacteria
in the formation of adenomas and carcinomas or it could be due to
differences in the behaviors and medications within these classes of
patients. One of the most exciting of these future directions is the
possibility that markers within the microbiota could be used to
potentially evaluate the effect of treatment and predict recurrence for
those diagnosed with carcinoma. If such an approach is effective, it
might be possible to target the microbiota as part of adjuvant therapy,
if the biomarkers identified play a key role in the disease process. Our
data provides additional evidence on the importance of the microbiota in
tumorigenesis by addressing the recovery of the microbiota after
treatment and opens interesting avenues of research into how these
changes may affect recurrence.

\newpage

\subsection{Methods}\label{methods}

\textbf{\emph{Study Design and Patient Sampling.}} Sampling and design
have been previously reported in Baxter, et al {[}24{]}. Briefly,
samples were stored on ice for at least 24h before freezing. Although we
cannot exclude that this sampling protocol may have impacted the gut
microbiota composition all samples were subjected to the same
methodology. Study exclusion involved those who had already undergone
surgery, radiation, or chemotherapy, had colorectal cancer before a
baseline fecal sample could be obtained, had IBD, a known hereditary
non-polyposis colorectal cancer, or familial adenomatous polyposis.
Samples used to build the models for prediction were collected either
prior to a colonoscopy or between one and two weeks after initial
colonoscopy. The bacterial community has been shown to normalize back to
a pre-colonoscopy community within this time period {[}27{]}. Our study
cohort consisted of 67 individuals with an initial sample as described
and a follow up sample obtained between 188 - 546 days after treatment
of lesion {[}Table 1{]}. Patients were diagnosed by colonoscopic
examination and histopathological review of any biopsies taken. Patients
were classified as having advanced adenoma if they had an adenoma
greater than 1 cm, more than three adenomas of any size, or an adenoma
with villous histology. This study was approved by the University of
Michigan Institutional Review Board. All study participants provided
informed consent and the study itself conformed to the guidelines set
out by the Helsinki Declaration.

\textbf{\emph{Treatment.}} For this study treatment refers specifically
to the removal of a lesion with or without chemotherapy and radiation.
The majority of patients undergoing treatment for adenoma or advanced
adenoma were not treated surgically {[}Table 1{]} but rather via
colonoscopy. All patients diagnosed with carcinomas were treated with at
least surgery or a combination of surgery and chemotherapy or surgery,
chemotherapy, and radiation. The type of chemotherapy used for patients
with CRC included Oxaliplatin, Levicovorin, Folfox, Xeloda,
Capecitabine, Avastin, Fluorouracil, and Glucovorin. These were used
individually or in combination with others depending on the patient
{[}Table 1{]}. If an individual was treated with radiation they were
also always treated with chemotherapy. Radiation therapy generally used
18 mV photons for treatment.

\textbf{\emph{16S rRNA Gene Sequencing.}} Sequencing was completed as
described by Kozich, et al. {[}28{]}. DNA extraction used the 96-well
Soil DNA isolation kit (MO BIO Laboratories) and an epMotion 5075
automated pipetting system (Eppendorf). The V4 variable region was
amplified and the resulting product was split between four sequencing
runs with normal, adenoma, and carcinoma evenly represented on each run.
Each group was randomly assigned to avoid biases based on sample
collection location. The pre and post-treatment samples were sequenced
on the same run.

\textbf{\emph{Sequence Processing.}} The mothur software package
(v1.37.5) was used to process the 16S rRNA gene sequences and has been
previously described {[}28{]}. The general workflow using mothur
included merging paired-end reads into contigs, filtering for low
quality contigs, aligning to the SILVA database {[}29{]}, screening for
chimeras using UCHIME {[}30{]}, classifying with a naive Bayesian
classifier using the Ribosomal Database Project (RDP){[}31{]}, and
clustered into Operational Taxonomic Units (OTUs) using a 97\%
similarity cutoff with an average neighbor clustering algorithm
{[}32{]}. The number of sequences for each sample was rarefied to 10523
to minimize the impacts of uneven sampling.

\textbf{\emph{Model Building.}} The Random Forest {[}33{]} algorithm was
used to create the three models used to classify pre and post-treatment
samples by diagnosis (adenoma, advanced adenoma, or carcinoma). The
total number of individuals in the pre versus post-treatment models was
67 individuals. There were a total of 22 individuals in the pre versus
post-treatment adenoma model, 19 individuals in the pre versus
post-treatment advanced adenoma model, and 26 individuals in the pre
versus post-treatment carcinoma model {[}Table 1{]}.

Similarly, the Random Forest {[}33{]} algorithm was also used to create
the three models used to classify normal versus diagnosis. These samples
were obtained using the same methodology as described earlier in this
section. All samples used for this component of model training were from
pre-treatment samples only. The total number of individuals in the
normal versus diagnosis models was 423 individuals {[}Table 2{]}. There
were a total of 239 individuals in the normal versus adenoma model, 262
individuals in the normal versus advanced adenoma model, and 266
individuals in the normal versus carcinoma model {[}Table 2{]}.

All models included only OTU data obtained from 16S rRNA sequencing and
were processed and cleaned using the R package caret (v6.0.76).
Optimization of the mtry hyper-parameter involved making 100 different
80/20 (train/test) splits of the data where the same proportion was
present within both the whole data set and the 80/20 split. For each of
the different splits, 20 repeated 10-fold cross validation was performed
on the 80\% component to optimize the mtry hyper-parameter by maximizing
the AUC (Area Under the Curve of the Receiver Operator Characteristic).
The resulting model was then tested on the hold out data obtained from
the 20\% component. For all pre versus post-treatment models the
optimized mtry was 2 and for all normal versus diagnosis models the
optimized mtry was 2. The hyper-parameter, mtry, defines the number of
variables to investigate at each split before a new division of the data
was created with the Random Forest model {[}33{]}.

For each of the pre versus post-treatment models assessment of the most
important OTUs was then made by taking the top 10 OTUs by mean decrease
in accuracy (MDA). These were then used to build each respective reduced
OTU pre versus post-treatment model by diagnosis group to help avoid
model over-fitting. These reduced models were then put through the same
process mentioned in the previous paragraph and were what was used for
the final pre versus post-treatment models. For the normal versus
diagnosis models the important OTUs were obtained by counting the number
of times an OTU was present in the top 10\% of MDA for each of the 100
different splits. This was then followed with filtering of this list to
variables that were only present in more than 50\% of these 100 runs.
These corresponding reduced OTU normal versus diagnosis models were then
put through the same process mentioned in the previous paragraph and
were what was used for the final normal versus diagnosis models. For the
pre versus post-treatment models the final optimized mtry was 2 and for
the normal versus diagnosis models the final optimized mtry was 2.

Each model was then applied to our 67-person cohort {[}Table 1{]} based
on diagnosis: adenoma (pre-treatment adenoma (adenoma n = 22 and disease
free n = 0) versus post-treatment adenoma (adenoma n = 0 and disease
free n = 22)), advanced adenoma pre-treatment advanced adenoma (advanced
adenoma n = 19 and disease free n = 0 ) versus post-treatment advanced
adenoma (advanced adenoma n = 0 and disease free n = 19), and carcinoma
(pre-treatment carcinoma (carcinoma n = 26 and disease free n = 0)
versus post-treatment carcinoma (carcinoma n = 1 and disease free n =
25)). The application of the pre versus post-treatment models generated
the probabilities that the sample was a pre-treatment sample. The
application of the normal versus diagnosis models generated the
probabilities that the sample was that specific diagnosis (adenoma,
advanced adenoma, or carcinoma).

\textbf{\emph{Statistical Analysis.}} The R software package (v3.4.1)
was used for all statistical analysis. Comparisons between bacterial
community structure utilized PERMANOVA {[}34{]} in the vegan package
(v2.4.3). Comparisons between probabilities as well as overall
differences in the median relative abundance of each OTU between pre and
post-treatment samples utilized a paired Wilcoxon ranked sum test.
Comparisons between different treatment for lesions utilized a Kruskal
Wallis test. Where multiple comparison testing was appropriate, a
Benjamini-Hochberg (BH) correction was applied {[}35{]} and a corrected
P-value of less than 0.05 was considered significant. The P-values
reported are those that were BH corrected. Model rank importance was
determined by obtaining the median MDA from the 100, 20 repeated 10-fold
cross validation and then ranking from largest to smallest MDA.

\textbf{\emph{Reproducible Methods.}} A detailed and reproducible
description of how the data were processed and analyzed can be found at
\url{https://github.com/SchlossLab/Sze_followUps_2017}. Raw sequences
have been deposited into the NCBI Sequence Read Archive (SRP062005 and
SRP096978) and the necessary metadata can be found at
\url{https://www.ncbi.nlm.nih.gov/Traces/study/} and searching the
respective SRA study accession.

\newpage

\subsection{Declarations}\label{declarations}

\subsubsection{Ethics approval and consent to
participate}\label{ethics-approval-and-consent-to-participate}

The University of Michigan Institutional Review Board approved this
study, and all subjects provided informed consent. This study conformed
to the guidelines of the Helsinki Declaration.

\subsubsection{Consent for publication}\label{consent-for-publication}

Not applicable.

\subsubsection{Availability of data and
material}\label{availability-of-data-and-material}

A detailed and reproducible description of how the data were processed
and analyzed can be found at
\url{https://github.com/SchlossLab/Sze_followUps_2017}. Raw sequences
have been deposited into the NCBI Sequence Read Archive (SRP062005 and
SRP096978) and the necessary metadata can be found at
\url{https://www.ncbi.nlm.nih.gov/Traces/study/} and searching the
respective SRA study accession.

\subsubsection{Competing Interests}\label{competing-interests}

All authors declare that they do not have any relevant competing
interests to report.

\subsubsection{Funding}\label{funding}

This study was supported by funding from the National Institutes of
Health to P. Schloss (R01GM099514, P30DK034933) and to the Early
Detection Research Network (U01CA86400).

\subsubsection{Authors' contributions}\label{authors-contributions}

All authors were involved in the conception and design of the study. MAS
analyzed the data. NTB processed samples and analyzed the data. All
authors interpreted the data. MAS and PDS wrote the manuscript. All
authors reviewed and revised the manuscript. All authors read and
approved the final manuscript.

\subsubsection{Acknowledgements}\label{acknowledgements}

The authors thank the Great Lakes-New England Early Detection Research
Network for providing the fecal samples that were used in this study. We
would also like to thank Amanda Elmore for reviewing and correcting code
error and providing feedback on manuscript drafts. We would also like to
thank Nicholas Lesniak for providing feedback on manuscript drafts.

\newpage

\subsection{References}\label{references}

\hypertarget{refs}{}
\hypertarget{ref-siegel_cancer_2016}{}
1. Siegel RL, Miller KD, Jemal A. Cancer statistics, 2016. CA: a cancer
journal for clinicians. 2016;66:7--30.

\hypertarget{ref-haggar_colorectal_2009}{}
2. Haggar FA, Boushey RP. Colorectal cancer epidemiology: Incidence,
mortality, survival, and risk factors. Clinics in Colon and Rectal
Surgery. 2009;22:191--7.

\hypertarget{ref-hellinger_reoperation_2006}{}
3. Hellinger MD, Santiago CA. Reoperation for recurrent colorectal
cancer. Clinics in Colon and Rectal Surgery. 2006;19:228--36.

\hypertarget{ref-ryuk_predictive_2014}{}
4. Ryuk JP, Choi G-S, Park JS, Kim HJ, Park SY, Yoon GS, et al.
Predictive factors and the prognosis of recurrence of colorectal cancer
within 2 years after curative resection. Annals of Surgical Treatment
and Research. 2014;86:143--51.

\hypertarget{ref-national_cancer_institute_seer_nodate}{}
5. Institute NC. SEER Cancer Stat Facts: Colon and Rectum Cancer
{[}Internet{]}. {[}cited 2017 Apr 27{]}. Available from:
\url{http://seer.cancer.gov/statfacts/html/colorect.html}

\hypertarget{ref-goodwin_polyamine_2011}{}
6. Goodwin AC, Destefano Shields CE, Wu S, Huso DL, Wu X, Murray-Stewart
TR, et al. Polyamine catabolism contributes to enterotoxigenic
Bacteroides fragilis-induced colon tumorigenesis. Proceedings of the
National Academy of Sciences of the United States of America.
2011;108:15354--9.

\hypertarget{ref-abed_fap2_2016}{}
7. Abed J, Emgård JEM, Zamir G, Faroja M, Almogy G, Grenov A, et al.
Fap2 Mediates Fusobacterium nucleatum Colorectal Adenocarcinoma
Enrichment by Binding to Tumor-Expressed Gal-GalNAc. Cell Host \&
Microbe. 2016;20:215--25.

\hypertarget{ref-arthur_microbial_2014}{}
8. Arthur JC, Gharaibeh RZ, Mühlbauer M, Perez-Chanona E, Uronis JM,
McCafferty J, et al. Microbial genomic analysis reveals the essential
role of inflammation in bacteria-induced colorectal cancer. Nature
Communications. 2014;5:4724.

\hypertarget{ref-kostic_fusobacterium_2013}{}
9. Kostic AD, Chun E, Robertson L, Glickman JN, Gallini CA, Michaud M,
et al. Fusobacterium nucleatum potentiates intestinal tumorigenesis and
modulates the tumor-immune microenvironment. Cell Host \& Microbe.
2013;14:207--15.

\hypertarget{ref-wu_human_2009}{}
10. Wu S, Rhee K-J, Albesiano E, Rabizadeh S, Wu X, Yen H-R, et al. A
human colonic commensal promotes colon tumorigenesis via activation of T
helper type 17 T cell responses. Nature Medicine. 2009;15:1016--22.

\hypertarget{ref-zackular_manipulation_2016}{}
11. Zackular JP, Baxter NT, Chen GY, Schloss PD. Manipulation of the Gut
Microbiota Reveals Role in Colon Tumorigenesis. mSphere. 2016;1.

\hypertarget{ref-zackular_gut_2013}{}
12. Zackular JP, Baxter NT, Iverson KD, Sadler WD, Petrosino JF, Chen
GY, et al. The gut microbiome modulates colon tumorigenesis. mBio.
2013;4:e00692--00613.

\hypertarget{ref-baxter_structure_2014}{}
13. Baxter NT, Zackular JP, Chen GY, Schloss PD. Structure of the gut
microbiome following colonization with human feces determines colonic
tumor burden. Microbiome. 2014;2:20.

\hypertarget{ref-flynn_metabolic_2016}{}
14. Flynn KJ, Baxter NT, Schloss PD. Metabolic and Community Synergy of
Oral Bacteria in Colorectal Cancer. mSphere. 2016;1.

\hypertarget{ref-wang_structural_2012}{}
15. Wang T, Cai G, Qiu Y, Fei N, Zhang M, Pang X, et al. Structural
segregation of gut microbiota between colorectal cancer patients and
healthy volunteers. The ISME journal. 2012;6:320--9.

\hypertarget{ref-chen_decreased_2013}{}
16. Chen H-M, Yu Y-N, Wang J-L, Lin Y-W, Kong X, Yang C-Q, et al.
Decreased dietary fiber intake and structural alteration of gut
microbiota in patients with advanced colorectal adenoma. The American
Journal of Clinical Nutrition. 2013;97:1044--52.

\hypertarget{ref-chen_human_2012}{}
17. Chen W, Liu F, Ling Z, Tong X, Xiang C. Human intestinal lumen and
mucosa-associated microbiota in patients with colorectal cancer. PloS
One. 2012;7:e39743.

\hypertarget{ref-shen_molecular_2010}{}
18. Shen XJ, Rawls JF, Randall T, Burcal L, Mpande CN, Jenkins N, et al.
Molecular characterization of mucosal adherent bacteria and associations
with colorectal adenomas. Gut Microbes. 2010;1:138--47.

\hypertarget{ref-kostic_genomic_2012}{}
19. Kostic AD, Gevers D, Pedamallu CS, Michaud M, Duke F, Earl AM, et
al. Genomic analysis identifies association of Fusobacterium with
colorectal carcinoma. Genome Research. 2012;22:292--8.

\hypertarget{ref-feng_gut_2015}{}
20. Feng Q, Liang S, Jia H, Stadlmayr A, Tang L, Lan Z, et al. Gut
microbiome development along the colorectal adenoma-carcinoma sequence.
Nature Communications. 2015;6:6528.

\hypertarget{ref-dejea_microbiota_2014}{}
21. Dejea CM, Wick EC, Hechenbleikner EM, White JR, Mark Welch JL,
Rossetti BJ, et al. Microbiota organization is a distinct feature of
proximal colorectal cancers. Proceedings of the National Academy of
Sciences of the United States of America. 2014;111:18321--6.

\hypertarget{ref-mima_fusobacterium_2015}{}
22. Mima K, Sukawa Y, Nishihara R, Qian ZR, Yamauchi M, Inamura K, et
al. Fusobacterium nucleatum and T Cells in Colorectal Carcinoma. JAMA
oncology. 2015;1:653--61.

\hypertarget{ref-arthur_intestinal_2012}{}
23. Arthur JC, Perez-Chanona E, Mühlbauer M, Tomkovich S, Uronis JM, Fan
T-J, et al. Intestinal inflammation targets cancer-inducing activity of
the microbiota. Science (New York, N.Y.). 2012;338:120--3.

\hypertarget{ref-baxter_microbiota-based_2016}{}
24. Baxter NT, Ruffin MT, Rogers MAM, Schloss PD. Microbiota-based model
improves the sensitivity of fecal immunochemical test for detecting
colonic lesions. Genome Medicine. 2016;8:37.

\hypertarget{ref-zeller_potential_2014}{}
25. Zeller G, Tap J, Voigt AY, Sunagawa S, Kultima JR, Costea PI, et al.
Potential of fecal microbiota for early-stage detection of colorectal
cancer. Molecular Systems Biology. 2014;10:766.

\hypertarget{ref-zackular_human_2014}{}
26. Zackular JP, Rogers MAM, Ruffin MT, Schloss PD. The human gut
microbiome as a screening tool for colorectal cancer. Cancer Prevention
Research (Philadelphia, Pa.). 2014;7:1112--21.

\hypertarget{ref-obrien_impact_2013}{}
27. O'Brien CL, Allison GE, Grimpen F, Pavli P. Impact of colonoscopy
bowel preparation on intestinal microbiota. PloS One. 2013;8:e62815.

\hypertarget{ref-kozich_development_2013}{}
28. Kozich JJ, Westcott SL, Baxter NT, Highlander SK, Schloss PD.
Development of a dual-index sequencing strategy and curation pipeline
for analyzing amplicon sequence data on the MiSeq Illumina sequencing
platform. Applied and Environmental Microbiology. 2013;79:5112--20.

\hypertarget{ref-pruesse_silva_2007}{}
29. Pruesse E, Quast C, Knittel K, Fuchs BM, Ludwig W, Peplies J, et al.
SILVA: A comprehensive online resource for quality checked and aligned
ribosomal RNA sequence data compatible with ARB. Nucleic Acids Research.
2007;35:7188--96.

\hypertarget{ref-edgar_uchime_2011}{}
30. Edgar RC, Haas BJ, Clemente JC, Quince C, Knight R. UCHIME improves
sensitivity and speed of chimera detection. Bioinformatics (Oxford,
England). 2011;27:2194--200.

\hypertarget{ref-wang_naive_2007}{}
31. Wang Q, Garrity GM, Tiedje JM, Cole JR. Naive Bayesian classifier
for rapid assignment of rRNA sequences into the new bacterial taxonomy.
Applied and Environmental Microbiology. 2007;73:5261--7.

\hypertarget{ref-schloss_assessing_2011}{}
32. Schloss PD, Westcott SL. Assessing and improving methods used in
operational taxonomic unit-based approaches for 16S rRNA gene sequence
analysis. Applied and Environmental Microbiology. 2011;77:3219--26.

\hypertarget{ref-breiman_random_2001}{}
33. Breiman L. Random Forests. Machine Learning. 2001;45:5--32.

\hypertarget{ref-anderson_permanova_2013}{}
34. Anderson MJ, Walsh DCI. PERMANOVA, ANOSIM, and the Mantel test in
the face of heterogeneous dispersions: What null hypothesis are you
testing? Ecological Monographs. 2013;83:557--74.

\hypertarget{ref-benjamini_controlling_1995}{}
35. Benjamini Y, Hochberg Y. Controlling the false discovery rate: A
practical and powerful approach to multiple testing. Journal of the
Royal Statistical Society. Series B (Methodological). 1995;57:289--300.

\newpage

\textbf{Table 1: Demographic data of patients in the pre and
post-treatment cohort}

\begin{longtable}[]{@{}lccc@{}}
\toprule
& Adenoma & Advanced Adenoma & Carcinoma\tabularnewline
\midrule
\endhead
n & 22 & 19 & 26\tabularnewline
Age (Mean ± SD) & 61.68 ± 7.2 & 63.11 ± 10.9 & 61.65 ±
12.9\tabularnewline
Sex (\%F) & 36.36 & 36.84 & 42.31\tabularnewline
BMI (Mean ± SD) & 26.86 ± 3.9 & 25.81 ± 4.7 & 28.63 ± 7.2\tabularnewline
Caucasian (\%) & 95.45 & 84.21 & 96.15\tabularnewline
Days Between Colonoscopy (Mean ± SD) & 255.41 ± 42 & 250.16 ± 41 &
350.85 ± 102\tabularnewline
Surgery Only & 4 & 4 & 12\tabularnewline
Surgery \& Chemotherapy & 0 & 0 & 9\tabularnewline
Surgery, Chemotherapy, \& Radiation & 0 & 0 & 5\tabularnewline
\bottomrule
\end{longtable}

\newpage

\textbf{Table 2: Demographic data of training cohort}

\begin{longtable}[]{@{}lcccc@{}}
\toprule
& Normal & Adenoma & Advanced Adenoma & Carcinoma\tabularnewline
\midrule
\endhead
n & 172 & 67 & 90 & 94\tabularnewline
Age (Mean ± SD) & 54.29 ± 9.9 & 63.01 ± 13.1 & 64.07 ± 11.3 & 64.37 ±
12.9\tabularnewline
Sex (\%F) & 64.53 & 46.27 & 37.78 & 43.62\tabularnewline
BMI (Mean ± SD) & 26.97 ± 5.3 & 25.69 ± 4.8 & 26.66 ± 4.9 & 29.27 ±
6.7\tabularnewline
Caucasian (\%) & 87.79 & 92.54 & 92.22 & 94.68\tabularnewline
\bottomrule
\end{longtable}

\newpage

\textbf{Figure 1: General differences between adenoma, advanced adenoma,
and carcinoma groups after treatment.} A) \(\theta\)\textsubscript{YC}
distances from pre versus post sample within each individual. A
significant difference was found between the adenoma and carcinoma group
(P-value = 5.36e-05). Solid black points represent the median value for
each diagnosis group. B) NMDS of the pre and post-treatment samples for
the adenoma group. C) NMDS of the pre and post-treatment samples for the
advanced adenoma group. D) NMDS of the pre and post-treatment samples
for the carcinoma group.

\textbf{Figure 2: The 10 OTUs used to classify treatment for adenoma,
advanced adenoma, and carcinoma.} A) Adenoma OTUs. B) Advanced Adenoma
OTUs. C) Carcinoma OTUs. The darker circle highlights the median log10
MDA value obtained from 100 different 80/20 splits while the lighter
colored circles represents the value obtained for a specific run.

\textbf{Figure 3: OTUs common to those models used to differentiate
between patients with normal colons and those with adenoma, advanced
adenoma, and carcinoma.} A) Venn diagram showing the OTU overlap between
each model. B) For each common OTU the lowest taxonomic identification
and importance rank for each model run is shown.

\textbf{Figure 4: Treatment response based on models built for adenoma,
advanced adenoma, or carcinoma.} A) Positive probability change from
initial to follow up sample in those with adenoma. B) Positive
probability change from initial to follow up sample in those with
advanced adenoma. C) Positive probability change from initial to follow
up sample in those with carcinoma.

\newpage

\textbf{Figure S1: ROC curves of the adenoma, advanced adenoma, and
carcinoma models.} A) Adenoma ROC curve: The light green shaded areas
represent the range of values of a 100 different 80/20 splits of the
test set data and the dark green line represents the model using 100\%
of the data set and what was used for subsequent classification. B)
Advanced Adenoma ROC curve: The light yellow shaded areas represent the
range of values of a 100 different 80/20 splits of the test set data and
the dark yellow line represents the model using 100\% of the data set
and what was used for subsequent classification. C) Carcinoma ROC curve:
The light red shaded areas represent the range of values of a 100
different 80/20 splits of the test set data and the dark red line
represents the model using 100\% of the data set and what was used for
subsequent classification.

\textbf{Figure S2: Summary of important OTUs for the adenoma, advanced
adenoma, and carcinoma models.} A) MDA of the most important variables
in the adenoma model. The dark green point represents the mean and the
lighter green points are the value of each of the 100 different runs. B)
Summary of Important Variables in the advanced adenoma model. MDA of the
most important variables in the SRN model. The dark yellow point
represents the mean and the lighter yellow points are the value of each
of the 100 different runs. C) MDA of the most important variables in the
carcinoma model. The dark red point represents the mean and the lighter
red points are the value of each of the 100 different runs.

\textbf{Figure S3: Pre and post-treatment relative abundance of CRC
associated OTUs within the carcinoma model.}

\newpage


\end{document}
