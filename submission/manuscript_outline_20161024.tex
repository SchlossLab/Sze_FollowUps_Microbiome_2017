\documentclass[12pt,]{article}
\usepackage{lmodern}
\usepackage{amssymb,amsmath}
\usepackage{ifxetex,ifluatex}
\usepackage{fixltx2e} % provides \textsubscript
\ifnum 0\ifxetex 1\fi\ifluatex 1\fi=0 % if pdftex
  \usepackage[T1]{fontenc}
  \usepackage[utf8]{inputenc}
\else % if luatex or xelatex
  \ifxetex
    \usepackage{mathspec}
  \else
    \usepackage{fontspec}
  \fi
  \defaultfontfeatures{Ligatures=TeX,Scale=MatchLowercase}
\fi
% use upquote if available, for straight quotes in verbatim environments
\IfFileExists{upquote.sty}{\usepackage{upquote}}{}
% use microtype if available
\IfFileExists{microtype.sty}{%
\usepackage{microtype}
\UseMicrotypeSet[protrusion]{basicmath} % disable protrusion for tt fonts
}{}
\usepackage[margin=1.0in]{geometry}
\usepackage{hyperref}
\hypersetup{unicode=true,
            pdfborder={0 0 0},
            breaklinks=true}
\urlstyle{same}  % don't use monospace font for urls
\usepackage{graphicx,grffile}
\makeatletter
\def\maxwidth{\ifdim\Gin@nat@width>\linewidth\linewidth\else\Gin@nat@width\fi}
\def\maxheight{\ifdim\Gin@nat@height>\textheight\textheight\else\Gin@nat@height\fi}
\makeatother
% Scale images if necessary, so that they will not overflow the page
% margins by default, and it is still possible to overwrite the defaults
% using explicit options in \includegraphics[width, height, ...]{}
\setkeys{Gin}{width=\maxwidth,height=\maxheight,keepaspectratio}
\IfFileExists{parskip.sty}{%
\usepackage{parskip}
}{% else
\setlength{\parindent}{0pt}
\setlength{\parskip}{6pt plus 2pt minus 1pt}
}
\setlength{\emergencystretch}{3em}  % prevent overfull lines
\providecommand{\tightlist}{%
  \setlength{\itemsep}{0pt}\setlength{\parskip}{0pt}}
\setcounter{secnumdepth}{0}
% Redefines (sub)paragraphs to behave more like sections
\ifx\paragraph\undefined\else
\let\oldparagraph\paragraph
\renewcommand{\paragraph}[1]{\oldparagraph{#1}\mbox{}}
\fi
\ifx\subparagraph\undefined\else
\let\oldsubparagraph\subparagraph
\renewcommand{\subparagraph}[1]{\oldsubparagraph{#1}\mbox{}}
\fi

%%% Use protect on footnotes to avoid problems with footnotes in titles
\let\rmarkdownfootnote\footnote%
\def\footnote{\protect\rmarkdownfootnote}

%%% Change title format to be more compact
\usepackage{titling}

% Create subtitle command for use in maketitle
\newcommand{\subtitle}[1]{
  \posttitle{
    \begin{center}\large#1\end{center}
    }
}

\setlength{\droptitle}{-2em}
  \title{}
  \pretitle{\vspace{\droptitle}}
  \posttitle{}
  \author{}
  \preauthor{}\postauthor{}
  \date{}
  \predate{}\postdate{}

\usepackage{helvet} % Helvetica font
\renewcommand*\familydefault{\sfdefault} % Use the sans serif version of the font
\usepackage[T1]{fontenc}

\usepackage[none]{hyphenat}

\usepackage{setspace}
\doublespacing
\setlength{\parskip}{1em}

\usepackage{lineno}

\usepackage{pdfpages}
\usepackage{comment}
\usepackage{lscape}

\begin{document}

\section{The Fecal Microbiome Before and After Treatment for Colorectal
Adenoma or
Carcinoma}\label{the-fecal-microbiome-before-and-after-treatment-for-colorectal-adenoma-or-carcinoma}

\vspace{25mm}

\begin{center}
Running Title: Human Microbiome before and after Colorectal Cancer

\vspace{10mm}

Marc A Sze${^1}$, Nielson T Baxter${^2}$, Mack T Ruffin IV${^3}$, Mary AM Rogers${^2}$, and Patrick D Schloss${^1}$${^\dagger}$

\vspace{20mm}

$\dagger$ To whom correspondence should be addressed: pschloss@umich.edu

$1$ Department of Microbiology and Immunology, University of Michigan, Ann Arbor, MI

$2$ Department of Internal Medicine, University of Michigan, Ann Arbor, MI   

$3$ Department of Family Medicine and Community Medicine, Penn State Hershey Medical Center, Hershey, PA    


\end{center}

\newpage

\linenumbers

\subsection{Abstract}\label{abstract}

\textbf{Background:} Colorectal cancer (CRC) is a worldwide health
problem and research suggests a correlation between the fecal bacterial
microbiome and CRC. This study tested the hypothesis that treatment for
adenoma or carcinoma result in changes to the bacterial community.
Specifically, we tried to identify components within the community that
were different before and after removal of lesion (adenoma or
carcinoma).

\textbf{Results:} There was a larger change in the bacterial community
in response to treatment for carcinoma versus adenoma cases (P-value
\textless{} 0.05). Yet no difference was found in the relative abundance
of any OTU before and after treatment for adenoma or carcinoma groups
(P-value \textgreater{} 0.05). A lesion model had an AUC range of 0.692
- 0.761 and follow up samples had no difference in the positive
probability of lesion versus initial samples (P-value \textgreater{}
0.05); suggesting that the lesion associated community persists after
treatment. A treatment model had an AUC range of 0.657 - 0.796 and had a
decrease positive probability for the follow up samples to be an initial
sample (P-value \textless{} 0.05); suggesting that there are members
within the community that repond to treatment. The lesion model used a
total of 54 variables while the initial sample model used a total of 70
variables. A total of 32 OTUs were common to both models with many
classifying to commensal bacteria (e.g. \emph{Lachnospiraceae},
\emph{Bacteroides}, \emph{Roseburia}, \emph{Blautia}, and
\emph{Ruminococcus}).

\textbf{Conclusions:} Our data partially supports the hypothesis that
the bacterial community changes after treatment. Individuals with
carcinoma have more drastic differences to the overall community then
those with adenoma. Commensal bacteria were crucial for accurate model
classification, suggesting that these bacteria may be important to
initial polyp formation and transition to carcinoma.

\newpage

\subsubsection{Keywords}\label{keywords}

bacterial microbiome; colorectal cancer; polyps; FIT; post-surgery; risk
factors

\newpage

\subsection{Background}\label{background}

Colorectal cancer (CRC) is currently the third most common cause of
cancer deaths {[}1,2{]}. The rate of disease mortality has seen a
significant decrease, thanks mainly to improvements in screening
{[}1{]}. However, despite this improvement there are still approximately
50,000 deaths from the disease per year {[}2{]}.

Recent studies in humans have shown that both the bacterial community
and specific members within it correlate with CRC pathogenesis
{[}3,4{]}. Further, Dejea, et al. observed that bacterial communities
are altered between normal and tumor tissue {[}5{]}. Mouse models of CRC
have further demonstrated the importance of the microbiome, both on a
community {[}3,6{]} and species level {[}4{]}, for tumorgenesis.
Collectively, these studies provide a tantalizing link between our gut
bacteria and CRC and suggest that biomarkers using our microbes could be
developed. Indeed, builidng models using 16S rRNA gene sequencing along
with clinical tests such as Fecal Immunoglobulin Test (FIT) result in
good predictions of CRC {[}7,8{]}. Although these studies show how our
gut bacteria can impact CRC progression via a changed community or
invasion by more inflammatory bacteria {[}9{]}. They provide very little
information as to whether these changed communities rebound after
successful treatment of lesion (adenoma or carcinoma).

In this study, we tested the hypothesis that there are detectable
changes to the bacterial community between pre- (initial) and post-
(follow up) treatment of lesion. We analyzed changes in alpha and beta
diversity as well as the relative abundance of specific Operational
Taxonomic Units (OTUs). We then investigated several important genera
from oral microbes that have been suggested to be important in CRC
pathogenesis {[}9{]}. We next utilized Random Forest to build two
models: The first was built to classify lesion versus non-lesion
(normal) while the second was built to classify initial versus follow up
treatment samples. Subsequent observations on how these models and OTUs
within them performed before and after treatment helped inform us as to
whether they were changing after treatment and whether it was towards a
normal community. We also investigated the two models for similar
important OTUs to identify which were important for classifying both
lesion versus normal and initial versus follow up treatment samples.
Finally, since treatment also included chemotherapy and radiation
therapy along with lesion removal, we tested if these additionally
treatments significantly lowered any of the metrics examined versus
those who received only lesion removal. This study helps to provide
evidence as to whether treatment can influence the community and if the
CRC microbiome, identified in previous studies, persists.

\newpage

\subsection{Results}\label{results}

\textbf{\emph{The Bacterial Community:}} Within our 67-person cohort we
tested whether those with adenoma (n = 41) or carcinoma (n = 26) had any
broad differences between their initial and follow up samples. We found
that those with carcinoma had a more dissimilar bacterial community
between their initial and follow up sample than those with adenoma
(P-value \textless{} 0.001) {[}Figure 1A{]}. The bacterial community
structure before and after surgery was visualized using NMDS for both
adenoma {[}Figure 1B{]} (PERMANOVA \textgreater{} 0.05) and carcinoma
{[}Figure 1C{]} (PERMANOVA \textless{} 0.05). Interestingly, when
initial and follow up samples were compared, regardless of whether the
lesions were adenoma or carcinoma, there was no significant overall
difference in beta diversity (PERMANOVA \textgreater{} 0.05). There was
no difference between initial and follow up samples when investigating
alpha diversity metrics for lesion, adenoma only, or carcinoma only for
any metric tested {[}Table S1{]}. Additionally, there was also no
difference in the relative abundance of any OTU between initial and
follow up samples for lesion, adenoma only, or carcinoma only {[}Figure
S1{]}.

\textbf{\emph{Carcinoma Associated Genera:}} For carcinoma {[}Figure
S2A{]} (P-value \textgreater{} 0.05) but not adenoma {[}Figure S2B{]}
(P-value \textgreater{} 0.05) there was a significant decrease in
\emph{Porphyromonas}, \emph{Parvimonas}, and \emph{Peptostreptococcus}
between initial and follow up samples {[}Figure S2{]}.
\emph{Fusobacterium} for both adenoma and carcinoma had no significant
differences between initial and follow up samples {[}FIgure S2{]}
(P-value \textgreater{} 0.05). Although there were significant
differences, only a small percentage of those with adenoma or carcinoma
were positive for any of these specific genera {[}Figure S2{]}.

\textbf{\emph{The Lesion Model:}} We tested whether follow up samples
were more normal than initial samples by building a model based solely
on OTUs alone that classified lesion versus normal. This model had an
AUC range of 0.692 - 0.761 after 100 iterations of 20 repeated 10-fold
cross validations. The ROC curve for the final lesion model used was
within the observed range of the 100-different test set AUC iterations
{[}Figure 2A{]}. There was a total of 54 OTUs that were used in this
model {[}Figure 2B{]}. \emph{Lachnospiraceae} (Otu000015) resulted in
the largest decrease in MDA {[}Figure 2B{]}. Heavily represented genera
within the model included OTUs from \emph{Lachnospiraceae},
\emph{Bacteroides}, \emph{Roseburia}, \emph{Blautia}, and
\emph{Ruminococcus}.

If there was movement towards a more normal community after treatment of
lesion, we would expect to find a decrease in the positive probability
of the follow up sample to be a lesion. We observed no such decrease in
positive probability for either those with adenoma or carcinoma
{[}Figure 2C \& 2D{]} (P-value \textgreater{} 0.05). We also observed a
significant difference between the predicted and actual calls (P-value
\textless{} 0.05). However, this model correctly classified the one
individual who still had carcinoma on follow up and this individual's
positive probability of lesion increased between their initial and
follow up sample {[}Figure 2C{]}.

\textbf{\emph{The Treatment Model:}} To test if there were differences
that could be used to classify initial and follow up treatment samples
we built a model to classify initial samples. This treatment model had
an AUC range of 0.657 to 0.796. after 100 iterations of 20 repeated
10-fold cross validations. The test set AUC range for this model
performed better than the training set AUCs. There was a marked decrease
in the ROC curve for the final model used when compared to the 100 test
set AUC iterations {[}Figure 3A{]}. There was a total of 70 OTUs that
were used for this model {[}Figure 3B{]}. The variable that resulted in
the largest MDA was \emph{Ruminococcaceae} (Otu000278) {[}Figure 3B{]}.
Similar to the lesion model, heavily represented genera included OTUs
from \emph{Lachnospiraceae}, \emph{Bacteroides}, \emph{Roseburia},
\emph{Blautia}, and \emph{Ruminococcus}.

If there were changes between initial and follow up samples we would
expect the positive probability of being an initial sample to be
decreased in the follow up. The is what we observed for the treatment
model (P-value \textless{} 0.001). When we separated lesion into adenoma
and carcinoma there was a decrease in positive probability for both the
carcinoma {[}Figure 3C{]} (P-value \textless{} 0.001) and adenoma group
{[}Figure 3D{]} \textless{} 0.001). For this model, there was no
difference between the predicted and actual classifications (P-value
\textgreater{} 0.05).

\textbf{\emph{Common OTUs to both Models:}} We next wanted to know what
predictors within the lesion model were also in the initial sample
model. The main purpose was to identify which OTUs could be important in
lesion formation and were impacted by treatment. When we compared the
two different models with each other there were a total of 32 common
OTUs. Some of the most common taxonomic identifications belonged to
\emph{Lachnospiraceae}, \emph{Bacteroides}, \emph{Roseburia},
\emph{Blautia}, \emph{Anaerostipes}, \emph{Dorea}, and
\emph{Ruminococcus}. These along withthe vast majority of the OTUs that
were common between models had classifications to bacteria typically
thought of as commensal {[}Table S2{]}.

\textbf{\emph{Chemotherapy and Radiation Differences:}} After observing
these changes due to treatment we assessed whether chemotherapy or
radiation impacted the observed results. Only the treatment model had a
significant change in positive probability for those treated with
chemotherapy or radiation versus those who received neither {[}Figure 4A
\& Table S3{]} (P-value \textless{} 0.05). Additionally, there was no
difference between the change in positive probability for chemotherapy
and radiation therapy (Uncorrected P-value \textgreater{} 0.05). Using
the 32 common OTUs we tested how chemotherapy or radiation affects
important members of the community. For both chemotherapy and radiation,
the directionality of change of these OTUs were similar {[}Figure 4B \&
4C{]}. After multiple comparison correction, only \emph{Blautia}
(Otu000006) remained significantly different between those who did and
did not receive chemotherapy. This data suggests that follow up samples
from those treated with either chemotherapy or radiation may have had a
larger change from initial sample due to effects on the bacterial
community.

\newpage

\subsection{Discussion}\label{discussion}

This study builds upon previous work from numerous labs that have
considered both how the bacterial community between those with and
without CRC differ and how it might be used as an early screening tool
{[}7,8,10--12{]}. Here we describe how the bacterial community changes
after treatment and which OTUs may be most important to lesion
classification. Interestingly, many of the most important OTUs for both
treatment and lesion classification had taxonomic identification for
resident gut microbes. This suggest that members within the commensal
community may be the first that change during CRC pathogenesis. These
changes, in turn, could be the first step in allowing more inflammatory
bacteria to gain a foothold within the colon {[}9{]}.

Unlike previous studies on the microbiome and CRC, ours focuses on
identifying commonalities within both adenoma and carcinoma groups
before and after treatment. Although there were differences for genera
associated with specific bacterium linked with CRC {[}Figure S2{]}.
These changes were not consistent across lesion. Instead, most of our
results provide support that the first members of the community to
change and potentially stay changed even after treatment are those that
are commensal bacteria {[}Figure 2-3 \& Table S3{]}. Additionally,
treatment with chemotherapy or radiation changed these bacteria and
provided a larger decrease in positive probability then removal of
lesion.

Curiously, we observed that the typical CRC associated bacteria were not
predictive within our models. There are several reasons why this may
have occurred. First, is that they were not present in enough
individuals to be able to classify those with and without disease with a
high degree of accuracy. Second, is that our Random Forest models were
able to gather the same information from other OTUs. Third, is that
changes in commensal bacteria are the common differences across adenoma
and carcinoma. Finally, it is also possible that all these explanations
could have played a role. Our observations though, would suggest that an
individual's resident bacteria have a large role to play in polyp
formation and could change in a way that allows predictive models to
lower the positive probability of lesion or initial sample before
treatment {[}Figure 2 \& 3{]}. It should be noted that our study does
not argue against the importance of these CRC associated bacteria in the
pathogenesis of disease but rather that they are not the main bacteria
changing after treatment. It is possible that these CRC associated
bacteria are important in the transition from adenoma to carcinoma and
would be one explanation as to why in our data we not only see high
initial relative abundances in carcinoma and not adenoma individuals but
also large decreases in relative abundance in some of those with
carcinoma but not in those with adenoma after treatment {[}Figure S2{]}.

Many of the common OTUs that we identified taxnomically classified to
potential butyrate producers {[}Table S2{]}. Another batch of OTUs
classified to bacteria that can either degrade polyphenols or are
inhibited by them. Both butyrate and polyphenols are thought to be
protective against cancer in part by reducing inflammation {[}13{]}.
These protective compounds are derived from the breakdown of fiber,
fruits, and vegetables by resident gut microbes. One example of this
potential diet-microbiome-inflammation-polyp axis is that
\emph{Bacteroides}, which was highly prevalent in our models, are known
to be increased in those with high non-meat based protein consumption
{[}14{]}. High protein consumption in general has been linked with an
increased CRC risk {[}15{]}. Conversely, \emph{Bacteroides} are
inhibited by polyphenols which are derived from fruits and vegetables
{[}16{]}. Our data fits with the hypothesis that the microbial
metabolites from breakdown products within our own diet could not only
help to shape the existing community but also have an effect on CRC risk
and disease progression.

A limitation, in our study, was that there was a significant difference
in the time elapsed in the collection of the follow up sample between
adenoma and carcinoma (uncorrected P-value \textless{} 0.05), with time
passed being less for adenoma (253 +/- 41.3 days) than carcinoma (351
+/- 102 days). These results would indicate that some of the differences
observed between the carcinoma and adenoma groups could be due to
differences in collection time. Specifically, it could confound the
observation that carcinomas changed more than adenomas {[}Figure 1A \&
1C{]}. This confounding though would not affect the observations where
these individuals were grouped together {[}Figures 2-4{]}.

Another limitation was that we do not know whether individuals who were
still classified as positive by the lesion model eventually had a
subsequent CRC diagnosis. This information would help to strengthen the
case for this model keeping several individuals above the cutoff
threshold even though at follow up they were diagnosed as no longer
having a lesion. This study also drew heavily from those with Caucasian
ancestry making it possible that the observations may not be
representative of those with either Asian or African ancestry. Although
our training and test set are relatively large we still run the risk of
over-fitting or having a model that may not be representative of other
populations. We've done our best to safeguard against this by not only
running 10-fold cross validation but also having over 100 different
80/20 splits to try and mimic the type of variation that might be
expected to occur.

Interestingly, within the treatment model the test data performed better
than the training data. This may have occurred because the training AUC
determined from 20 repeated 10-fold cross validation removed samples at
random and did not consider that they were matched samples. Another
potential reason is that the model itself may have been over-fit since
the total number of samples was not that large. However, the lesion
model did not suffer from these discrepancies. Further independent
studies need to be carried out to verify our findings on lesion and
treatment changes due to these limitations.

Despite these shortcomings our findings add to the existing scientific
knowledge on CRC and the microbiome: That there is a measurable
difference in the bacterial community after adenoma and carcinoma
treatment. Further, the ability for machine learning algorithms to take
OTU data and successfully lower positive probability of lesion or
initial sample after treatment provides evidence that there are specific
signatures, mostly attributable to commensal organisms, associated with
both treatment and lesion. Our data provides evidence that commensal
bacteria may be important in the development of polyps and potentially
the transition from adenoma to carcinoma.

\newpage

\subsection{Methods}\label{methods}

\textbf{\emph{Study Design and Patient Sampling:}} Sampling and design
have been previously reported in Baxter, et al {[}7{]}. Briefly, study
exclusion involved those who had already undergone surgery, radiation,
or chemotherapy, had colorectal cancer before a baseline fecal sample
could be obtained, had IBD, a known hereditary non-polyposis colorectal
cancer, or familial adenomatous polyposis. Samples used to build the
models for prediction were collected either prior to a colonoscopy or
between 1 - 2 weeks after. The bacterial community has been shown to
normalize back to a pre-colonscopy community within this time period
{[}17{]}. Our study cohort consisted of 67 individuals with an initial
sample as described and a follow up sample obtained between 188 - 546
days after treatment of lesion. This study was approved by the
University of Michigan Institutional Review Board. All study
participants provided informed consent and the study itself conformed to
the guidelines set out by the Helsinki Declaration.

\textbf{\emph{16S rRNA Gene Sequencing:}} Sequencing was completed as
described by Kozich, et al. {[}18{]}. DNA extraction used the 96-well
Soil DNA isolation kit (MO BIO Laboratories) and an epMotion 5075
automated pipetting system (Eppendorf). The V4 variable region was
amplified and the resulting product was split between three sequencing
runs with normal, adenoma, and carcinoma evenly represented on each run.
Each group was randomly assigned to avoid biases based on sample
collection location. The initial and follow up samples were sequenced on
the same run.

\textbf{\emph{Sequence Processing:}} The mothur software package
(v1.37.5) was used to process the 16S rRNA gene sequences and has been
previously described {[}18{]}. The general workflow using mothur was:
Paired-end reads were first merged into contigs, quality filtered,
aligned to the SILVA database, screened for chimeras, classified with a
naive Bayesian classifier using the Ribosomal Database Project (RDP),
and clustered into Operational Taxonomic Units (OTUs) using a 97\%
similarity cutoff with an average neighbor clustering algorithm. The
number of sequences for each sample was rarefied to 10523 to minimize
uneven sampling.

\textbf{\emph{Lesion Model Creation:}} The Random Forest {[}19{]}
algorithm was used to create the model used for prediction of lesion
(adenoma or carcinoma) with the main training and testing of the model
completed on an independent data set of 423 individuals. This model was
then applied to our 67-person cohort. It should be noted that all
individuals with an adenoma or carcinoma were grouped together to form
the lesion group and the model was not created to find differences
between normal, adenoma, and carcinoma but rather differences between
both adenoma and carcinoma versus normal.

The model included only OTU data obtained from 16S rRNA sequencing.
Non-binary data was checked for near zero variance and OTUs that had
near zero variance were removed. This pre-processing was performed with
the R package caret (v6.0.73). Optimization of the mtry hyper-parameter
involved making 100 different 80/20 (train/test) splits of the data
where normal and lesion were represented in the same proportion within
both the whole data set and the 80/20 split. For each a 20 repeated
10-fold cross validation was performed on 80\% component to optimize the
mtry hyper-parameter by maximizing the AUC (Area Under the Curve of the
Receiver Operator Characteristic). The resulting model was then tested
on the hold out data obtained from the 20\% component. Assessment of the
most important OTUs to the model involved counting the number of times
an OTU was present in the top 10\% of mean decrease in accuracy (MDA)
for each of the 100 different splits run. This was then followed with
filtering of this list to variables that were only present in more than
50\% of these 100 runs. The final collated list of variables was then
run through the mtry optimization again. Once the ideal mtry was found
the entire 423 sample set was used to create the final Random Forest
model on which classifications on the 67-person cohort was completed.

The default cutoff of 0.5 was used as the threshold to classify
individuals as positive or negative for lesion. The hyper-parameter,
mtry, defines the number of variables to investigate at each split
before a new division of the data was created with the Random Forest
model.

\textbf{\emph{Treatment Model Creation:}} We also investigated whether a
model could be created that could identify pre- (initial) and post-
(follow up) treatment samples. The main difference was that only the
67-person cohort was used at all stages of model building and
classification. Other than this difference the creation of this model
and optimization of the mtry hyper-parameter was completed using the
same procedure as was used for the lesion model. Instead of classifying
samples as positive or negative of lesion this model classified samples
as positive or negative for being an initial sample prior to treatment.

\textbf{\emph{Statistical Analysis:}} The R software package (v3.3.2)
was used for all statistical analysis. Comparisons between bacterial
community structure utilized PERMANOVA {[}20{]} in the vegan package
(v2.4.1). Comparisons between probabilities as well as overall OTU
differences between initial and follow up samples utilized a paired
Wilcoxson ranked sum test. Where multiple comparison testing was
appropriate, a Benjamini-Hochberg (BH) correction was applied {[}21{]}
and a corrected P-value of less than 0.05 was considered significant.
Unless otherwise stated the P-values reported are those that were BH
corrected.

\textbf{\emph{Analysis Overview:}} We first tested for any differences
based on whether the individual had an adenoma or carcinoma. This was
done by testing initial and follow up samples for differences in alpha
and beta diversity, testing all OTUs, and investigating the relative
abundance of genera from previously associated CRC bacteria
(\emph{Fusobacterium}, \emph{Parvimonas}, \emph{Peptostreptococcus}, and
\emph{Porphyromonas}). Next, the lesion model was tested for accuracy in
prediction and whether it reduced the positive probability of lesion in
follow up samples. We then used the treatment model to assess whether it
could classify samples better than the lesion model and whether it could
reduce the positive probability of an initial sample in the follow up
samples. Common OTUs were found for the two different models used to
assess which were important for both models. Finally, differences
between those receiving chemotherapy and radiation versus those who
received neither were tested.

\textbf{\emph{Reproducible Methods:}} A detailed and reproducible
description of how the data were processed and analyzed can be found at
\url{https://github.com/SchlossLab/Sze_followUps_2017}. Raw sequences
have been deposited into the NCBI Sequence Read Archive (SRP062005 and
SRP096978) and the necessary metadata can be found at
\url{https://www.ncbi.nlm.nih.gov/Traces/study/} and searching the
respective SRA study accession.

\newpage

\textbf{Figure 1: General Differences between the Adenoma and Carcinoma
Group.} A) A significant difference was found between the adenoma and
carcinoma group for thetayc (P-value = 0.000472). Advanced adenomas are
denoted as Screen Relevant Neoplasia (SRN). B) NMDS of the initial and
follow up samples for the adenoma group. C) NMDS of the initial and
follow up samples for the carcinoma group.

\textbf{Figure 2: The Lesion Model.} A) ROC curve: The shaded areas
represent the range of values of a 100 different 80/20 splits of the
test set data and the blue line represents the model using 100\% of the
data set and what was used for subsequent classification. B) Summary of
Important Variables. MDA of the most important variables in the lesion
model. The black point represents the mean and the different colors are
the values of each different run up to 100. C) Positive probability
change from initial to follow up sample in those with carcinoma. D)
Positive probability change from initial to follow up sample if those
with adenoma or advanced adenoma (Screen Relevant Neoplasia (SRN)).

\textbf{Figure 3: The Treatment Model.} A) ROC curve: The shaded areas
represent the range of values of a 100 different 80/20 splits of the
test set data and the blue line represents the model using 100\% of the
data set and what was used for subsequent classification. B) Summary of
Important Variables. MDA of the most important variables in the initial
sample model. The black point represents the mean and the different
colors are the values of each different run up to 100. C) Positive
probability change from initial to follow up sample in those with
carcinoma. D) Positive probability change from initial to follow up
sample of those with adenoma or advanced adenoma (Screen Relevant
Neoplasia (SRN)).

\textbf{Figure 4: Chemotherapy and Radiation Therapy Effects on
Community.} A) Treatment model initial sample positive probability
reduction for chemotherapy. B) Treatment model initial sample positive
probability reduction for radiation therapy. C) Common OTUs based on
whether chemotherapy was received or not. The * denotes a P-value
\textless{} 0.05 after multiple comparison correction. D) Common OTUs
based on whether radiation therapy was received or not.

\newpage

\textbf{Figure S1: Distribution of P-values from Paired Wilcoxson
Analysis of All OTUs for Initial versus Follow Up}

\textbf{Figure S2: Previously Associated CRC Bacteria in Initial and
Follow Up Samples.} A) Carcinoma initial and follow up samples had an
observed significant difference in initial and follow up sample for the
OTUs classified as \emph{Parvimonas} (P-value = 0.0059),
\emph{Porphyromonas} (P-value = 0.225), \emph{Peptostreptococcus}
(P-value = 0.00424). B) Adenoma initial and follow up samples. There
were no significant differences between initial and follow up (P-value =
0.881.

\newpage

\subsection{Declarations}\label{declarations}

\subsubsection{Ethics approval and consent to
participate}\label{ethics-approval-and-consent-to-participate}

The University of Michigan Institutional Review Board approved this
study, and all subjects provided informed consent. This study conformed
to the guidelines of the Helsinki Declaration.

\subsubsection{Consent for publication}\label{consent-for-publication}

Not applicable.

\subsubsection{Availability of data and
material}\label{availability-of-data-and-material}

A detailed and reproducible description of how the data were processed
and analyzed can be found at
\url{https://github.com/SchlossLab/Sze_followUps_2017}. Raw sequences
have been deposited into the NCBI Sequence Read Archive (SRP062005 and
SRP096978) and the necessary metadata can be found at
\url{https://www.ncbi.nlm.nih.gov/Traces/study/} and searching the
respective SRA study accession.

\subsubsection{Competing Interests}\label{competing-interests}

All authors declare that they do not have any relevant competing
interests to report.

\subsubsection{Funding}\label{funding}

This study was supported by funding from the National Institutes of
Health to P. Schloss (R01GM099514, P30DK034933) and to the Early
Detection Research Network (U01CA86400).

\subsubsection{Authors' contributions}\label{authors-contributions}

All authors were involved in the conception and design of the study. MAS
analyzed the data. NTB processed samples and analyzed the data. All
authors interpreted the data. MAS and PDS wrote the manuscript. All
authors reviewed and revised the manuscript. All authors read and
approved the final manuscript.

\subsubsection{Acknowledgements}\label{acknowledgements}

The authors thank the Great Lakes-New England Early Detection Research
Network for providing the fecal samples that were used in this study. We
would also like to thank Amanda Elmore for reviewing and correcting code
error and providing feedback on manuscript drafts. We would also like to
thank Nicholas Lesniak for providing feedback on manuscript drafts.

\newpage

\subsection*{References}\label{references}
\addcontentsline{toc}{subsection}{References}

\hypertarget{refs}{}
\hypertarget{ref-jemal_cancer_2010}{}
1. Jemal A, Siegel R, Xu J, Ward E. Cancer statistics, 2010. CA: a
cancer journal for clinicians. 2010;60:277--300.

\hypertarget{ref-haggar_colorectal_2009}{}
2. Haggar FA, Boushey RP. Colorectal cancer epidemiology: Incidence,
mortality, survival, and risk factors. Clinics in Colon and Rectal
Surgery. 2009;22:191--7.

\hypertarget{ref-zackular_manipulation_2016}{}
3. Zackular JP, Baxter NT, Chen GY, Schloss PD. Manipulation of the Gut
Microbiota Reveals Role in Colon Tumorigenesis. mSphere. 2016;1.

\hypertarget{ref-arthur_microbial_2014}{}
4. Arthur JC, Gharaibeh RZ, Mühlbauer M, Perez-Chanona E, Uronis JM,
McCafferty J, et al. Microbial genomic analysis reveals the essential
role of inflammation in bacteria-induced colorectal cancer. Nature
Communications. 2014;5:4724.

\hypertarget{ref-dejea_microbiota_2014}{}
5. Dejea CM, Wick EC, Hechenbleikner EM, White JR, Mark Welch JL,
Rossetti BJ, et al. Microbiota organization is a distinct feature of
proximal colorectal cancers. Proceedings of the National Academy of
Sciences of the United States of America. 2014;111:18321--6.

\hypertarget{ref-zackular_gut_2013}{}
6. Zackular JP, Baxter NT, Iverson KD, Sadler WD, Petrosino JF, Chen GY,
et al. The gut microbiome modulates colon tumorigenesis. mBio.
2013;4:e00692--00613.

\hypertarget{ref-baxter_microbiota-based_2016}{}
7. Baxter NT, Ruffin MT, Rogers MAM, Schloss PD. Microbiota-based model
improves the sensitivity of fecal immunochemical test for detecting
colonic lesions. Genome Medicine. 2016;8:37.

\hypertarget{ref-zeller_potential_2014}{}
8. Zeller G, Tap J, Voigt AY, Sunagawa S, Kultima JR, Costea PI, et al.
Potential of fecal microbiota for early-stage detection of colorectal
cancer. Molecular Systems Biology. 2014;10:766.

\hypertarget{ref-flynn_metabolic_2016}{}
9. Flynn KJ, Baxter NT, Schloss PD. Metabolic and Community Synergy of
Oral Bacteria in Colorectal Cancer. mSphere. 2016;1.

\hypertarget{ref-yu_metagenomic_2017}{}
10. Yu J, Feng Q, Wong SH, Zhang D, Liang QY, Qin Y, et al. Metagenomic
analysis of faecal microbiome as a tool towards targeted non-invasive
biomarkers for colorectal cancer. Gut. 2017;66:70--8.

\hypertarget{ref-zackular_human_2014}{}
11. Zackular JP, Rogers MAM, Ruffin MT, Schloss PD. The human gut
microbiome as a screening tool for colorectal cancer. Cancer Prevention
Research (Philadelphia, Pa.). 2014;7:1112--21.

\hypertarget{ref-warren_co-occurrence_2013}{}
12. Warren RL, Freeman DJ, Pleasance S, Watson P, Moore RA, Cochrane K,
et al. Co-occurrence of anaerobic bacteria in colorectal carcinomas.
Microbiome. 2013;1:16.

\hypertarget{ref-louis_gut_2014}{}
13. Louis P, Hold GL, Flint HJ. The gut microbiota, bacterial
metabolites and colorectal cancer. Nature Reviews Microbiology
{[}Internet{]}. 2014 {[}cited 2017 Feb 14{]};12:661--72. Available from:
\url{http://www.nature.com/doifinder/10.1038/nrmicro3344}

\hypertarget{ref-zhu_intake_2016}{}
14. Zhu Y, Lin X, Li H, Li Y, Shi X, Zhao F, et al. Intake of Meat
Proteins Substantially Increased the Relative Abundance of Genus
Lactobacillus in Rat Feces. PloS One. 2016;11:e0152678.

\hypertarget{ref-mu_colonic_2016}{}
15. Mu C, Yang Y, Luo Z, Guan L, Zhu W. The Colonic Microbiome and
Epithelial Transcriptome Are Altered in Rats Fed a High-Protein Diet
Compared with a Normal-Protein Diet. The Journal of Nutrition.
2016;146:474--83.

\hypertarget{ref-ozdal_reciprocal_2016}{}
16. Ozdal T, Sela DA, Xiao J, Boyacioglu D, Chen F, Capanoglu E. The
Reciprocal Interactions between Polyphenols and Gut Microbiota and
Effects on Bioaccessibility. Nutrients {[}Internet{]}. 2016 {[}cited
2017 Feb 14{]};8:78. Available from:
\url{http://www.mdpi.com/2072-6643/8/2/78}

\hypertarget{ref-obrien_impact_2013}{}
17. O'Brien CL, Allison GE, Grimpen F, Pavli P. Impact of colonoscopy
bowel preparation on intestinal microbiota. PloS One. 2013;8:e62815.

\hypertarget{ref-kozich_development_2013}{}
18. Kozich JJ, Westcott SL, Baxter NT, Highlander SK, Schloss PD.
Development of a dual-index sequencing strategy and curation pipeline
for analyzing amplicon sequence data on the MiSeq Illumina sequencing
platform. Applied and Environmental Microbiology. 2013;79:5112--20.

\hypertarget{ref-breiman_random_2001}{}
19. Breiman L. Random Forests. Machine Learning {[}Internet{]}. 2001
{[}cited 2013 Feb 7{]};45:5--32. Available from:
\href{http://link.springer.com/article/10.1023/A\%3A1010933404324\%20http://link.springer.com/article/10.1023\%2FA\%3A1010933404324?LI=true}{http://link.springer.com/article/10.1023/A\%3A1010933404324 http://link.springer.com/article/10.1023\%2FA\%3A1010933404324?LI=true}

\hypertarget{ref-anderson_permanova_2013}{}
20. Anderson MJ, Walsh DCI. PERMANOVA, ANOSIM, and the Mantel test in
the face of heterogeneous dispersions: What null hypothesis are you
testing? Ecological Monographs {[}Internet{]}. 2013 {[}cited 2017 Jan
5{]};83:557--74. Available from:
\url{http://doi.wiley.com/10.1890/12-2010.1}

\hypertarget{ref-benjamini_controlling_1995}{}
21. Benjamini Y, Hochberg Y. Controlling the false discovery rate: A
practical and powerful approach to multiple testing. Journal of the
Royal Statistical Society. Series B (Methodological). 1995;57:289--300.


\end{document}
