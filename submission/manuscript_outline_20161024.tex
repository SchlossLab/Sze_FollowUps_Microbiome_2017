\documentclass[12pt,]{article}
\usepackage{lmodern}
\usepackage{amssymb,amsmath}
\usepackage{ifxetex,ifluatex}
\usepackage{fixltx2e} % provides \textsubscript
\ifnum 0\ifxetex 1\fi\ifluatex 1\fi=0 % if pdftex
  \usepackage[T1]{fontenc}
  \usepackage[utf8]{inputenc}
\else % if luatex or xelatex
  \ifxetex
    \usepackage{mathspec}
  \else
    \usepackage{fontspec}
  \fi
  \defaultfontfeatures{Ligatures=TeX,Scale=MatchLowercase}
\fi
% use upquote if available, for straight quotes in verbatim environments
\IfFileExists{upquote.sty}{\usepackage{upquote}}{}
% use microtype if available
\IfFileExists{microtype.sty}{%
\usepackage{microtype}
\UseMicrotypeSet[protrusion]{basicmath} % disable protrusion for tt fonts
}{}
\usepackage[margin=1.0in]{geometry}
\usepackage{hyperref}
\hypersetup{unicode=true,
            pdfborder={0 0 0},
            breaklinks=true}
\urlstyle{same}  % don't use monospace font for urls
\usepackage{graphicx,grffile}
\makeatletter
\def\maxwidth{\ifdim\Gin@nat@width>\linewidth\linewidth\else\Gin@nat@width\fi}
\def\maxheight{\ifdim\Gin@nat@height>\textheight\textheight\else\Gin@nat@height\fi}
\makeatother
% Scale images if necessary, so that they will not overflow the page
% margins by default, and it is still possible to overwrite the defaults
% using explicit options in \includegraphics[width, height, ...]{}
\setkeys{Gin}{width=\maxwidth,height=\maxheight,keepaspectratio}
\IfFileExists{parskip.sty}{%
\usepackage{parskip}
}{% else
\setlength{\parindent}{0pt}
\setlength{\parskip}{6pt plus 2pt minus 1pt}
}
\setlength{\emergencystretch}{3em}  % prevent overfull lines
\providecommand{\tightlist}{%
  \setlength{\itemsep}{0pt}\setlength{\parskip}{0pt}}
\setcounter{secnumdepth}{0}
% Redefines (sub)paragraphs to behave more like sections
\ifx\paragraph\undefined\else
\let\oldparagraph\paragraph
\renewcommand{\paragraph}[1]{\oldparagraph{#1}\mbox{}}
\fi
\ifx\subparagraph\undefined\else
\let\oldsubparagraph\subparagraph
\renewcommand{\subparagraph}[1]{\oldsubparagraph{#1}\mbox{}}
\fi

%%% Use protect on footnotes to avoid problems with footnotes in titles
\let\rmarkdownfootnote\footnote%
\def\footnote{\protect\rmarkdownfootnote}

%%% Change title format to be more compact
\usepackage{titling}

% Create subtitle command for use in maketitle
\newcommand{\subtitle}[1]{
  \posttitle{
    \begin{center}\large#1\end{center}
    }
}

\setlength{\droptitle}{-2em}
  \title{}
  \pretitle{\vspace{\droptitle}}
  \posttitle{}
  \author{}
  \preauthor{}\postauthor{}
  \date{}
  \predate{}\postdate{}

\usepackage{helvet} % Helvetica font
\renewcommand*\familydefault{\sfdefault} % Use the sans serif version of the font
\usepackage[T1]{fontenc}

\usepackage[none]{hyphenat}

\usepackage{setspace}
\doublespacing
\setlength{\parskip}{1em}

\usepackage{lineno}

\usepackage{pdfpages}
\usepackage{comment}

\begin{document}

\section{The Fecal Microbiome Before and After Treatment for Colorectal
Adenoma or
Carcinoma}\label{the-fecal-microbiome-before-and-after-treatment-for-colorectal-adenoma-or-carcinoma}

\vspace{25mm}

\begin{center}
Running Title: Human Microbiome before and after Colorectal Cancer

\vspace{10mm}

Marc A Sze${^1}$, Nielson T Baxter${^2}$, Mack T Ruffin IV${^3}$, Mary AM Rogers${^2}$, and Patrick D Schloss${^1}$${^\dagger}$

\vspace{20mm}

$\dagger$ To whom correspondence should be addressed: pschloss@umich.edu

$1$ Department of Microbiology and Immunology, University of Michigan, Ann Arbor, MI

$2$ Department of Internal Medicine, University of Michigan, Ann Arbor, MI   

$3$ Department of Family Medicine and Community Medicine, Penn State Hershey Medical Center, Hershey, PA    


\end{center}

\newpage

\linenumbers

\subsection{Abstract}\label{abstract}

\textbf{Background:} Colorectal cancer (CRC) continues to be a worldwide
health problem with previous research suggesting that a link may exist
between the fecal bacterial microbiome and CRC. The overall objective of
our study was to test the hypothesis that changes in the bacterial
microbiome occur after lesion (i.e.~adenoma and carcinoma) removal.
Specifically, we wanted to identify components within the community that
were different before and after removal of said lesion.

\textbf{Results:} The bacterial microbiome changed more in response to
lesion removal in carcinoma cases compared to adenoma cases (P-value
\textless{} 0.05). There was no difference for either the adenoma or
carcinoma group in the relative abundance of any OTU between pre and
post lesion removal (P-value \textgreater{} 0.05). A model built to
classify lesion had an AUC range of 0.811 - 0.866 and follow up samples
had a decrease in the positive probability of lesion (P-value
\textless{} 0.05) suggesting a movement towards a more normal bacterial
community. A second model built to classify initial samples had an AUC
range of 0.641 - 0.805 and had a decrease in positive probability for
the follow up samples to be an initial sample (P-value \textless{}
0.05). The lesion model used a total of 53 variables while the initial
sample model used a total of 70 variables. A total of 23 OTUs were
common to both models with the majority of these classifying to
commensal bacteria (e.g. \emph{Bacteroides}, \emph{Clostridiales},
\emph{Blautia}, and \emph{Ruminococcaceae}).

\textbf{Conclusions:} Our data supports the hypothesis that there are
changes in the bacterial microbiome following colorectal lesion removal.
Individuals with carcinoma have more drastic differences to the overall
community then those with adenoma. Changes to commensal bacteria were
some of the most important variables for model classification,
suggesting that these bacteria may be central to initial polyp formation
and transition to carcinoma.

\newpage

\subsubsection{Keywords}\label{keywords}

bacterial microbiome; colorectal cancer; polyps; FIT; post surgery; risk
factors

\newpage

\subsection{Background}\label{background}

Colorectal cancer (CRC) continues to be a leading cause of cancer
related deaths and is currently the third most common cause of cancer
deaths {[}1,2{]}. Over the last few years the rate of disease mortality
has seen a significant decrease, thanks mainly to improvements in
screening {[}1{]}. However, despite this improvement there are still
approximately 50,000 deaths from the disease per year {[}2{]}.

Over the last few years studies have shown how the bacterial microbiome
{[}3{]} or specific members within it {[}4{]} could be directly involved
with the pathogenesis of CRC. There has also been research into how the
bacterial microbiome could be altered directly on tumor tissue itself
{[}5{]}. These studies have helped to provide a tantalizing link between
the bacterial microbiome and CRC. Building off of these findings, there
has been promising work on the bacterial microbiome and its ability to
complement existing screening methods such as Fecal Immunoglobulin Test
(FIT) or act alone as a screening tool {[}6,7{]}. Although these studies
suggest that the bacterial microbiome might change after treatment there
remains limited information on the bacterial microbiome response to
lesion (adenoma and carcinoma) removal.

In this study we tested the hypothesis that the bacterial microbiome
changes between pre (initial) and post (follow up) removal of lesion. We
analyzed changes in alpha and beta diversity as well as the relative
abundance of specific Operational Taxonomic Units (OTUs). We utilized
Random Forest to build two models. The first model was built to classify
lesion versus non-lesion (normal) samples while the second was built to
classify initial versus follow up samples. Subsequent observations on
how these models, as well as specific OTUs within them, performed pre
and post lesion removal helped inform us as to whether initial and
follow up samples were changing and whether it was towards a more normal
bacterial microbiome or not. We also investigated the two models for
similar important OTUs to identify the crucial OTUs for not only
classifying lesion or normal samples but also initial versus follow up.

\newpage

\subsection{Results}\label{results}

\textbf{\emph{Bacterial Community and FIT:}} Within our 67 person cohort
we first wanted to test whether there were any broad differences between
initial and follow up samples based on lesion being either adenoma (n =
41) or carcinoma (n = 26). We found that the bacterial community in
those with carcinoma were more dissimilar to their initial sample than
those with adenoma (P-value \textless{} 0.001) {[}Figure 1A{]}. We also
found that there were larger changes in fecal blood (measured by FIT)
for those with carcinoma versus adenoma (P-value \textless{} 0.0001)
{[}Figure 1B{]}. The bacterial community structure before and after
surgery was visualized using NMDS for both adenoma {[}Figure 1C{]}
(PERMANOVA \textgreater{} 0.05) and carcinoma {[}Figure 1D{]} (PERMANOVA
\textless{} 0.05). Interestingly, when initial and follow up samples
were compared, regardless of whether the lesions were adenoma or
carcinoma, there was no significant overall difference in beta diversity
(PERMANOVA \textgreater{} 0.05). When investigating alpha diversity
metrics there was no difference found between initial and follow up
samples for lesion, adenoma only, or carcinoma only for any metric
tested {[}Table S1{]}. We also observed that there was no difference in
the relative abundance of any OTU between initial and follow up samples
for lesion, adenoma only, or carcinoma only {[}Figure S1{]}.

\textbf{\emph{Carcinoma Associated Bacteria:}} Previous literature has
suggested that a number of oral microbes may be important in CRC
pathogenesis {[}8{]}. So we next examined whether there were changes in
previously well described carcinoma associated OTUs, such as
\emph{Porphyromonas asaccharolytica} (Otu000202), \emph{Fusobacterium
nucleatum} (Otu000442), \emph{Parvimonas micra} (Otu001273), and
\emph{Peptostreptococcus stomatis} (Otu001682). We first observed that
only a small percentage of those with adenoma or carcinoma were positive
or had a relative abundance above 0.1\% for any of these respective OTUs
{[}Figure 2{]}. Despite this, those with carcinoma had a decrease in
relative abundance from initial to follow up for \emph{Parvimonas micra}
(P-value \textless{} 0.05) and \emph{Porphyromonas asaccharolytica}
(P-value \textless{} 0.05) {[}Figure 2A{]}. In contrast, there was no
difference in relative abundance in any of these OTUs when considering
only those with adenoma {[}Figure 2B{]}.

\textbf{\emph{The Lesion Model:}} We next wanted to identify if there
were any common bacterial microbiome changes in individuals with adenoma
and carcinoma versus normal controls. We investigated this by creating a
model to classify samples as lesion versus normal based on the bacterial
community and FIT measurements. This model had an AUC range of 0.811 -
0.866 after 100 iterations of 20 repeated 10-fold cross validations. The
ROC curve for the final lesion model used was within the observed range
of the 100 different test set AUC iterations {[}Figure 3A{]}. There were
a total of 53 variables that were used in this model {[}Figure 3B{]}.
The FIT measurement for fecal blood resulted in the largest decrease in
MDA while the OTU with the largest MDA was \emph{Lachnospiraceae}
(Otu000015) {[}Figure 3B{]}.

If there were common OTUs that could separate adenoma and carcinoma from
normal controls, we would expect to find a decrease in the positive
probability of the follow up sample to be a lesion. This is what we
observed for the lesion model (P-value \textless{} 0.001). When we
separated individuals based on whether they had an adenoma or carcinoma
there was only a decrease in positive probability for the carcinoma
group {[}Figure 3C{]} (P-value \textless{} 0.001) and not for the
adenoma group {[}Figure 3D{]} (P-value \textgreater{} 0.05). We also
observed that there were no significant differences between the
predicted and actual calls (P-value \textgreater{} 0.05). The lesion
model was also able to correctly classify the one individual who still
had a carcinoma on follow up {[}Figure 3C{]}.

\textbf{\emph{The Initial Sample Model:}} After building a model to
classify based on lesion we built a separate model specifically to be
able to classify whether samples were initial (before lesion was
removed) samples based on the bacterial community and FIT measurements.
The initial sample model had an AUC range of 0.641 to 0.805 after 100
iterations of 20 repeated 10-fold cross validations. The test set AUC
range for this model performed better then the training set AUCs. There
was a marked decrease in the ROC curve for the final model used when
compared to the 100 test set AUC iterations {[}Figure 4A{]}. There were
a total of 70 variables that were used for this model {[}Figure 4B{]}.
The variable that resulted in the largest MDA was \emph{Pseudomonas}
(Otu000438) while FIT measurement for fecal blood resulted in the sixth
largest decrease in MDA {[}Figure 4B{]}.

If there were common OTUs that could separate initial from follow up
sample regardless of whether the lesion was adenoma or carcinoma we
would expect to find a decrease in the positive probability of the
follow up sample to be an initial sample. This is what we observed for
the initial sample model (P-value \textless{} 0.001). When we separated
individuals based on whether they had an adenoma or carcinoma there was
a decrease in positive probability for both the carcinoma group
{[}Figure 4C{]} (P-value \textless{} 0.001) and for the adenoma group
{[}Figure 4D{]} (P-value \textless{} 0.001). For this model there was no
difference between the predicted and actual classifications (P-value
\textgreater{} 0.05).

\textbf{\emph{Common OTUs to both Models:}} We next wanted to compare
the similarity between the OTU variables used in either model. The main
purpose was to identify which OTUs were important not only for the
classification of lesion but also for the classification of initial or
follow up sample. Potentially, these specific OTUs are the most
important with respect to the bacterial microbiome response to removal
of lesion. When we compared the two different models with each other
there were a total of 23 common OTUs. Some of the most common taxonomic
identifications belonged to \emph{Bacteroides}, \emph{Clostridiales},
\emph{Blautia}, and \emph{Ruminococcaceae}. The vast majority of these
OTUs had classifications to bacteria typically thought of as commensal
{[}Table S2{]}.

\textbf{\emph{Treatment Differences:}} After observing these changes in
the bacterial community and positive probability we wanted to assess
whether additional treatments, such as chemotherapy and radiation, could
have an impact on the results that we observed. There was only a
significant difference for change in positive probability for those
treated with chemotherapy for the initial sample model (P-value
\textless{} 0.05). This suggests that follow up samples for those
treated for chemotherapy my have had a larger change from the initial
sample then those without such treatment. All other variables that were
tested showed no difference based on whether chemotherapy or radiation
was used {[}Table S3{]}.

\newpage

\subsection{Discussion}\label{discussion}

This study builds upon previous work from numerous labs that have looked
into the bacterial microbiome as a potential screening tool
{[}6,7,9--11{]} by exploring what happens to the bacterial community
after lesion removal. Based on previous work by Arthur, et al. {[}12{]}
it may not be surprising to have E.coli as one of the most important
OTUs and one that was common to both models. Interestingly, many of the
most important OTUs had taxonomic identification for resident gut
microbes. This could suggest that the bacterial community is one of the
first components that could change during the pathogenesis of disease.
These bacterial microbiome changes could be the first step in allowing
more inflammatory bacterium to gain a foothold within the colon {[}8{]}.

From our results there were large observed differences in the bacterial
microbiome in samples before and after lesion removal based on whether
the individual had an adenoma or carcinoma. In individuals with
carcinoma compared to adenoma there were much larger differences between
initial and follow up samples based on the community beta diversity and
in fecal blood as measured by FIT {[}Figure 1{]}. However, there were no
differences between initial and follow up samples for any alpha
diversity metric measured regardless of whether the individual had an
adenoma or carcinoma {[}Table S1{]}. There was also no differences in
relative abundance of any specific OTU for lesion, adenoma only, or
carcinoma only {[}Figure S1{]}.

Although there were no differences when investigating all OTUs, when
looking specifically at four OTUs that taxonomically classified to
previously suggested carcinoma associated microbes we found that only
2/4 had a decrease in relative abundance between initial and follow up
for those with carcinoma and 0/4 had differences for those with adenoma.
This data would suggest that these specific OTUs may be important in the
transition of an adenoma to a carcinoma but less so in the initiation of
an adenoma from benign tissue.

We next created a model that incorporated FIT and the bacterial
microbiome to be able to classify lesions (adenoma and carcinoma). Based
on this model we found that the follow up samples were closer to normal
then the initial samples due to a decrease in positive probability for
lesion and that the commonly associated CRC bacteria were not highly
represented within this model with the exception of \emph{Porphyromonas
asaccharolytica} {[}Figure 3{]}. Although there was a detectable change
towards what would be expected for normal controls, it should be noted
that the follow up samples may not be a completely normal bacterial
microbiome.

After creating the lesion mode we then created a second model to
classify initial versus follow up samples. We found that this model was
able to accurately classify initial versus follow up samples suggesting
that regardless of adenoma or carcinoma there are distinct common
changes within the bacterial microbiome that occurs after lesion removal
{[}Figure 4{]}. Both models had OTUs that overwhelmingly belonged to
commensal bacteria. Providing additional information on the importance
of commensal bacteria was that there were a total of 23 OTUs in common
to both models and the vast majority belonged to regular residents of
our gut community {[}Table S2{]}.

Within our study there was a significant difference for the time elapsed
in the collection of the follow up sample between adenoma and carcinoma
(uncorrected P-value \textless{} 0.05), with time passed being less for
adenoma (253 +/- 41.3 days) than carcinoma (351 +/- 102 days). These
results would indicate that the findings described were specific to the
surgical intervention and that some of the differences observed between
carcinoma and adenoma samples could be due to differences in collection
time between samples for the two different groups. Specifically, it
could confound the observation that carcinomas changed more than
adenomas {[}Figure 1A \& 1D{]}. This confounding though would not affect
the observations where these individuals were grouped together {[}Figure
3 \& 4{]}.

Curiously, we observed that the typical CRC associated bacteria were not
predictive within our models. There are a number of reasons why this may
have occurred. First, is that they were not present in enough
individuals to be able to classify those with and without disease with a
high degree of accuracy. Second, is that our Random Forest models were
able to gather the same information from measures such as FIT or other
OTUs. It is also possible that both of these explanations could have
played a role. Regardless, our observations would suggest that an
individual's resident bacteria have a large role to play in disease
initiation and could change in a way that allows predictive models to
lower the positive probability of a lesion after removal {[}Figure 3C \&
3D{]}. It should be noted that our study does not argue against the
importance of these CRC associated bacteria in the pathogenesis of
disease but rather that they are not the main bacteria changing after
removal of lesion. In fact, it is possible that these CRC associated
bacteria are important in the transition from adenoma to carcinoma and
would be one explanation as to why in our data we not only see high
initial relative abundances in carcinoma and not adenoma individuals but
also large decreases in relative abundance in some of those with
carcinoma but not in those with adenoma after lesion removal {[}Figure
2{]}.

Many of the common OTUs between the two models had OTUs that
taxonomically classified to potential butyrate producers {[}Table S2{]}.
Another batch of OTUs classified to bacteria that can either degrade
polyphenols or are inhibited by them. Both butyrate and polyphenols are
thought to be protective against cancer in part by reducing inflammation
{[}13{]}. These protective compounds are derived from the breakdown of
fiber, fruits, and vegetables by resident gut microbes. One example of
this potential diet-microbiome-inflammation-polyp axis is that
\emph{Bacteroides}, which was highly prevalent in our models, are known
to be increased in those with high non-meat based protein consumption
{[}14{]}. High protein consumption in general has been linked with an
increased CRC risk {[}15{]}. Conversely, \emph{Bacteroides} are
inhibited by polyphenols which are derived from fruits and vegetables
{[}16{]}. Our data fits with the hypothesis that the microbial
metabolites from breakdown products within our own diet could not only
help to shape the existing community but also have an effect on CRC risk
and disease progression.

One limitation of our study is that we do not know whether individuals
who were still classified as positive by the lesion model eventually had
a subsequent CRC diagnosis. This information would help to strengthen
the case for our lesion model keeping a number of individuals above the
cutoff threshold even though at follow up they were diagnosed as no
longer having a lesion. Another limitation is that we do not know if
adding modern tests such as the stool DNA test {[}17{]} could help
improve our overall AUC. This study also drew heavily from those with
Caucasian ancestry making it possible that the observations may not be
representative of those with either Asian or African ancestry. Although
our training and test set are relatively large we still run the risk of
over-fitting or having a model that may not be representative of other
populations. We've done our best to safeguard against this by not only
running 10-fold cross validation but also having over 100 different
80/20 splits to try and mimic the type of variation that might be
expected to occur.

Interestingly, within the initial sample model the test data performed
better than the training data. This may have occurred because the
training AUC determined from 20 repeated 10 fold cross validation
removed samples at random and did not take into account that they were
matched samples. Another potential reason is that the model itself may
be over-fit since the total number of samples was not that large.
However, the lesion model did not suffer from these discrepancies.
Further independent studies need to be carried out to verify our
findings since we are dealing with correlations that may not be truly
representative of the pathogenesis of disease.

Despite these limitations our findings add to the existing scientific
knowledge on CRC and the bacterial microbiome: That there is a
measurable difference in the bacterial community after adenoma and
carcinoma removal. Further, the ability for machine learning algorithms
to take bacterial microbiome data and successfully lower positive
probability after adenoma and carcinoma removal provides evidence that
there are specific signatures, mostly attributable to commensal
organisms, associated with these lesions. Our data provides evidence
that commensal bacteria may be important in the development of polyps
and also potentially the transition from adenoma to carcinoma.

\newpage

\subsection{Methods}\label{methods}

\textbf{\emph{Study Design and Patient Sampling:}} The sampling and
design were similar to that reported in Baxter, et al {[}6{]}. In brief,
study exclusion involved those who had already undergone surgery,
radiation, or chemotherapy, had colorectal cancer before a baseline
fecal sample could be obtained, had IBD, a known hereditary
non-polyposis colorectal cancer, or familial adenomatous polyposis.
Samples used to build the models for prediction were collected either
prior to a colonoscopy or between 1 - 2 weeks after. The bacterial
microbiome has been shown to normalize back to a pre-colonscopy
community within this time period {[}18{]}. Our follow up data set had a
total of 67 individuals that not only had a sample as described but also
a follow up sample between 188 - 546 days after lesion removal and
treatment had been completed. This study was approved by the University
of Michigan Institutional Review Board. All study participants provided
informed consent and the study itself conformed to the guidelines set
out by the Helsinki Declaration.

\textbf{\emph{FIT and 16S rRNA Gene Sequencing:}} FIT was analyzed as
previously published using both OC FIT-CHEK and OC-Auto Micro 80
automated system (Polymedco Inc.) {[}19{]}. 16S rRNA gene sequencing was
completed as previously described by Kozich, et al. {[}20{]}. DNA
extraction used the 96 well Soil DNA isolation kit (MO BIO Laboratories)
and an epMotion 5075 automated pipetting system (Eppendorf). The V4
variable region was amplified and the resulting product was split
between three sequencing runs with normal, adenoma, and carcinoma evenly
represented on each run. Each group was randomly assigned to avoid
biases based on sample collection location.

\textbf{\emph{Sequence Processing:}} The mothur software package
(v1.37.5) was used to process the 16S rRNA gene sequences. This process
has been previously described {[}20{]}. The general processing workflow
using mothur was as follows: Paired-end reads were first merged into
contigs, quality filtered, aligned to the SILVA database, screened for
chimeras, classified with a naive Bayesian classifier using the
Ribosomal Database Project (RDP), and clustered into Operational
Taxonomic Units (OTUs) using a 97\% similarity cutoff with an average
neighbor clustering algorithm. The number of sequences for each sample
was rarefied to 10523 in an attempt to minimize uneven sampling.

\textbf{\emph{Lesion Model Creation:}} The Random Forest {[}21{]}
algorithm was used to create the model used for prediction of lesion
(adenoma or carcinoma) with the main training and testing of the model
completed on an independent data set of 423 individuals. This model was
then applied to our follow up data set of 67 individuals. It should be
noted that all individuals with an adenoma or carcinoma were grouped
together to form the lesion group and the model was not created to find
differences between normal, adenoma, and carcinoma but rather
differences between both adenoma and carcinoma versus normal.

In brief, the model included data on FIT and the bacterial microbiome.
Non-binary data was checked for near zero variance and OTUs that had
near zero variance were removed. This pre-processing was performed with
the R package caret (v6.0.73). Optimization of the mtry hyper-parameter
involved taking the samples and making 100 different 80/20 (train/test)
splits of the data where normal and lesion were represented in the same
proportion within both the whole data set and the 80/20 split. Each of
these splits were then run through 20 repeated 10-fold cross validations
to optimize the mtry hyper-parameter by maximizing the AUC (Area Under
the Curve of the Receiver Operator Characteristic). This resulting model
was then tested on the 20\% of the data that was originally held out
from this overall process. Next, in order to assess which variables were
most important to the model we counted the number of times a variable
was present in the top 10\% of mean decrease in accuracy (MDA) for each
of the 100 different 80/20 split models and then filtered this list to
variables that were only present more than 50\% of the time. This final
collated list of variables was what was considered the most important
for the model. This reduced data set was then run through the mtry
optimization again. Once the ideal mtry was found the entire 423 sample
set was used to create the final Random Forest model on which
classifications on the 67-person cohort was completed.

The default cutoff of 0.5 was used as the threshold to classify
individuals as positive or negative for lesion. The hyper-parameter,
mtry, defines the number of variables to investigate at each split
before a new division of the data was created with the Random Forest
model.

\textbf{\emph{Initial Sample Model Creation:}} We also investigated
whether a model could be created that could identify pre (initial) and
post (follow up) lesion removal samples from each other. The main
difference was that only the 67-person cohort was used at all stages of
model building and classification. Other than this difference the
creation of this model and optimization of the mtry hyper-parameter was
completed using the same procedure as was used for the lesion model.
Instead of classifying samples as positive or negative of lesion this
model classified samples as positive or negative for being an initial
sample prior to lesion removal.

\textbf{\emph{Statistical Analysis:}} The R software package (v3.3.2)
was used for all statistical analysis. Comparisons between bacterial
community structure utilized PERMANOVA {[}22{]} in the vegan package
(v2.4.1). Comparisons between probabilities as well as overall OTU
differences between initial and follow up samples utilized a paired
Wilcoxson ranked sum test. Where multiple comparison testing was
appropriate, a Benjamini-Hochberg (BH) correction was applied {[}23{]}
and a corrected P-value of less than 0.05 was considered significant.
Unless otherwise stated the P-values reported are those that were BH
corrected.

\textbf{\emph{Analysis Overview:}} We first wanted to test if there were
any differences based on whether the individual had an adenoma or
carcinoma. This was done by testing initial and follow up samples for
differences in alpha and beta diversity, testing differences in FIT
between initial and follow ups, testing all OTUs, and investigating the
relative abundance of specific previously associated CRC bacteria
(\emph{Fusobacterium nucleatum}, \emph{Parvimonas micra},
\emph{Peptostreptococcus assacharolytica}, and \emph{Porphyromonas
stomatis}) based on adenoma or carcinoma. From here the lesion model was
then tested for accuracy in prediction and whether it reduced the
positive probability of lesion after surgery. We then used the initial
sample model to assess whether it could classify samples better then the
lesion model and whether it could reduce the positive probability of an
initial sample in the follow up samples. Finally, a list of common OTUs
were found for the two different models used.

\textbf{\emph{Reproducible Methods:}} A detailed and reproducible
description of how the data were processed and analyzed can be found at
\url{https://github.com/SchlossLab/Sze_followUps_2017}. Raw sequences
have been deposited into the NCBI Sequence Read Archive (SRP062005 and
SRP096978) and the necessary metadata can be found at
\url{https://www.ncbi.nlm.nih.gov/Traces/study/} and searching the
respective SRA study accession.

\newpage

\textbf{Figure 1: General Differences between the Adenoma and Carcinoma
Group.} A) A significant difference was found between the adenoma and
carcinoma group for thetayc (P-value = 0.000472). Advanced adenomas are
denoted as Screen Relevant Neoplasia (SRN). B) A significant difference
was found between the adenoma and carcinoma group for change in FIT
measurement (P-value = 2.15e-05). Advanced adenomas are denoted as
Screen Relevant Neoplasia (SRN). C) NMDS of the initial and follow up
samples for the adenoma group. D) NMDS of the initial and follow up
samples for the carcinoma group.

\textbf{Figure 2: Previously Associated CRC Bacteria in Initial and
Follow Up Samples.} A) Carcinoma initial and follow up samples had an
observed significant difference in initial and follow up sample for the
OTUs classified as \emph{Parvimonas micra} (P-value = 0.0116) and
\emph{Porphyromonas asaccharolytica} (P-value = 0.00842). B) Adenoma
initial and follow up samples. There were no significant differences
between initial and follow up (P-value = 0.37.

\textbf{Figure 3: The Lesion Model.} A) ROC curve: The shaded areas
represents the range of values of a 100 different 80/20 splits of the
test set data and the blue line represents the model using 100\% of the
data set and what was used for subsequent classification. B) Summary of
Important Variables. MDA of the most important variables in the lesion
model. The black point represents the mean and the different colors are
the values of each different run up to 100. C) Positive probability
change from initial to follow up sample in those with carcinoma. D)
Positive probability change from initial to follow up sample if those
with adenoma or advanced adenoma (Screen Relevant Neoplasia (SRN)).

\textbf{Figure 4: The Initial Sample Model.} A) ROC curve: The shaded
areas represents the range of values of a 100 different 80/20 splits of
the test set data and the blue line represents the model using 100\% of
the data set and what was used for subsequent classification. B) Summary
of Important Variables. MDA of the most important variables in the
initial sample model. The black point represents the mean and the
different colors are the values of each different run up to 100. C)
Positive probability change from initial to follow up sample in those
with carcinoma. D) Positive probability change from initial to follow up
sample of those with adenoma or advanced adenoma (Screen Relevant
Neoplasia (SRN)).

\newpage

\textbf{Figure S1: Distribution of P-values from Paired Wilcoxson
Analysis of All OTUs for Initial versus Follow Up}

\textbf{Figure S2: Thetayc Versus Time of Follow up Sample from Initial}

\newpage

\subsection{Declarations}\label{declarations}

\subsubsection{Ethics approval and consent to
participate}\label{ethics-approval-and-consent-to-participate}

The University of Michigan Institutional Review Board approved this
study, and all subjects provided informed consent. This study conformed
to the guidelines of the Helsinki Declaration.

\subsubsection{Consent for publication}\label{consent-for-publication}

Not applicable.

\subsubsection{Availability of data and
material}\label{availability-of-data-and-material}

A detailed and reproducible description of how the data were processed
and analyzed can be found at
\url{https://github.com/SchlossLab/Sze_followUps_2017}. Raw sequences
have been deposited into the NCBI Sequence Read Archive (SRP062005 and
SRP096978) and the necessary metadata can be found at
\url{https://www.ncbi.nlm.nih.gov/Traces/study/} and searching the
respective SRA study accession.

\subsubsection{Competing Interests}\label{competing-interests}

All authors declare that they do not have any relevant competing
interests to report.

\subsubsection{Funding}\label{funding}

This study was supported by funding from the National Institutes of
Health to P. Schloss (R01GM099514, P30DK034933) and to the Early
Detection Research Network (U01CA86400).

\subsubsection{Authors' contributions}\label{authors-contributions}

All authors were involved in the conception and design of the study. MAS
analyzed the data. NTB processed samples and analyzed the data. All
authors interpreted the data. MAS and PDS wrote the manuscript. All
authors reviewed and revised the manuscript. All authors read and
approved the final manuscript.

\subsubsection{Acknowledgements}\label{acknowledgements}

The authors thank the Great Lakes-New England Early Detection Research
Network for providing the fecal samples that were used in this study. We
would also like to thank Amanda Elmore for reviewing and correcting code
error and providing feedback on manuscript drafts. We would also like to
thank Nicholas Lesniak for providing feedback on manuscript drafts.

\newpage

\subsection*{References}\label{references}
\addcontentsline{toc}{subsection}{References}

\hypertarget{refs}{}
\hypertarget{ref-jemal_cancer_2010}{}
1. Jemal A, Siegel R, Xu J, Ward E. Cancer statistics, 2010. CA: a
cancer journal for clinicians. 2010;60:277--300.

\hypertarget{ref-haggar_colorectal_2009}{}
2. Haggar FA, Boushey RP. Colorectal cancer epidemiology: Incidence,
mortality, survival, and risk factors. Clinics in Colon and Rectal
Surgery. 2009;22:191--7.

\hypertarget{ref-zackular_manipulation_2016}{}
3. Zackular JP, Baxter NT, Chen GY, Schloss PD. Manipulation of the Gut
Microbiota Reveals Role in Colon Tumorigenesis. mSphere. 2016;1.

\hypertarget{ref-arthur_microbial_2014}{}
4. Arthur JC, Gharaibeh RZ, Mühlbauer M, Perez-Chanona E, Uronis JM,
McCafferty J, et al. Microbial genomic analysis reveals the essential
role of inflammation in bacteria-induced colorectal cancer. Nature
Communications. 2014;5:4724.

\hypertarget{ref-dejea_microbiota_2014}{}
5. Dejea CM, Wick EC, Hechenbleikner EM, White JR, Mark Welch JL,
Rossetti BJ, et al. Microbiota organization is a distinct feature of
proximal colorectal cancers. Proceedings of the National Academy of
Sciences of the United States of America. 2014;111:18321--6.

\hypertarget{ref-baxter_microbiota-based_2016}{}
6. Baxter NT, Ruffin MT, Rogers MAM, Schloss PD. Microbiota-based model
improves the sensitivity of fecal immunochemical test for detecting
colonic lesions. Genome Medicine. 2016;8:37.

\hypertarget{ref-zeller_potential_2014}{}
7. Zeller G, Tap J, Voigt AY, Sunagawa S, Kultima JR, Costea PI, et al.
Potential of fecal microbiota for early-stage detection of colorectal
cancer. Molecular Systems Biology. 2014;10:766.

\hypertarget{ref-flynn_metabolic_2016}{}
8. Flynn KJ, Baxter NT, Schloss PD. Metabolic and Community Synergy of
Oral Bacteria in Colorectal Cancer. mSphere. 2016;1.

\hypertarget{ref-yu_metagenomic_2017}{}
9. Yu J, Feng Q, Wong SH, Zhang D, Liang QY, Qin Y, et al. Metagenomic
analysis of faecal microbiome as a tool towards targeted non-invasive
biomarkers for colorectal cancer. Gut. 2017;66:70--8.

\hypertarget{ref-zackular_human_2014}{}
10. Zackular JP, Rogers MAM, Ruffin MT, Schloss PD. The human gut
microbiome as a screening tool for colorectal cancer. Cancer Prevention
Research (Philadelphia, Pa.). 2014;7:1112--21.

\hypertarget{ref-warren_co-occurrence_2013}{}
11. Warren RL, Freeman DJ, Pleasance S, Watson P, Moore RA, Cochrane K,
et al. Co-occurrence of anaerobic bacteria in colorectal carcinomas.
Microbiome. 2013;1:16.

\hypertarget{ref-arthur_intestinal_2012}{}
12. Arthur JC, Perez-Chanona E, Mühlbauer M, Tomkovich S, Uronis JM, Fan
T-J, et al. Intestinal inflammation targets cancer-inducing activity of
the microbiota. Science (New York, N.Y.). 2012;338:120--3.

\hypertarget{ref-louis_gut_2014}{}
13. Louis P, Hold GL, Flint HJ. The gut microbiota, bacterial
metabolites and colorectal cancer. Nature Reviews Microbiology
{[}Internet{]}. 2014 {[}cited 2017 Feb 14{]};12:661--72. Available from:
\url{http://www.nature.com/doifinder/10.1038/nrmicro3344}

\hypertarget{ref-zhu_intake_2016}{}
14. Zhu Y, Lin X, Li H, Li Y, Shi X, Zhao F, et al. Intake of Meat
Proteins Substantially Increased the Relative Abundance of Genus
Lactobacillus in Rat Feces. PloS One. 2016;11:e0152678.

\hypertarget{ref-mu_colonic_2016}{}
15. Mu C, Yang Y, Luo Z, Guan L, Zhu W. The Colonic Microbiome and
Epithelial Transcriptome Are Altered in Rats Fed a High-Protein Diet
Compared with a Normal-Protein Diet. The Journal of Nutrition.
2016;146:474--83.

\hypertarget{ref-ozdal_reciprocal_2016}{}
16. Ozdal T, Sela DA, Xiao J, Boyacioglu D, Chen F, Capanoglu E. The
Reciprocal Interactions between Polyphenols and Gut Microbiota and
Effects on Bioaccessibility. Nutrients {[}Internet{]}. 2016 {[}cited
2017 Feb 14{]};8:78. Available from:
\url{http://www.mdpi.com/2072-6643/8/2/78}

\hypertarget{ref-cotter_long-term_2016}{}
17. Cotter TG, Burger KN, Devens ME, Simonson JA, Lowrie KL, Heigh RI,
et al. Long-Term Follow-up of Patients Having False Positive
Multi-target Stool DNA Tests after Negative Screening Colonoscopy: The
LONG-HAUL Cohort Study. Cancer Epidemiology, Biomarkers \& Prevention: A
Publication of the American Association for Cancer Research, Cosponsored
by the American Society of Preventive Oncology. 2016;

\hypertarget{ref-obrien_impact_2013}{}
18. O'Brien CL, Allison GE, Grimpen F, Pavli P. Impact of colonoscopy
bowel preparation on intestinal microbiota. PloS One. 2013;8:e62815.

\hypertarget{ref-daly_evaluation_2013}{}
19. Daly JM, Bay CP, Levy BT. Evaluation of fecal immunochemical tests
for colorectal cancer screening. Journal of Primary Care \& Community
Health. 2013;4:245--50.

\hypertarget{ref-kozich_development_2013}{}
20. Kozich JJ, Westcott SL, Baxter NT, Highlander SK, Schloss PD.
Development of a dual-index sequencing strategy and curation pipeline
for analyzing amplicon sequence data on the MiSeq Illumina sequencing
platform. Applied and Environmental Microbiology. 2013;79:5112--20.

\hypertarget{ref-breiman_random_2001}{}
21. Breiman L. Random Forests. Machine Learning {[}Internet{]}. 2001
{[}cited 2013 Feb 7{]};45:5--32. Available from:
\href{http://link.springer.com/article/10.1023/A\%3A1010933404324\%20http://link.springer.com/article/10.1023\%2FA\%3A1010933404324?LI=true}{http://link.springer.com/article/10.1023/A\%3A1010933404324 http://link.springer.com/article/10.1023\%2FA\%3A1010933404324?LI=true}

\hypertarget{ref-anderson_permanova_2013}{}
22. Anderson MJ, Walsh DCI. PERMANOVA, ANOSIM, and the Mantel test in
the face of heterogeneous dispersions: What null hypothesis are you
testing? Ecological Monographs {[}Internet{]}. 2013 {[}cited 2017 Jan
5{]};83:557--74. Available from:
\url{http://doi.wiley.com/10.1890/12-2010.1}

\hypertarget{ref-benjamini_controlling_1995}{}
23. Benjamini Y, Hochberg Y. Controlling the false discovery rate: A
practical and powerful approach to multiple testing. Journal of the
Royal Statistical Society. Series B (Methodological). 1995;57:289--300.


\end{document}
