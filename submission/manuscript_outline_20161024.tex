\documentclass[12pt,]{article}
\usepackage{lmodern}
\usepackage{amssymb,amsmath}
\usepackage{ifxetex,ifluatex}
\usepackage{fixltx2e} % provides \textsubscript
\ifnum 0\ifxetex 1\fi\ifluatex 1\fi=0 % if pdftex
  \usepackage[T1]{fontenc}
  \usepackage[utf8]{inputenc}
\else % if luatex or xelatex
  \ifxetex
    \usepackage{mathspec}
  \else
    \usepackage{fontspec}
  \fi
  \defaultfontfeatures{Ligatures=TeX,Scale=MatchLowercase}
\fi
% use upquote if available, for straight quotes in verbatim environments
\IfFileExists{upquote.sty}{\usepackage{upquote}}{}
% use microtype if available
\IfFileExists{microtype.sty}{%
\usepackage{microtype}
\UseMicrotypeSet[protrusion]{basicmath} % disable protrusion for tt fonts
}{}
\usepackage[margin=1.0in]{geometry}
\usepackage{hyperref}
\hypersetup{unicode=true,
            pdfborder={0 0 0},
            breaklinks=true}
\urlstyle{same}  % don't use monospace font for urls
\usepackage{graphicx,grffile}
\makeatletter
\def\maxwidth{\ifdim\Gin@nat@width>\linewidth\linewidth\else\Gin@nat@width\fi}
\def\maxheight{\ifdim\Gin@nat@height>\textheight\textheight\else\Gin@nat@height\fi}
\makeatother
% Scale images if necessary, so that they will not overflow the page
% margins by default, and it is still possible to overwrite the defaults
% using explicit options in \includegraphics[width, height, ...]{}
\setkeys{Gin}{width=\maxwidth,height=\maxheight,keepaspectratio}
\IfFileExists{parskip.sty}{%
\usepackage{parskip}
}{% else
\setlength{\parindent}{0pt}
\setlength{\parskip}{6pt plus 2pt minus 1pt}
}
\setlength{\emergencystretch}{3em}  % prevent overfull lines
\providecommand{\tightlist}{%
  \setlength{\itemsep}{0pt}\setlength{\parskip}{0pt}}
\setcounter{secnumdepth}{0}
% Redefines (sub)paragraphs to behave more like sections
\ifx\paragraph\undefined\else
\let\oldparagraph\paragraph
\renewcommand{\paragraph}[1]{\oldparagraph{#1}\mbox{}}
\fi
\ifx\subparagraph\undefined\else
\let\oldsubparagraph\subparagraph
\renewcommand{\subparagraph}[1]{\oldsubparagraph{#1}\mbox{}}
\fi

%%% Use protect on footnotes to avoid problems with footnotes in titles
\let\rmarkdownfootnote\footnote%
\def\footnote{\protect\rmarkdownfootnote}

%%% Change title format to be more compact
\usepackage{titling}

% Create subtitle command for use in maketitle
\newcommand{\subtitle}[1]{
  \posttitle{
    \begin{center}\large#1\end{center}
    }
}

\setlength{\droptitle}{-2em}
  \title{}
  \pretitle{\vspace{\droptitle}}
  \posttitle{}
  \author{}
  \preauthor{}\postauthor{}
  \date{}
  \predate{}\postdate{}

\usepackage{helvet} % Helvetica font
\renewcommand*\familydefault{\sfdefault} % Use the sans serif version of the font
\usepackage[T1]{fontenc}

\usepackage[none]{hyphenat}

\usepackage{setspace}
\doublespacing
\setlength{\parskip}{1em}

\usepackage{lineno}

\usepackage{pdfpages}
\usepackage{comment}

\begin{document}

\section{Differences in the fecal Microbiome Before and After Colorectal
Cancer
Treatment}\label{differences-in-the-fecal-microbiome-before-and-after-colorectal-cancer-treatment}

\vspace{25mm}

\begin{center}
Running Title: Human Microbiome and Colorectal Cancer

\vspace{10mm}

Marc A Sze${^1}$, Nielson T Baxter${^2}$, Mack T Ruffin IV${^3}$, Mary AM Rogers${^2}$, and Patrick D Schloss${^1}$${^\dagger}$

\vspace{20mm}

$\dagger$ To whom correspondence should be addressed: pschloss@umich.edu

$1$ Department of Microbiology and Immunology, University of Michigan, Ann Arbor, MI

$2$ Department of Internal Medicine, University of Michigan, Ann Arbor, MI   

$3$ Department of Family Medicine and Community Medicine, Penn State Hershey Medical Center, Hershey, PA    


\end{center}

\newpage

\linenumbers

\subsection{Abstract}\label{abstract}

\textbf{Background:} Colorectal cancer (CRC) continues to be a worldwide
health problem with previous research suggesting that a link may exist
between the fecal bacterial microbiome and CRC. The overall objective of
our study was to test the hypothesis that changes in the bacterial
microbiome occur after surgery in patients with lesions (i.e.~adenoma or
carcinoma). Specifically, we wanted to identify what within the
community was different before and after surgical removal of said
lesion.

\textbf{Results:} The bacterial microbiome in pre and post surgery
samples of individuals with adenoma were more similar to each other than
those with carcinoma (P-value \textless{} 0.05). There was no difference
in the relative abundance of any OTU between the pre and post surgery
samples (P-value \textgreater{} 0.05). A model with a total of 53
variables was able to classify lesion (AUC = 0.811 - 0.866) while a
model built to classify pre surgery samples had 70 variables (AUC =
0.641 - 0.805). The post surgery sample for both models had a decrease
in the positive probability for either lesion or pre surgery sample
(P-value \textless{} 0.05). In total there were 23 OTUs that were common
to both models and the majority of these classified to commensal
bacteria (e.g. \emph{Bacteroides}, \emph{Clostridiales}, \emph{Blautia},
and \emph{Ruminococcaceae}).

\textbf{Conclusions:} Our data supports the hypothesis that there are
differences in the bacterial microbiome between pre and post surgical
samples. With individuals with carcinoma having more drastic differences
to the overall community then those with adenoma. Changes to commensal
bacteria were some of the most important variables for model
classification, suggesting that these bacteria may be central to initial
polyp formation and transition to carcinoma.

\newpage

\subsubsection{Keywords}\label{keywords}

bacterial microbiome; colorectal cancer; polyps; FIT; post surgery; risk
factors

\newpage

\subsection{Background}\label{background}

Colorectal cancer (CRC) continues to be a leading cause of cancer
related deaths and is currenlty the third most common cause of cancer
deaths {[}1,2{]}. Over the last few years death due to the disease has
seen a significant decrease, thanks mainly to improvements in screening
{[}1{]}. However, despite this improvement there are still approximately
50,000 deaths from the disease a year {[}2{]}.

Recently, there has been promising work on the bacterial microbiome and
it's ability to be able to complement existing screening methods such as
Fecal Immunoglobulin Test (FIT) or act alone as a screening tool
{[}3,4{]}. There has also been research into how this microbiome could
be altered directly on tumor tissue itself {[}5{]}. A few studies have
also shown how this microbiome {[}6{]} or specific members within it
{[}7{]} could be directly involved with the pathogenesis of CRC. These
studies have helped to provide a tantalizing link between the bacterial
microbiome and CRC. Although these studies suggest that the bacterial
microbiome might change after treatment there remains limited
information on the bacterial microbiome before and after surgerical
removal of lesion (adenoma or carcinoma) and whether the community
changes at all.

In this study we tested the hypothesis that the bacterial microbiome
changes after surgery for individuals with a lesion. Our analysis
included both alpha and beta diversity analysis along with investigation
of individual operational taxonomic units (OTUs). We also utilized
Random Forest models and observed how these models as well as specific
OTUs within them performed pre (initial) and post (follow up) surgery.
We also used these models to look for similar important OTUs to identify
the crucial OTUs for not only classifying initial and follow up samples
but also lesion or normal.

\newpage

\subsection{Results}\label{results}

\textbf{\emph{Bacterial Community and FIT:}} We first wanted to test
whether there were any broad differences between initial and follow up
samples based on lesion being either adenoma or carcinoma. What we found
was that the bacterial community in those with carcinoma were more
dissimilar (as measured by thetayc) to their initial sample then those
with adenoma (P-value \textless{} 0.001) {[}Figure 1a{]}. We also found
that there were larger changes in fecal blood (measured by FIT) for
those with carcinoma versus adenoma (P-value \textless{} 0.0001)
{[}Figure 1b{]}. The broad shift in bacterial community structure before
and after surgery was visualized using NMDS for both adenoma
{[}Figure1c{]} (PERMANOVA \textgreater{} 0.05) and carcinoma {[}Figure
1d{]} (PERMANOVA \textless{} 0.05). Interestingly, when initial and
follow up samples were compared to each other, regardless of whether
they were adenoma or carcinoma (lesion), there was no significant
overall difference between them (PERMANOVA \textgreater{} 0.05). When
investigating more broad alpha diversity metrics there was no difference
found between initial and follow up samples for lesion, adenoma only, or
carcinoma only for any metric tested {[}Table S1{]}. We also observed
that there was no difference in OTU relative abundance between initial
and follow up samples for lesion, adenoma only, or carcinoma only
{[}Figure S1{]}.

\textbf{\emph{Cancer Associated Bacteria:}} Previous literature has
suggested that a number of oral microbes may be important in CRC
pathogenesis {[}8{]}. So we next examined whether there were differences
in previously well described carcinoma associated OTUs. These included
the OTUs that aligned with \emph{Porphyromonas asaccharolytica}
(Otu000202), \emph{Fusobacterium nucleatum} (Otu000442),
\emph{Parvimonas micra} (Otu001273), and \emph{Peptostreptococcus
stomatis} (Otu001682). There was a difference in relative abundance in
initial and follow up samples for lesion and carcinoma for
\emph{Parvimonas micra} (P-value \textless{} 0.05), and
\emph{Porphyromonas asaccharolytica} (P-value \textless{} 0.05). In
contrast, there was no difference in relative abundance in any of these
OTUs for those with adenoma {[}Figure 2{]}. We also observed that only a
small percentage of those with adenoma or carcinoma were positive or had
an appreciable relative abundance of any of these respective OTUs
{[}Figure 2{]}.

\textbf{\emph{Full and Reduced Model:}} We next wanted to identify if
there were any common bacterial microbiome changes in individuals with
adenoma or carcinoma. In order to investigate this we created two
different models: one to classify lesion versus normal and one to
classify pre (initial) versus post (follow up) samples based on the
bacterial community and FIT measurements. The lesion model had an AUC
range of 0.73 to 0.797 while the initial sample model had an AUC range
of 0.485 to 0.686 after 100 iterations of 20 repeated 10-fold cross
validations. By identifying the most important variables for each
respective model and then reducing them to only these factors we were
able to increase the AUC in the lesion model (0.811 - 0.866) and initial
sample model (0.641 - 0.805).\\
The test set AUC range for the full and reduced lesion model were
similar to that reported for the training set AUC ranges and the ROC
curve ranges overlap with each other {[}Figure 3a{]}. The ROC curve for
the final lesion model used was within the range of the reduced lesion
model {[}Figure 3a{]}. Interestingly, the test set AUC range for the
initial sample model performed much better then the training set AUCs.
Both the full and reduced initial sample models overlapped with each
other {[}Figure 3b{]} but there was a marked decrease in the ROC curve
for the final before sample model used.

\textbf{\emph{Common OTUs to both Models:}} The reduced models were
built based on the most important variables to either classification
model. For the lesion model there were a total of 53 variables {[}Figure
S2{]} whereas for the initial sample model there were a total of 70
variables {[}Figure S3{]}. For both models FIT measurement resulted in
one of the largest decreases in MDA {[}Figure S2a \& S3a{]}. When we
compared the two different reduced models with each other there were a
total of 23 common OTUs. Some of the most common taxonomic
identifications belonged to \emph{Bacteroides}, \emph{Clostridiales},
\emph{Blautia}, and \emph{Ruminococcaceae}. The vast majority of these
OTUs had classifications to bacteria typically thought of as commensal
{[}Table S2{]}.

\textbf{\emph{Positive Probability after Lesion Removal:}} If there were
common OTUs for individuals with adenoma and carcinoma, that were
different versus normal controls, we would expect to find a decrease in
the positive probability of the follow up sample to be either lesion or
an initial sample. This is what we observed regardless of model used
(lesion or initial sample) or whether it was built on the full or
reduced variable data set {[}Figure 4 \& S4{]} (P-value \textless{}
0.001).

When we seperated individuals based on whether they had an adenoma or
carcinoma there was only a decrease in positive probability for the
carcinoma group (P-value \textless{} 0.001) and not for the adenoma
group (P-value \textgreater{} 0.05). We also observed that there were no
significant differences in whether the models classified the samples as
having lesion between the predicted and actual (P-value \textgreater{}
0.05). This lack of difference between the predicted and actual
classifications were also observed for the initial sample model (P-value
\textgreater{} 0.05). Even though the lesion model was not as accurate
in classifying samples as the initial sample model. It was able to
correctly keep the one individual who still had a carcinoma on follow up
above the cut off threshold for a positive call {[}Figure 4a \& S4a{]}
while the initial sample model did not {[}Figure 4b \& S4b{]}.

\textbf{\emph{Treatment and Time Differences:}} After observing these
changes in the bacterial community and positive probability we wanted to
assess whether additional treatments, such as chemotherapy and
radiation, could have an impact on the results that we observed. There
was no difference in the amount of change in positive probability for
either the full or reduced lesion model for either chemotherapy (P-value
\textgreater{} 0.05) or radiation therapy (P-value \textgreater{} 0.05).
In contrast, we observed for the the before sample model a significant
difference in decreased positive probability for those treated with
chemotherapy (P-value \textless{} 0.05). All other variables that were
tested showed no difference based on whether chemotherapy or radiation
was used {[}Table S3{]}. Finally, we wanted to know if the length of
time between the initial and follow up sample could be a possible
confounder. Within our study there was a significant difference for the
time elapsed in the collection of the follow up sample between adenoma
and carcinoma (uncorrected P-value \textless{} 0.05), with time passed
being less for adenoma (253 +/- 41.3 days) then carcinoma (351 +/- 102
days).

\newpage

\subsection{Discussion}\label{discussion}

From our results there were some large observed differences in the
bacterial microbiome between pre and post surgery samples based on
whether the individual had an adenoma or carcinoma. There was much
larger differences between initial and follow up samples based on the
thetayc distance metric and in fecal blood as measured by FIT for
individuals with carcinoma versus adenoma {[}Figure 1{]}. However, there
were no differences between initial and follow up samples for Shannon
Diversity, observed OTUs, or evenness regardless of whether the
individual had an adenoma or carcinoma {[}Table S1{]}. There was also no
differences in relative abundance of any specific OTU for lesion,
adenoma only, or carcinoma only {[}Figure s1{]}.

Although there were no differences when investigating all OTUs, when
looking specifically at four OTUs that taxonomically classified to
previously suggested cancer causing microbes we found that only 2/4 had
a decrease in relative abundance between initial and follow up for those
with carcinoma and 0/4 had differences for those with adenoma. This data
would suggest that these specific OTUs may be important in the
transition of an adenoma to a carcinoma but less so in the initiation of
an adenoma from benign tissue.

We next created a model that incorporated FIT and the bacterial
microbiome to either be able to classify lesions (adenoma or carcinoma)
or initial samples in order to find common OTUs in the community that
change for both adenoma and carcinoma. What we found was that the
commonly associated CRC bacteria were not higly represented within our
models but rather that OTUs that made up the most important variables
overwhelmingly belonged to commensal bacteria. With only the lesion
model having a single OTU from a previously associated cancer bacterium
(\emph{Porphyromonas asaccharolytica}). Using only these important OTUs
and FIT both models (lesion and before sample) significantly decreased
positive probability of either lesion or being an initial sample on
follow up {[}Figure 4 \& S4{]}. Further confirmation of the importance
of the changes of commensal bacteria to these classifications was that a
total of 23 OTUs were common to both models and the vast majority
belonged to regular residents of our gut community.

For the majority of tests performed there were no differences in the
bacterial microbiome based on whether chemotherapy or radiation was
received {[}Table S3{]}. There was a difference in the length of time
between initial and follow up sample between adenoma and carcinoma.
These results would indicate that the findings described were specific
to the surgical intervention and that some of the differences observed
between carcinoma and adenoma samples could be due to differences in
collection time between samples for the two different groups.

This study builds upon previous work from numerous labs that have looked
into the bacterial microbiome as a potential screening tool {[}3,4{]} by
exploring what happens to the bacterial community after surgical removal
of a lesion. Based on previous work by Arthur, et al. {[}9{]} it may not
be surprising to have E.coli as one of the most important OTUs and one
that was common to both models. Interestingly, many of the most
important OTUs had taxonomic identification for resident gut microbes.
This could suggest that the bacterial community is one of the first
components that could change during the pathogenesis of disease. These
bacterial microbiome changes could be the first step in allowing more
inflammatory bacterium to gain a foothold within the colon {[}8{]}.

Curiously, we observed that the typical CRC associated bacteria were not
predictive within our models. There are a number of reasons why this may
have occurred. First, is that they were not present in enough
individuals to be able to classify those with and without disease with a
high degree of accuracy. Second, is that it is possible that our Random
Forest models were able to gather the same information from measures
such as FIT or other OTUs. It is also possible that both of these
explanations could have played a role. Regardless, our observations
would suggest that an individual's resident bacteria have a large role
to play in disease initiation and could change in a way that allows
predictive models to lower the positive probability of a lesion after
surgery {[}Figure 4{]}. It should be noted that our study does not argue
against the importance of these CRC associated bacteria in the
pathogenesis of disease but rather that the models do not utilize these
specific bacteria for classification purposes (lesion or before sample).
In fact, it is possible that these CRC associated bacteria are important
in the transition from adenoma to carcinoma and would be one explanation
as to why in our data we not only see high initial relative abundances,
in certain individuals, but also large decreases in relative abundance
in those with carcinoma but not in those with adenoma after surgery
{[}Figure 2{]}.

Many of the common OTUs between the different models used had many OTUs
that taxonomically classifed to potential butyrate producers {[}Table
S2{]}. Another batch of OTUs classifed to bacteria that can either
degrade polyphenols or are inhibited by them. Both butyrate and
polyphenols are thought to protective against cancer in part by reducing
inflammation {[}10{]}. These protective compounds are derived from the
breakdown of fiber, fruits, and vegetables by resident gut microbes. One
example of this potential diet-microbiome-inflammation-polyp axis is
that \emph{Bacteroides}, which was highly prevalent in our models, are
known to be increased in those with high non-meat based protein
consumption {[}11{]}. High protein consumption in general has been
linked with an increased CRC risk {[}12{]}. Conversely,
\emph{Bacteroides} are inhibited by polyphenols which are derived from
fruits and vegetables {[}13{]}. Our data fits with the hypothesis that
the microbial metabolites from breakdown products within our own diet
could not only help to shape the exisiting community but also have an
effect on CRC risk and disease progression.

One limitation of our study is that we do not know whether individuals
who were still classified as positive by the lesion model eventually had
a subsequent CRC diagnosis. This information would help to strengthen
the case for our lesion model to have kept a number of individuals above
the cutoff threshold even though at follow up they were diagnosed as no
longer having a lesion. Another limitation is that we do not know if
adding modern tests such as the stool DNA test {[}14{]} could help
improve our overall AUC. Another limitation is that this study drew
heavily from those with Caucasian ancestry. The results may not be
immediately representative of those with either Asian or African
ancestry. Finally, although our training and test set are relatively
large we still run the risk of over-fitting or having a model that may
not be immediately extrapolate-able to other populations. We've done our
best to safeguard against this by not only running 10-fold cross
validation but also having over 100 different 80/20 splits to try and
mimic the type of variation that might be expected to occur. The time
difference in collection of sample between adenoma and carcinoma could
have affected our observed results for the differences between
individuals with adenoma or carcinoma. This confounding though does not
affect the observations based on the overall lesion results.

Another interesting outcome was that within figure 3 the before sample
model showed better test AUC results then the training set AUC. This may
have occurred because the training AUC that was determined from 20
repeated 10 fold cross validation removed samples at random and did not
take into account that they were matched samples. Another potential
reason is that the model itself may be over-fit since the total number
of samples was not that large. However, the lesion model did not suffer
from these discrepancies and similar conclusions can be drawn solely
from this model. Regardless, further independent studies will need to be
carried out to verify our findings since not only are we dealing with
feces, which could be very different than the communities present on the
actual tissue, but also are dealing with correlations that may not be
representative of the true pathogenesis of disease.

Despite these limitations we think that these findings significantly add
to the existing scientific knowledge on CRC and the bacterial
microbiome: That there is a measurable difference in the bacterial
community after surgical removal of lesion. Further, the ability for
machine learning algorithms to take bacterial microbiome data and
successfully lower positive probability after either adenoma or
carcinoma removal provides evidence that there are specific signatures,
mostly attributable to commensal organisms, associated with these
lesions. Our data provides evidence that commensal bacteria are
important in the development of polyps and also potentially the
transition from adenoma to carcinoma.

\newpage

\subsection{Methods}\label{methods}

\textbf{\emph{Study Design and Patient Sampling}} The sampling and
design of the study was similar to that reported in Baxter, et al
{[}3{]}. In brief, study exclusion involved those who had already
undergone surgery, radiation, or chemotherapy, had colorectal cancer
before a baseline fecal sample could be obtained, had IBD, a known
hereditary non-polyposis colorectal cancer, or Familial adenomatous
polyposis. Samples used to build the models for prediction were
collected either prior to a colonoscopy or between 1 - 2 weeks after.
The bacterial microbiome has been shown to normalize within this time
period {[}15{]}. Our follow up data set had a total of 67 individuals
that not only had a sample as described but also a follow up sample
between 188 - 546 days after surgery and treatment had been completed.
This study was approved by the University of Michigan Institutional
Review Board. All study participants provided informed consent and the
study itself conformed to the guidelines set out by the Helsinki
Declaration.

\textbf{\emph{FIT and 16S rRNA Gene Sequencing}} FIT was analyzed as
previously published using both OC FIT-CHEK and OC-Auto Micro 80
automated system (Polymedco Inc.) {[}16{]}. 16S rRNA gene sequencing was
completed as previously described by Kozich, et al. {[}17{]}. In brief,
DNA extraction used the 96 well Soil DNA isolation kit (MO BIO
Laboratories) and an epMotion 5075 automated pipetting system
(Eppendorf). The V4 variable region was amplified and the resulting
product was split between three sequencing runs with normal, adenoma,
and carcinoma evenly represented on each run. Each group was randomly
assigned to avoid biases based on sample collection location.

\textbf{\emph{Sequence Processing}} The mothur software package
(v1.37.5) was used to process the 16S rRNA gene sequences. This process
has been previously described {[}17{]}. The general processing workflow
using mothur was as follows: Paired-end reads were first merged into
contigs, quality filtered, aligned to the SILVA database, screened for
chimeras, classified with a naive Bayesian classifier using the
Ribosomal Database Project (RDP), and clustered into Operational
Taxonomic Units (OTUs) using a 97\% similarity cutoff with an average
neighbor clustering algorithm. The number of sequences for each sample
was rarefied to 10523 in an attempt to minimize uneven sampling.

\textbf{\emph{Lesion Model Creation}} The Random Forest {[}18{]}
algorithm was used to create the model used for prediction of lesion
(adenoma or carcinoma) with the main testing and training of the model
completed on a data set of 490 individuals. This model was then applied
to our follow up data set of 67 individuals. The model included data on
FIT and the bacterial microbiome. Non-binary data was checked for near
zero variance and OTUs that had near zero variance were removed. This
pre-processing was performed with the R package caret (v6.0.73).
Optimization of the mtry hyper-parameter involved taking the samples and
making 100 different 80/20 (train/test) splits of the data where normal
and lesion were represented in the same proportion within both the whole
data set and the 80/20 split. Each of these splits were then run through
20 repeated 10-fold cross validations to optimize the mtry
hyper-parameter by maximizing the AUC (Area Under the Curve of the
Receiver Operator Characteristic). This resulting model was then tested
on the 20\% of the data that was originally held out from this overall
process. Once the ideal mtry was found the entire 490 sample set was
used to create the final Random Forest model on which classifications on
the 67-person cohort was completed. The default cutoff of 0.5 was used
as the threshold to classify individuals as positive or negative for
lesion. The hyper-parameter, mtry, defines the number of variables to
investigate at each split before a new division of the data is created
with the Random Forest model.

\textbf{\emph{Before Sample Model Creation}} We also investigated
whether a model could be created that could identify before and after
surgery samples. The main difference was that only the 67-person cohort
was used at all stages of model building and classification. Other than
this difference the creation of this model and optimization of the mtry
hyper-parameter was completed using the same procedure that was used to
create the lesion model. Instead of classifying samples as positive or
negative of lesion this model classified samples as positive or negative
for being a before surgery sample.

\textbf{\emph{Selection of Important OTUs}} In order to assess which
variables were most important to all the models we counted the number of
times a variable was present in the top 10\% of mean decrease in
accuracy (MDA) for each of the 100 different 80/20 split models and then
filtered this list to variables that were only present more than 50\% of
the time. This final collated list of variables was what was considered
the most important for the lesion or before sample models.

\textbf{\emph{Statistical Analysis}} The R software package (v3.3.2) was
used for all statistical analysis. Comparisons between bacterial
community structure utilized PERMANOVA {[}19{]} in the vegan package
(v2.4.1). Comparisons between probabilities as well as overall OTU
differences between initial and follow up samples utilized a paired
Wilcoxson ranked sum test. Where multiple comparison testing was needed
a Benjamini-Hochberg (BH) correction was applied {[}20{]} and a
corrected P-value of less than 0.05 was considered significant. Unless
otherwise stated the P-values reported are those that were BH corrected.

\textbf{\emph{Analysis Overview}} Initial and follow up samples were
analyzed for differences in alpha and beta diversity. Next, differences
in FIT between initial and follow ups for either adenoma or carcinoma
were investigated. From here, all OTUs that were used in either model
were then analyzed using a paired Wilcoxson test. We then investigated
the relative abundance of specific previously associated CRC bacteria,
specifically, OTUs that taxonomically classified to \emph{Fusobacterium
nucleatum}, \emph{Parvimonas micra}, \emph{Peptostreptococcus
assacharolytica}, and \emph{Porphyromonas stomatis}. We wanted to test
if there were any differences based on whether the individual had an
adenoma or carcinoma. From here the lesion model was then tested for
accuracy in prediction and whether it reduced the positive probability
of lesion after surgery. The most important OTUs for this were used to
build a reduced model and it was assessed for similarity to the original
model. We then used the before sample model to assess whether it could
classify samples better then the lesion model. The most important OTUs
were then identified from this model and used to create a reduced
feature before sample model. This reduced feature model, as was done
with the lesion model, was compared to the full model for loss of
accuracy. Finally, a list of common OTUs were found for the two
different models used.

\textbf{\emph{Reproducible Methods.}} A detailed and reproducible
description of how the data were processed and analyzed can be found at
\url{https://github.com/SchlossLab/Sze_followUps_2017}. Raw sequences
have been deposited into the NCBI Sequence Read Archive (SRP062005 and
SRP096978) and the necessary metadata can be found at
\url{https://www.ncbi.nlm.nih.gov/Traces/study/} and searching the
respective SRA study accession.

\newpage

\textbf{Figure 1: General Differences between the Adenoma or Carcinoma
Group.} A) A significant difference was found between the adenoma and
carcinoma group for thetayc (P-value = 0.000472). B) A significant
difference was found between the adenoma and carcinoma group for change
in FIT (P-value = 2.15e-05). C) NMDS of the initial and follow up
samples for the Adenoma group. D) NMDS of the initial and follow up
samples for the Carcinoma group. For C) and D) the teal represents
initial samples and the pink represents follow up samples.

\textbf{Figure 2: Previously Associated CRC Bacteria in Initial and
Follow up Samples.} A) Carcinoma initial and follow up samples. There
was a significant difference in initial and follow up sample for the
OTUs classified as \emph{Peptostreptococcus stomatis} (P-value = 0.0496)
and \emph{Porphyromonas asaccharolytica} (P-value = 0.00842). B) Adenoma
initial and follow up samples. There were no significant differences
between initial and follow up.

\textbf{Figure 3: Graph of the Receiver Operating Characteristic Curve
for lesion and Before Sample Models.} The shaded areas represents the
range of values of a 100 different 80/20 splits of the test set data
using either all variables (grey) or reduced variable (red) models. The
blue line represents the reduced variable model using 100\% of the data
set. A) Lesion model. B) Before sample model

\textbf{Figure 4: Breakdown by Carcinoma and Adenoma of Prediction
Results for Lesion and Before Sample Reduced Variable Models} A) Lesion
positive probability adjustment of those with carcinoma from initial to
follow up sample B) Initial follow up positive probability adjustment of
those with carcinoma from initial to follow up sample C) Lesion positive
probability adjustment of those with adenoma as well as those with SRN
and the probability adjustment from initial to follow up sample. D)
Initial follow up positive probability adjustment of those with adenoma
as well as those with SRN and the probability adjustment from initial to
follow up sample. The dotted line represents the threshold used to make
the decision of whether a sample was positive or not.

\newpage

\textbf{Figure S1: Distribution of P-values from Paired Wilcoxson
Analysis of OTUs in Initial versus Follow Up}

\textbf{Figure S2: Summary of Important Variables in the Lesion Model}
A) MDA of the most important variables in the lesion model. The black
point represents the median and the different colors are the different
runs up to 100. B) The total number of appearances of each variable in
the 100 different lesion models. The cutoff of 50\% was used to assess
importance.

\textbf{Figure S3: Summary of Important Variables in Before Sample
Model} A) MDA of the most important variables in the lesion model. The
black point represents the median and the different colors are the
different runs up to 100. B) The total number of appearances of each
variable in the 100 different lesion models. The cutoff of 50\% was used
to assess importance.

\textbf{Figure S4: Breakdown by Carcinoma and Adenoma of Prediction
Results for Lesion and Before Sample Full Variable Models} A) Lesion
positive probability adjustment of those with carcinoma from initial to
follow up sample B) Initial follow up positive probability adjustment of
those with carcinoma from initial to follow up sample C) Lesion positive
probability adjustment of those with adenoma as well as those with SRN
and the probability adjustment from initial to follow up sample. D)
Initial follow up positive probability adjustment of those with adenoma
as well as those with SRN and the probability adjustment from initial to
follow up sample. The dotted line represents the threshold used to make
the decision of whether a sample was positive or not.

\textbf{Figure S5: Thetayc Graphed Against Time of Follow up Sample from
Initial}

\newpage

\subsection{Declarations}\label{declarations}

\subsubsection{Ethics approval and consent to
participate}\label{ethics-approval-and-consent-to-participate}

\subsubsection{Consent for publication}\label{consent-for-publication}

\subsubsection{Availability of data and
material}\label{availability-of-data-and-material}

\subsubsection{Competing Interests}\label{competing-interests}

All authors declare that they do not have any relevant competing
interests to report.

\subsubsection{Funding}\label{funding}

This study was supported by funding from the National Institutes of
Health to P. Schloss (R01GM099514, P30DK034933) and to the Early
Detection Research Network (U01CA86400).

\subsubsection{Authors' contributions}\label{authors-contributions}

All authors were involved in the conception and design of the study. MAS
analyzed the data. NTB processed samples and analyzed the data. All
authors interpreted the data. MAS and PDS wrote the manuscript. All
authors reviewed and revised the manuscript. All authors read and
approved the final manuscript.

\subsubsection{Acknowledgements}\label{acknowledgements}

The authors thank the Great Lakes-New England Early Detection Research
Network for providing the fecal samples that were used in this study.

\newpage

\subsection*{References}\label{references}
\addcontentsline{toc}{subsection}{References}

\hypertarget{refs}{}
\hypertarget{ref-jemal_cancer_2010}{}
1. Jemal A, Siegel R, Xu J, Ward E. Cancer statistics, 2010. CA: a
cancer journal for clinicians. 2010;60:277--300.

\hypertarget{ref-haggar_colorectal_2009}{}
2. Haggar FA, Boushey RP. Colorectal cancer epidemiology: Incidence,
mortality, survival, and risk factors. Clinics in Colon and Rectal
Surgery. 2009;22:191--7.

\hypertarget{ref-baxter_microbiota-based_2016}{}
3. Baxter NT, Ruffin MT, Rogers MAM, Schloss PD. Microbiota-based model
improves the sensitivity of fecal immunochemical test for detecting
colonic lesions. Genome Medicine. 2016;8:37.

\hypertarget{ref-zeller_potential_2014}{}
4. Zeller G, Tap J, Voigt AY, Sunagawa S, Kultima JR, Costea PI, et al.
Potential of fecal microbiota for early-stage detection of colorectal
cancer. Molecular Systems Biology. 2014;10:766.

\hypertarget{ref-dejea_microbiota_2014}{}
5. Dejea CM, Wick EC, Hechenbleikner EM, White JR, Mark Welch JL,
Rossetti BJ, et al. Microbiota organization is a distinct feature of
proximal colorectal cancers. Proceedings of the National Academy of
Sciences of the United States of America. 2014;111:18321--6.

\hypertarget{ref-zackular_manipulation_2016}{}
6. Zackular JP, Baxter NT, Chen GY, Schloss PD. Manipulation of the Gut
Microbiota Reveals Role in Colon Tumorigenesis. mSphere. 2016;1.

\hypertarget{ref-arthur_microbial_2014}{}
7. Arthur JC, Gharaibeh RZ, Mühlbauer M, Perez-Chanona E, Uronis JM,
McCafferty J, et al. Microbial genomic analysis reveals the essential
role of inflammation in bacteria-induced colorectal cancer. Nature
Communications. 2014;5:4724.

\hypertarget{ref-flynn_metabolic_2016}{}
8. Flynn KJ, Baxter NT, Schloss PD. Metabolic and Community Synergy of
Oral Bacteria in Colorectal Cancer. mSphere. 2016;1.

\hypertarget{ref-arthur_intestinal_2012}{}
9. Arthur JC, Perez-Chanona E, Mühlbauer M, Tomkovich S, Uronis JM, Fan
T-J, et al. Intestinal inflammation targets cancer-inducing activity of
the microbiota. Science (New York, N.Y.). 2012;338:120--3.

\hypertarget{ref-louis_gut_2014}{}
10. Louis P, Hold GL, Flint HJ. The gut microbiota, bacterial
metabolites and colorectal cancer. Nature Reviews Microbiology
{[}Internet{]}. 2014 {[}cited 2017 Feb 14{]};12:661--72. Available from:
\url{http://www.nature.com/doifinder/10.1038/nrmicro3344}

\hypertarget{ref-zhu_intake_2016}{}
11. Zhu Y, Lin X, Li H, Li Y, Shi X, Zhao F, et al. Intake of Meat
Proteins Substantially Increased the Relative Abundance of Genus
Lactobacillus in Rat Feces. PloS One. 2016;11:e0152678.

\hypertarget{ref-mu_colonic_2016}{}
12. Mu C, Yang Y, Luo Z, Guan L, Zhu W. The Colonic Microbiome and
Epithelial Transcriptome Are Altered in Rats Fed a High-Protein Diet
Compared with a Normal-Protein Diet. The Journal of Nutrition.
2016;146:474--83.

\hypertarget{ref-ozdal_reciprocal_2016}{}
13. Ozdal T, Sela DA, Xiao J, Boyacioglu D, Chen F, Capanoglu E. The
Reciprocal Interactions between Polyphenols and Gut Microbiota and
Effects on Bioaccessibility. Nutrients {[}Internet{]}. 2016 {[}cited
2017 Feb 14{]};8:78. Available from:
\url{http://www.mdpi.com/2072-6643/8/2/78}

\hypertarget{ref-cotter_long-term_2016}{}
14. Cotter TG, Burger KN, Devens ME, Simonson JA, Lowrie KL, Heigh RI,
et al. Long-Term Follow-up of Patients Having False Positive
Multi-target Stool DNA Tests after Negative Screening Colonoscopy: The
LONG-HAUL Cohort Study. Cancer Epidemiology, Biomarkers \& Prevention: A
Publication of the American Association for Cancer Research, Cosponsored
by the American Society of Preventive Oncology. 2016;

\hypertarget{ref-obrien_impact_2013}{}
15. O'Brien CL, Allison GE, Grimpen F, Pavli P. Impact of colonoscopy
bowel preparation on intestinal microbiota. PloS One. 2013;8:e62815.

\hypertarget{ref-daly_evaluation_2013}{}
16. Daly JM, Bay CP, Levy BT. Evaluation of fecal immunochemical tests
for colorectal cancer screening. Journal of Primary Care \& Community
Health. 2013;4:245--50.

\hypertarget{ref-kozich_development_2013}{}
17. Kozich JJ, Westcott SL, Baxter NT, Highlander SK, Schloss PD.
Development of a dual-index sequencing strategy and curation pipeline
for analyzing amplicon sequence data on the MiSeq Illumina sequencing
platform. Applied and Environmental Microbiology. 2013;79:5112--20.

\hypertarget{ref-breiman_random_2001}{}
18. Breiman L. Random Forests. Machine Learning {[}Internet{]}. 2001
{[}cited 2013 Feb 7{]};45:5--32. Available from:
\href{http://link.springer.com/article/10.1023/A\%3A1010933404324\%20http://link.springer.com/article/10.1023\%2FA\%3A1010933404324?LI=true}{http://link.springer.com/article/10.1023/A\%3A1010933404324 http://link.springer.com/article/10.1023\%2FA\%3A1010933404324?LI=true}

\hypertarget{ref-anderson_permanova_2013}{}
19. Anderson MJ, Walsh DCI. PERMANOVA, ANOSIM, and the Mantel test in
the face of heterogeneous dispersions: What null hypothesis are you
testing? Ecological Monographs {[}Internet{]}. 2013 {[}cited 2017 Jan
5{]};83:557--74. Available from:
\url{http://doi.wiley.com/10.1890/12-2010.1}

\hypertarget{ref-benjamini_controlling_1995}{}
20. Benjamini Y, Hochberg Y. Controlling the false discovery rate: A
practical and powerful approach to multiple testing. Journal of the
Royal Statistical Society. Series B (Methodological). 1995;57:289--300.


\end{document}
