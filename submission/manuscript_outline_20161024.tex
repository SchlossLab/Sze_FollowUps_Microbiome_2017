\documentclass[12pt,]{article}
\usepackage{lmodern}
\usepackage{amssymb,amsmath}
\usepackage{ifxetex,ifluatex}
\usepackage{fixltx2e} % provides \textsubscript
\ifnum 0\ifxetex 1\fi\ifluatex 1\fi=0 % if pdftex
  \usepackage[T1]{fontenc}
  \usepackage[utf8]{inputenc}
\else % if luatex or xelatex
  \ifxetex
    \usepackage{mathspec}
  \else
    \usepackage{fontspec}
  \fi
  \defaultfontfeatures{Ligatures=TeX,Scale=MatchLowercase}
\fi
% use upquote if available, for straight quotes in verbatim environments
\IfFileExists{upquote.sty}{\usepackage{upquote}}{}
% use microtype if available
\IfFileExists{microtype.sty}{%
\usepackage{microtype}
\UseMicrotypeSet[protrusion]{basicmath} % disable protrusion for tt fonts
}{}
\usepackage[margin=1.0in]{geometry}
\usepackage{hyperref}
\hypersetup{unicode=true,
            pdfborder={0 0 0},
            breaklinks=true}
\urlstyle{same}  % don't use monospace font for urls
\usepackage{graphicx,grffile}
\makeatletter
\def\maxwidth{\ifdim\Gin@nat@width>\linewidth\linewidth\else\Gin@nat@width\fi}
\def\maxheight{\ifdim\Gin@nat@height>\textheight\textheight\else\Gin@nat@height\fi}
\makeatother
% Scale images if necessary, so that they will not overflow the page
% margins by default, and it is still possible to overwrite the defaults
% using explicit options in \includegraphics[width, height, ...]{}
\setkeys{Gin}{width=\maxwidth,height=\maxheight,keepaspectratio}
\IfFileExists{parskip.sty}{%
\usepackage{parskip}
}{% else
\setlength{\parindent}{0pt}
\setlength{\parskip}{6pt plus 2pt minus 1pt}
}
\setlength{\emergencystretch}{3em}  % prevent overfull lines
\providecommand{\tightlist}{%
  \setlength{\itemsep}{0pt}\setlength{\parskip}{0pt}}
\setcounter{secnumdepth}{0}
% Redefines (sub)paragraphs to behave more like sections
\ifx\paragraph\undefined\else
\let\oldparagraph\paragraph
\renewcommand{\paragraph}[1]{\oldparagraph{#1}\mbox{}}
\fi
\ifx\subparagraph\undefined\else
\let\oldsubparagraph\subparagraph
\renewcommand{\subparagraph}[1]{\oldsubparagraph{#1}\mbox{}}
\fi

%%% Use protect on footnotes to avoid problems with footnotes in titles
\let\rmarkdownfootnote\footnote%
\def\footnote{\protect\rmarkdownfootnote}

%%% Change title format to be more compact
\usepackage{titling}

% Create subtitle command for use in maketitle
\newcommand{\subtitle}[1]{
  \posttitle{
    \begin{center}\large#1\end{center}
    }
}

\setlength{\droptitle}{-2em}
  \title{}
  \pretitle{\vspace{\droptitle}}
  \posttitle{}
  \author{}
  \preauthor{}\postauthor{}
  \date{}
  \predate{}\postdate{}

\usepackage{helvet} % Helvetica font
\renewcommand*\familydefault{\sfdefault} % Use the sans serif version of the font
\usepackage[T1]{fontenc}

\usepackage[none]{hyphenat}

\usepackage{setspace}
\doublespacing
\setlength{\parskip}{1em}

\usepackage{lineno}

\usepackage{pdfpages}
\usepackage{comment}

\begin{document}

\section{Differences in the Stool Microbiome Before and After Colorectal
Cancer
Treatment}\label{differences-in-the-stool-microbiome-before-and-after-colorectal-cancer-treatment}

\vspace{25mm}

\begin{center}
Running Title: Human Microbiome and Colorectal Cancer


\vspace{10mm}

Marc A Sze, Nielson T Baxter, Mack T Ruffin IV, Mary AM Rogers, and Patrick D Schloss${^\dagger}$

\vspace{10mm}

Contributions: All authors were involved in the conception and design of the study. MAS analyzed the data. NTB processed samples and analyzed the data. All authors interpreted the data. MAS and PDS wrote the manuscript. All authors reviewed and revised the manuscript. All authors read and approved the final manuscript.

\vspace{20mm}

$\dagger$ To whom correspondence should be addressed: pschloss@umich.edu

Department of Microbiology and Immunology, University of Michigan, Ann Arbor, MI
\end{center}

\newpage

\linenumbers

\subsection{Abstract}\label{abstract}

Colorectal cancer (CRC) continues to be a worldwide health problem with
early detection being used as a key component in mitigating deaths due
to the disease. Previous research suggests a link between stool
bacterial microbiome and CRC. In this study, we used a model based on
the microbiome, demographics, and prior medical history to classify
individuals as having a lesion (i.e.~adenoma or carcinoma). We then used
this model to characterize the change in the gut microbiota before and
after surgery. The overall objective was to investigate the changes in
the microbiome after surgery in patients with adenomas or carcinomas.
This model was tested on a 66 person group that included samples before
and after treatment to allow for the assessment of how the model adjusts
risk after treatment. The model used for prediction had an AUC of 0.763.
For the follow up samples our Random Forest model significantly
decreased the positive probability of lesion compared to the initial
samples for both adenoma (P-value = 3.64e-11) and carcinoma (P-value =
7.95e-08). Our model predicted that 36.4\% of the 67-person cohort had
normal colons and 63.6\% had a lesion. Some OTUs that changed the most
before and after treatment included OTUs that were affiliated with
members of Blautia, Clostridium\_XlVa and Escherichia/Shigella. Our
model suggests that treatment does significantly reduce the probability
of having a colonic lesion. Further surveillance of these individuals
will enable us to determine whether models such as the one we present
here can also be used to predict recurrence of colorectal cancer.

\newpage

\subsubsection{Importance}\label{importance}

This is one of the first studies to investigate within humans what
happens to the bacterial microbiome before and after adenoma and
carcinoma treatment. Specifically, it aims to assess how a random forest
machine learning algorithm built model respond to treatment and adjust
it's positive probability calls of whether the individual has an adenoma
or carcinoma due to surgical removal of the lesion.

\newpage

\subsection{Introduction}\label{introduction}

Colorectal cancer (CRC) continues to be a leading cause of cancer
related deaths and is the second most common cancer death among men aged
40-79 years of age (1, 2). Over the last few years death due to the
disease has seen a significant decrease thanks mainly to improvements in
screening (1). However, despite this giant improvement there are still
approximately 50,000 deaths from the disease a year (2). It is estimated
that around 5-10\% of all CRCs can be explained by autosomal dominant
inheritance (3). The vast majority of CRCs are not inherited and the
exact etiology to disease has not been well worked out (2). Although
many risk factors have been identified ({\textbf{???}}) and non-invasive
screening techniques have started to be put into consistent use (4, 5)
there has been a consistent increase in the incidence of CRC in the
younger population.

This increased incidence of CRC in the younger population is concerncing
since having either an adenoma or carcinoma increases ones risk for
future adenomas or carcinomas (6--8). This increased risk can also carry
with it an increased risk of mortality due to this recurrence (9, 10).
Therefore there has been a great amount of interest in early risk
stratification tools (11, 12) that can help identify those they may be
at risk of reccurence. Concurrently with this there has been a lot of
interest in new areas such as the gut bacterial microbiome for insight
into potential disease pathology.

There has been promising work on the bacterial microbiome and it's
ability to be able to complement existing screening methods such as
Fecal Immunoglobulin Test (FIT) or act alone as a screening tool (13,
14). There has also been research into how this microbiome could be
altered directly on tumor tissue itself (15). There have also been a few
studies that have shown how this microbiome (16) or specific members
within it (17) could be directly involved ith the pathogenesis of CRC.
These studies have helped to provide a tantilzing link between the
bacterial microbiome and CRC. However, at this present time there
remains limited information on the bacterial microbiome before and after
successful surgery for removal of the adenoma or carcinoma.

In this study we investigated what happened to the bacterial microbiome
before and after surgery for both adenoma and carcinoma individuals. Our
anlaysis includes both alpha and beta diversity analysis along with
investigating indvidual operational taxonomic units (OTUs). We next
utilized a Random Forest model that was trained on a completely separate
data set and observed how this model as well as specific OTUs within
this model performed on the initial and follow up samples. We then
created a new model to classify intial and follow up samples and
investigated how the model performs and what specific OUTs within the
model change. Finally, we observed how specific bacteria that have been
previously implicated in CRC changed before and after treatment in both
the adenoma and carcinoma individuals.

\newpage

\subsection{Results}\label{results}

\textbf{\emph{Bacterial Community and Fit Changes before and after
Treatment}} Based on thetayc distance metrics, comparing the initial to
the follow up samples, there was no difference between the adenoma and
carcinoma groups (P-value = 0.697) {[}Figure 1a{]}. There was a
difference in FIT between initial and follow up samples with the
carcinoma group having a significant decrease in FIT versus the adenoma
group (P-value = 2.15e-05) {[}Figure 1b{]}. Although the thetayc
distance metric change was similar between adenoma and carcinoma the
directionality of the change was significant in the cancer group between
initial and follow up (P-value = 0.002) but not for the adenoma group
(P-value = 0.997). This change can be visualized using an NMDS {[}Figure
2{]}. When all follow up samples were compared to each other there was
no significant overall difference between them (P-value = 0.085). There
was no significant difference between initial and follow up samples for
observed OTUs, Shannon diversity, or evenness after correction for
multiple comparisons {[}Table S1{]}. Time of follow up sample from
initial sampling, did not have a significant difference between adenoma
and carcinoma (uncorrected P-value = 0.784).

\textbf{\emph{Outcome of Model Training}} The range of the AUC for model
training ranged from a minimum of 0.723 to a maximum of 0.795 with the
middle of all 100 runs having an AUC of 0.761. Interestingly, the worst
AUC model from training performed the best on it's respective 80/20
split test data {[}Figure 3{]}. In fact the 80/20 test performance
showed that the AUC for the middle model chosen was the most stable
(best training model test set AUC = 0.646, middle training model test
set AUC = 0.744, worse training model test set AUC = 0.904). That is to
say it had the smallest change in AUC in comparison to the minimum and
maximum AUC trained models. The middle model was close to the full
training data AUC 0.763. There was no significant difference between the
AUC of the best and middle training models (P-value = 1). There was also
no difference in the middle model versus worse (P-value = 0.0431) or
full data model (P-value = 1). The two comparisons with a significant
difference were between the worse training model and the best training
model (P-value = 6.83e-04) and the worse training model and full
training model (P-value = 1.2e-03).

\textbf{\emph{Most Important Variables to the Model}} Overall, there
were a total of 37 variables identified as being present in more than
50\% of the training models {[}Table S2{]}. The top 5 most important
bacterial OTUs were Bacteria (Otu000013), Escherichia/Shigella
(Otu000018), Bacteria (Otu000020), Ruminococcus (Otu000017), and
Porphyromonas (Otu000153). These 5 OTUs were present in at least 90 out
of the total 100 different 80/20 runs.

\textbf{\emph{Surgical Removal of an Adenoma or Carcinoma Results in a
Decrease in Positive Probability Prediction}} A total of 1 sample was
omitted from the original 67 test sample set since it was missing a
complete set of follow up data. This left a total of 66 samples for test
predictions. After multiple comparison correction there was a
significant overall decrease in positive probability of a carcinoma and
adenoma (P-value = 1.11e-11) {[}Figure 4{]}. This decrease was
significant for both adenoma (P-value = 3.64e-11 {[}Figure 4a{]} and
carcinoma (P-value = 7.95e-08) {[}Figure 4b{]} alone. This decrease in
probability of lesion also held specifically for those with screen
relative neoplasias (SRN) (P-value = 7.63e-06). A total of 66 or 100\%
of all samples were correctly predicted to have a lesion. Although there
was a decrease in positive probability only 24 of the total 66
individuals were classified as adenoma or carcinoma free on follow up
(successful classification of 37.9\%). There was no significant
difference between the predictions and actual diagnosis for the initial
samples in the 67-sample cohort test set. However, the predictions were
significantly discordent with the diagnosis for the follow up samples
(P-value = 4.19e-10). Although there were discordent results the
respective sensitivity for the initial group was 100\% and for follow up
was 100\%, respectively.

There was 1 individual who still clearly had CRC on follow up as well as
5 individuals whose status on follow up was unknown. Although the 1
individual had a decrease in positive probability their follow up sample
was still higher than the cutoff threshold of 0.5 (positive probability
= 0.903). Interestingly, 1 individuals who were unknown on follow up
still were over the threshold cutoff of 0.5 even though, like the 1
individual with clear CRC on follow up, the probability of an adenoma or
carcinoma decreased {[}Table S3{]}.

The follow up positive probabilities were not affected by either
chemotherapy treatment (uncorrected P-value = 0.621) or radiation
therapy (uncorrected P-value = 0.255). There was also no difference in
the amount of change in the positive probability based on whether
individuals received chemotherapy (unccorected P-value = 0.718) or
radiation therapy (uncorrected P-value = 0.431).

\textbf{\emph{Specific OTUs in the Lesion Model are not Detected in
Follow Up Versus Initial Samples}} Overall, there were a total of 8 OTUs
that were common between the main lesion model and the model for
classifying initial and follow up samples specifically {[}Table S4{]}. A
total of 1 OTU was still significant after multiple comparison
correction and it's lowest taxonomic identification was to Blautia. In
general, Otu000012 (Blautia) was decreased from initial to follow up
{[}Figure 5{]}. The relative abundance was not drastically different
then the mean of the values observed in the control training set
{[}Figure 5{]}.

\textbf{\emph{Differences in Adenoma and Carcinoma in Previously
Associated Cancer Bacteria}} First, there was a clear magnitude
difference in these specific OTUs based on whether they were from
adenoma or carcinoma individuals {[}Figure 6{]}. The carcinoma samples
showed a significant difference between initial and follow up samples
for Peptostreptococcus stomatis (P-value = 0.0183) and Porphyromonas
asaccharolytica (P-value = 0.0154) whereas there were no signficant
differences in any of these OTUs in the adenoma samples {[}Table S5{]}.

\newpage

\subsection{Discussion}\label{discussion}

In our training set we show that the overall community structure as
measured by different alpha diversity metrics, shows very little change
between controls and those with either adenoma or carcinoma {[}Table
S1{]}. With respect to our test set there was very little difference in
magnitude of change in the thetayc distance metric between those with
adenoma or carcinoma {[}Figure 1a{]}. In contrast, FIT had a large
change in the initial and follow up samples in the carcinoma group
versus the adenoma {[}Figure 1b{]}. An NMDS showed that there was very
little observable change between initial and follow up for the adenoma
group but there was one for the carcinoma group {[}Figure 2{]}. This
cursory information is suggestive that treatment of carcinoma, had the
largest response.

We next created a model that incorporated both patient metadata, FIT,
and the bacterial microbiome to be able to predict lesions (adenoma or
carcinoma). Our middle training model, based on AUC, from 100 80/20
(train/test) splits was similar to the full training data model. It's
10-fold cross validated AUC was similar to it's test set AUC which was
not the case for both the best and worse training model {[}Figure 3{]}.
Using the full training data model we predicted the probability of a
lesion in the inital and follow up samples {[}Figure 4{]}. There was a
signficant decrease in positive probability regardless of whether the
sample was a carcinoma or adenoma. The overall sensitivity for lesion
detection in the intial samples was 100 and for follow ups was 100.
Although there was a decrease in overall probability of an adenoma or
carcinoma only 24 were below the 0.5 threshold out of the total 65
individuals who were diagnosed as not having a carcinoma on follow up.

We then investigated which OTUs could potentially be more important in
our model {[}Figure 5 \& Table S4{]}. Many of the OTUs identified
classified to normal flora bacterium {[}Table S4{]}. Only a single OTU
though was significant after multiple comparison correction and the
lowest taxonomic identification of Otu000012 was to Blautia. Although
there was a difference in the relative abundance at initial and follow
up these values were not drastically different from the relative
abundance values observed in the control individuals of the training set
{[}Figure 5{]}. Although we were interested in what we could use to
classify those with either adenoma or carcinoma versus normal. We found
that the traditional bacteria associated with CRC were higher in
magnitude in the carcinoma group and there were significant differences
in some of these OTUs between the initial and follow up samples
{[}Figure 5 and Tabl S5{]}. This research provides evidence that it is
possible to use bacterial microbiome data to create a highily sensitive
model, that is reactive to therapy, for detection of adenoma or
carcinoma. It accomplishes this by using a unique sample set in which
before and after surgery stool samples are available for assessment. By
using these types of samples we are not only able to show sensitivity of
lesion prediction but also able to show that this model is reactive.
That is to say that after surgery for removal of the adenoma or
carcinoma it decreases the positive probability to reflect a lower
likelihood of the individual having an adenoma or carcinoma.

This study builds upon previous work from numerous labs that have looked
into the bacterial microbiome as a potential screening tool
(\textbf{insert citation}). Based on previous work by Jobin, et al.
(\textbf{insert citation}) it may not be surprising to see E.coli in the
top 5 OTUs for this model. Similarily, Porphyromonas has also been
implicated in colorectal cancer (\textbf{insert citation}).
Interestingly, many of the other OTUs had taxonomic identification for
resident gut microbes. This could suggest that changes to the resident
microbiome are important to the initiation of adenoma or carcinoma
formation (\textbf{insert citation}) and provide support for the
hypothesis that an initial change in the bacterial microbiome could pave
the way for more inflammatory species: whether by creation of a new
niche for oral microbes (\textbf{insert citation}) or allowing for a
bloom of existing pro-inflammatory residents (\textbf{insert citation}).

Naturally, it is curious that normal staples of many screening studies
such as Fusobacterium, Parvimonas, and Peptostreptococcus were not
present in the majority of the training models. One potential
explanation for this is that FIT provides the same information to the
model as these three organisms and so the model uses FIT preferentially
over them. This has been suggested to be the case in a previous study
(\textbf{insert Baxter Study}). It is also possible that these specific
bacteria play a major role in the progression to carcinoma but may not
be as important in the initiation of an adenoma, which would be
supported by our data {[}Figure 5{]}. Regardless, our study does not
argue against the importance of these bacterium in CRC initiation or
pathogenesis but rather that the model does not utilize these specific
bacteria for prediction purposes. Another potential reason why we did
not identify the ``usual suspects''" is that these bacteria may not
change much between initial and follow up samples in those with an
identified lesion. That is to say that the bacteria are consistently
present even after removal of the lesion by surgery. Finally, it is
likely that within our test set there was not enough indviduals in which
detection was made or relative abundance high enough for these bacteria
to be significant using a paired wilcoxson test.

One limitation in this study is that we do not know whether individuals
in our test set eventually had a subsequent CRC diagnosis. This
information would help to strengthen the case for our Random Forest
based model keeping a number of individuals above the cutoff threshold
even though at follow up they were diagnosed as no longer having a
lesion. Another limitation is that we do not know if adding modern tests
such as the stool DNA test (\textbf{insert citation}) could help improve
our overall AUC. Another limitation is that this study drew heavily from
those with caucasian ancestry. The results may not be immediately
representative of those with either Asian or African ancestry. Finally,
although our training and test set are relatively large we still run the
risk of overfitting or having a model that may not be immediately
extrapolateable to other populations. We've done our best to safeguard
against this by not only running 10-fold cross validation but also
having over 100 different 80/20 splits to try and mimic the type of
variation that might be expected to occur.

By adding patient data such as age, BMI, etc. to the model and showing
that it can successfully help to predict both carcinoma and adenoma our
study provides further data that these patient factors in conjunction
with the bacterial microbiome could potentially influence CRC and
perhaps have a role in formation of adenomas. Further studies need to be
carried out to verify our findings since not only are we dealing with
stool, which could be very different than the communities present on the
actual tissue, but also are dealing with correlations that may not be
representative of the true pathogensis of disease.

Despite these limitations we think that these findings significantly add
to the existing scientific knowledge on CRC and the bacterial
microbiome. The ability for machine learning algorithms to take
bacterial microbiome data and successfully lower positive probability
after either adenoma or carcinoma removal provides evidence that there
are specific signatures associated with these lesions. It also shows
that these algorithms can not only successfully react to successful
treatment regimens but also may be able to one day diagnose CRC with a
high level of accuracy.

\newpage

\subsection{Methods}\label{methods}

\textbf{\emph{Study Design and Patient Sampling}} The sampling and
design of the study was similar to that reported in Baxter, et al
(\textbf{insert citation}). In brief, study exclusion involved those who
had already undergone surgery, radiation, or chemotherapy, had
colorectal cancer before a baseline stool sample could be obtained, had
IBD, a known hereditary non-polyposis colorectal cancer, or Familial
adenomatous polyposis. Samples used to build the model used for
prediction were collected either prior to a colonoscopy or between 1 - 2
weeks after. The bacterial microbiome has been shown to nomralize within
this time period (\textbf{insert citation}). Kept apart from this
training set were a total of 67 individuals that not only had a sample
as described previoulsy but also a follow up sample between 188 - 546
days after surgery and treatment had been completed. This study was
approved by the University of Michigan Institutional Review Board. All
study participants provided informed consent and the study itself
conformed to the guidelines set out by the Helsinki Declaration.

\textbf{\emph{Fecal Immunochemical Test and 16S rRNA Gene Sequencing}}
FIT was analyzed as previously published using both OC FIT-CHEK and
OC-Auto Micro 80 automated system (Polymedco Inc.) (\textbf{insert
citation}). 16S rRNA gene sequencing was completed as previously
described by Kozich, et al. (\textbf{insert citation}). In brief, DNA
extraction used the 96 well Soil DNA isolation kit (MO BIO Laboratories)
and an epMotion 5075 automated pipetting system (Eppendorf). The V4
variable region was amplified and the resulting product was split
between three sequencing runs with control, adenoma, and carcinoma
evenly represented on each run. Each group was randomly assigned to
avoid biases based on sample collection location.

\textbf{\emph{Sequence Processing}} The mothur software package
(v1.37.5) was used to process the 16S rRNA gene sequences. This process
has been previously described (\textbf{insert citations}). The general
processing workflow using mothur is as follows: Paired-end reads were
first merged into contigs, quality filtered, aligned to the SILVA
database, screening for chimeras, classified with a naive Bayesian
classifier using the Ribosomal Database Project (RDP), and clustered
into Operational Taxonomic Units (OTUs) using a 97\% similarity cutoff
with an average neighbor clustering algorithm. The numer of sequences
for each sample was rarified to 10521 in an attempt to minimize uneven
sampling.

\textbf{\emph{Model Creation}} The Random Forest (\textbf{insert
citation}) algorithm was used to create the model used for prediction of
lesion (adenoma or carcinoma) for the 67 individuals with follow up
samples. The model included data on FIT, the bacterial microbiome, sex,
age, Body Mass Index (BMI), whether the indivdiual was caucasion or not,
history of cancer, and family history of cancer. Non-binary data was
checked for near zero variance and auto correlation. Data columns that
had near zero variance were removed. Columns that were correlated with
each other over a Spearman correlation coefficient of 0.75 had one of
the two columns removed. This pre-processing was performed with the R
package caret (v6.0.73). Optimiation of the mtry hyperparameter as well
as data on the best, middle, and worse performance of the model involved
taking the 490 samples and making 100 80/20 (train/test) splits in the
data where control and lesion were equally represented in the 80 and 20
split, respectively. This 80\% portion was then split again into an
80/20 split, and run through 20 repeated 10-fold cross validations to
optimzie the model's AUC (Area Under the Curve of the Receiver Operator
Characteristic). This resulting model was then tested on the 20\% of the
data that was originally held out from this overall process. Once the
ideal mtry was found the entire 490 sample set was used to create the
final Random Forest model on which testing on the 67-person cohort was
completed. The default cutoff of 0.5 was used as the threshold to
classify individuals as positive or negative for lesion.

\textbf{\emph{Selection of Important OTUs}} In order to assess which
variables were most central to all the models we counted the number of
times a variable was present in the top 10\% of mean decrease in
accuracy (MDA) for each different 80/20 split model and then filtered
this list to variables that were only present more than 50\% of the
time. This final collated list of variables was what was considered the
most important for the lesion prediction model.

A second model using the same types of conditions was used to identify
OTUs that could predict initial and follow up samples. These OTUs were
considered the most important for predicting initial and follow up
samples. This new model important OTUs were then compared to the overall
lesion model important OTUs and the resulting common OTUs to both models
were selected to test for changes between initial and follow up using a
paired wilcoxson rank sum test.

\textbf{\emph{Investigation of Important CRC Bacteria}} In order to
investigate how specific bacteria that have been previously associated
with CRC with selected OTUs that taxonomically classified to
Fusobacterium Nucleatum, Parvimonas Micra, Peptostreptococcus
Assacharolytica, and Porphyromonas Stomatis. Specifically we wanted to
observe if there were any differences based on whether the individual
had an adenoma or carcinoma that may have been missed with our lesion
versus normal model.

\textbf{\emph{Statistical Analysis}} The R software package (v3.3.0) was
used for all statisitical analysis. Comparisons between bacterial
community structure utilized PERMANOVA (\textbf{insert citation}) in the
vegan package (v2.4.1) while comparisons between ROC curves utilized the
method by DeLong et al. (\textbf{insert citation}) executed by the pROC
(v1.8) package. Comparisons between probabilities as well as overall
amount of OTU between initial and follow up samples utilized a paired
wilcoxson ranked sum test. Where multiple comparison testing was needed
a Benjamini-Hochberg (BH) correction was applied (\textbf{insert
citation}) and a corrected P-value of less than 0.05 was considered
significant. Unless otherwise stated the P-values reported are those of
the BH corrected ones.

\textbf{\emph{Reproducible methods.}} A detailed and reproducible
description of how the data were processed and analyzed can be found at
\url{https://github.com/SchlossLab/Baxter_followUps_2016}.

\newpage

\subsection{Acknowledgements}\label{acknowledgements}

The authors thank the Great Lakes-New England Early Detection Research
Network for providing the fecal samples that were used in this study.
This study was supported by funding from the National Institutes of
Health to P. Schloss (R01GM099514, P30DK034933) and to the Early
Detection Research Network (U01CA86400).

\newpage

\textbf{Figure 1: Change in Thetayc and Fit between initial and follow
up in adenoma or carcinoma group.} A) No significant difference was
found between the adenoma and carcinoma group for thetayc (P-value =
0.697). B) A significant difference was found between the adenoma and
carcinoma group for FIT (P-value = 2.15e-05).

\textbf{Figure 2: NMDS of the Overall Bacterial Community Changes.} A)
NMDS of the intial and follow up samples for the Adenoma group. B) NMDS
of the initial and follow up samples for the Carcinoma group.

\textbf{Figure 3: Graph of the Receiver Operating Characteristic Curve
on Test Set Performance of the Best, Middle, Worse, and Full Training
Models.} For each of the 100 training cohort sets used had 392
individuals and the testing cohort sets had 13 individuals. The AUC on
the test sets for the best, middle, and worse models from training were
0.646, 0.744, and 0.904, respectively. cvAUC is the 20 times repeated
10-fold cross-validated AUC from training.

\textbf{Figure 4: Breakdown by Carcinoma and Adenoma of Prediction
Results for Initial and Follow Up}* A) Positive probability adjustment
of those with carcinoma from intial to follow up sample B) Positive
probability adjustment of those with adenoma as well as those with SRN
and the probability adjustment from initial to follow up sample. The
dotted line represents the threshold used to make the decision of
whether a sample was lesion positive or not.

\textbf{Figure 5: Lesion Model OTU with a Significant Decrease in
Relative Abundance that is also Predictive of Initial and Follow Up.}
After multiple comparison correction 1 (Blautia) was the only one with a
P-value \textless{} 0.05. The dotted line represents the average
relative abundance in the control training group.

\textbf{Figure 6: Previously Associated CRC Bacteria in Initial and
Follow up Samples.} A) Carcinoma initial and follow up samples. There
was a significant difference in initial and follow up sample for the
OTUs classfied as Peptostreptococcus stomatis (P-value = 0.0183) and
Porphyromonas asaccharolytica (P-value = 0.0154). B) Adenoma initial and
follow up samples. There were no significant differences between initial
and follow up.

\newpage

\textbf{Figure S1: Thetayc Graphed Against Time of Follow up Sample from
Initial}

\newpage

\subsection*{References}\label{references}
\addcontentsline{toc}{subsection}{References}

\hypertarget{refs}{}
\hypertarget{ref-jemal_cancer_2010}{}
1. \textbf{Jemal A}, \textbf{Siegel R}, \textbf{Xu J}, \textbf{Ward E}.
2010. Cancer statistics, 2010. CA: a cancer journal for clinicians
\textbf{60}:277--300.
doi:\href{https://doi.org/10.3322/caac.20073}{10.3322/caac.20073}.

\hypertarget{ref-haggar_colorectal_2009}{}
2. \textbf{Haggar FA}, \textbf{Boushey RP}. 2009. Colorectal cancer
epidemiology: Incidence, mortality, survival, and risk factors. Clinics
in Colon and Rectal Surgery \textbf{22}:191--197.
doi:\href{https://doi.org/10.1055/s-0029-1242458}{10.1055/s-0029-1242458}.

\hypertarget{ref-green_very_2007}{}
3. \textbf{Green RC}, \textbf{Green JS}, \textbf{Buehler SK},
\textbf{Robb JD}, \textbf{Daftary D}, \textbf{Gallinger S},
\textbf{McLaughlin JR}, \textbf{Parfrey PS}, \textbf{Younghusband HB}.
2007. Very high incidence of familial colorectal cancer in Newfoundland:
A comparison with Ontario and 13 other population-based studies.
Familial Cancer \textbf{6}:53--62.
doi:\href{https://doi.org/10.1007/s10689-006-9104-x}{10.1007/s10689-006-9104-x}.

\hypertarget{ref-liao_application_2013}{}
4. \textbf{Liao C-S}, \textbf{Lin Y-M}, \textbf{Chang H-C}, \textbf{Chen
Y-H}, \textbf{Chong L-W}, \textbf{Chen C-H}, \textbf{Lin Y-S},
\textbf{Yang K-C}, \textbf{Shih C-H}. 2013. Application of quantitative
estimates of fecal hemoglobin concentration for risk prediction of
colorectal neoplasia. World Journal of Gastroenterology
\textbf{19}:8366--8372.
doi:\href{https://doi.org/10.3748/wjg.v19.i45.8366}{10.3748/wjg.v19.i45.8366}.

\hypertarget{ref-johnson_multi-target_2016}{}
5. \textbf{Johnson DH}, \textbf{Kisiel JB}, \textbf{Burger KN},
\textbf{Mahoney DW}, \textbf{Devens ME}, \textbf{Ahlquist DA},
\textbf{Sweetser S}. 2016. Multi-target stool DNA test: Clinical
performance and impact on yield and quality of colonoscopy for
colorectal cancer screening. Gastrointestinal Endoscopy.
doi:\href{https://doi.org/10.1016/j.gie.2016.11.012}{10.1016/j.gie.2016.11.012}.

\hypertarget{ref-laiyemo_short-_2013}{}
6. \textbf{Laiyemo AO}, \textbf{Doubeni C}, \textbf{Brim H},
\textbf{Ashktorab H}, \textbf{Schoen RE}, \textbf{Gupta S},
\textbf{Charabaty A}, \textbf{Lanza E}, \textbf{Smoot DT}, \textbf{Platz
E}, \textbf{Cross AJ}. 2013. Short- and long-term risk of colorectal
adenoma recurrence among whites and blacks. Gastrointestinal Endoscopy
\textbf{77}:447--454.
doi:\href{https://doi.org/10.1016/j.gie.2012.11.027}{10.1016/j.gie.2012.11.027}.

\hypertarget{ref-matsuda_five-year_2009}{}
7. \textbf{Matsuda T}, \textbf{Fujii T}, \textbf{Sano Y}, \textbf{Kudo
S-e}, \textbf{Oda Y}, \textbf{Igarashi M}, \textbf{Iishi H},
\textbf{Murakami Y}, \textbf{Ishikawa H}, \textbf{Shimoda T},
\textbf{Kaneko K}, \textbf{Yoshida S}. 2009. Five-year incidence of
advanced neoplasia after initial colonoscopy in Japan: A multicenter
retrospective cohort study. Japanese Journal of Clinical Oncology
\textbf{39}:435--442.
doi:\href{https://doi.org/10.1093/jjco/hyp047}{10.1093/jjco/hyp047}.

\hypertarget{ref-ren_long-term_2016}{}
8. \textbf{Ren J}, \textbf{Kirkness CS}, \textbf{Kim M}, \textbf{Asche
CV}, \textbf{Puli S}. 2016. Long-term risk of colorectal cancer by
gender after positive colonoscopy: Population-based cohort study.
Current Medical Research and Opinion \textbf{32}:1367--1374.
doi:\href{https://doi.org/10.1080/03007995.2016.1174840}{10.1080/03007995.2016.1174840}.

\hypertarget{ref-loberg_long-term_2014}{}
9. \textbf{Løberg M}, \textbf{Kalager M}, \textbf{Holme Ø}, \textbf{Hoff
G}, \textbf{Adami H-O}, \textbf{Bretthauer M}. 2014. Long-term
colorectal-cancer mortality after adenoma removal. The New England
Journal of Medicine \textbf{371}:799--807.
doi:\href{https://doi.org/10.1056/NEJMoa1315870}{10.1056/NEJMoa1315870}.

\hypertarget{ref-freeman_natural_2013}{}
10. \textbf{Freeman HJ}. 2013. Natural history and long-term outcomes of
patients treated for early stage colorectal cancer. Canadian Journal of
Gastroenterology = Journal Canadien De Gastroenterologie
\textbf{27}:409--413.

\hypertarget{ref-lee_identification_2016}{}
11. \textbf{Lee JH}, \textbf{Lee JL}, \textbf{Park IJ}, \textbf{Lim
S-B}, \textbf{Yu CS}, \textbf{Kim JC}. 2016. Identification of
Recurrence-Predictive Indicators in Stage I Colorectal Cancer. World
Journal of Surgery.
doi:\href{https://doi.org/10.1007/s00268-016-3833-2}{10.1007/s00268-016-3833-2}.

\hypertarget{ref-richards_evidence-based_2016}{}
12. \textbf{Richards CH}, \textbf{Ventham NT}, \textbf{Mansouri D},
\textbf{Wilson M}, \textbf{Ramsay G}, \textbf{Mackay CD},
\textbf{Parnaby CN}, \textbf{Smith D}, \textbf{On J}, \textbf{Speake D},
\textbf{McFarlane G}, \textbf{Neo YN}, \textbf{Aitken E},
\textbf{Forrest C}, \textbf{Knight K}, \textbf{McKay A}, \textbf{Nair
H}, \textbf{Mulholland C}, \textbf{Robertson JH}, \textbf{Carey FA},
\textbf{Steele R}, \textbf{Scottish Surgical Research Group}. 2016. An
evidence-based treatment algorithm for colorectal polyp cancers: Results
from the Scottish Screen-detected Polyp Cancer Study (SSPoCS). Gut.
doi:\href{https://doi.org/10.1136/gutjnl-2016-312201}{10.1136/gutjnl-2016-312201}.

\hypertarget{ref-baxter_microbiota-based_2016}{}
13. \textbf{Baxter NT}, \textbf{Ruffin MT}, \textbf{Rogers MAM},
\textbf{Schloss PD}. 2016. Microbiota-based model improves the
sensitivity of fecal immunochemical test for detecting colonic lesions.
Genome Medicine \textbf{8}:37.
doi:\href{https://doi.org/10.1186/s13073-016-0290-3}{10.1186/s13073-016-0290-3}.

\hypertarget{ref-zeller_potential_2014}{}
14. \textbf{Zeller G}, \textbf{Tap J}, \textbf{Voigt AY},
\textbf{Sunagawa S}, \textbf{Kultima JR}, \textbf{Costea PI},
\textbf{Amiot A}, \textbf{Böhm J}, \textbf{Brunetti F},
\textbf{Habermann N}, \textbf{Hercog R}, \textbf{Koch M},
\textbf{Luciani A}, \textbf{Mende DR}, \textbf{Schneider MA},
\textbf{Schrotz-King P}, \textbf{Tournigand C}, \textbf{Tran Van Nhieu
J}, \textbf{Yamada T}, \textbf{Zimmermann J}, \textbf{Benes V},
\textbf{Kloor M}, \textbf{Ulrich CM}, \textbf{Knebel Doeberitz M von},
\textbf{Sobhani I}, \textbf{Bork P}. 2014. Potential of fecal microbiota
for early-stage detection of colorectal cancer. Molecular Systems
Biology \textbf{10}:766.

\hypertarget{ref-dejea_microbiota_2014}{}
15. \textbf{Dejea CM}, \textbf{Wick EC}, \textbf{Hechenbleikner EM},
\textbf{White JR}, \textbf{Mark Welch JL}, \textbf{Rossetti BJ},
\textbf{Peterson SN}, \textbf{Snesrud EC}, \textbf{Borisy GG},
\textbf{Lazarev M}, \textbf{Stein E}, \textbf{Vadivelu J},
\textbf{Roslani AC}, \textbf{Malik AA}, \textbf{Wanyiri JW}, \textbf{Goh
KL}, \textbf{Thevambiga I}, \textbf{Fu K}, \textbf{Wan F}, \textbf{Llosa
N}, \textbf{Housseau F}, \textbf{Romans K}, \textbf{Wu X},
\textbf{McAllister FM}, \textbf{Wu S}, \textbf{Vogelstein B},
\textbf{Kinzler KW}, \textbf{Pardoll DM}, \textbf{Sears CL}. 2014.
Microbiota organization is a distinct feature of proximal colorectal
cancers. Proceedings of the National Academy of Sciences of the United
States of America \textbf{111}:18321--18326.
doi:\href{https://doi.org/10.1073/pnas.1406199111}{10.1073/pnas.1406199111}.

\hypertarget{ref-zackular_manipulation_2016}{}
16. \textbf{Zackular JP}, \textbf{Baxter NT}, \textbf{Chen GY},
\textbf{Schloss PD}. 2016. Manipulation of the Gut Microbiota Reveals
Role in Colon Tumorigenesis. mSphere \textbf{1}.
doi:\href{https://doi.org/10.1128/mSphere.00001-15}{10.1128/mSphere.00001-15}.

\hypertarget{ref-arthur_microbial_2014}{}
17. \textbf{Arthur JC}, \textbf{Gharaibeh RZ}, \textbf{Mühlbauer M},
\textbf{Perez-Chanona E}, \textbf{Uronis JM}, \textbf{McCafferty J},
\textbf{Fodor AA}, \textbf{Jobin C}. 2014. Microbial genomic analysis
reveals the essential role of inflammation in bacteria-induced
colorectal cancer. Nature Communications \textbf{5}:4724.
doi:\href{https://doi.org/10.1038/ncomms5724}{10.1038/ncomms5724}.


\end{document}
