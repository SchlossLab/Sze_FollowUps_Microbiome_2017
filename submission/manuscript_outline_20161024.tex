\documentclass[12pt,]{article}
\usepackage{lmodern}
\usepackage{amssymb,amsmath}
\usepackage{ifxetex,ifluatex}
\usepackage{fixltx2e} % provides \textsubscript
\ifnum 0\ifxetex 1\fi\ifluatex 1\fi=0 % if pdftex
  \usepackage[T1]{fontenc}
  \usepackage[utf8]{inputenc}
\else % if luatex or xelatex
  \ifxetex
    \usepackage{mathspec}
  \else
    \usepackage{fontspec}
  \fi
  \defaultfontfeatures{Ligatures=TeX,Scale=MatchLowercase}
\fi
% use upquote if available, for straight quotes in verbatim environments
\IfFileExists{upquote.sty}{\usepackage{upquote}}{}
% use microtype if available
\IfFileExists{microtype.sty}{%
\usepackage{microtype}
\UseMicrotypeSet[protrusion]{basicmath} % disable protrusion for tt fonts
}{}
\usepackage[margin=1.0in]{geometry}
\usepackage{hyperref}
\hypersetup{unicode=true,
            pdfborder={0 0 0},
            breaklinks=true}
\urlstyle{same}  % don't use monospace font for urls
\usepackage{graphicx,grffile}
\makeatletter
\def\maxwidth{\ifdim\Gin@nat@width>\linewidth\linewidth\else\Gin@nat@width\fi}
\def\maxheight{\ifdim\Gin@nat@height>\textheight\textheight\else\Gin@nat@height\fi}
\makeatother
% Scale images if necessary, so that they will not overflow the page
% margins by default, and it is still possible to overwrite the defaults
% using explicit options in \includegraphics[width, height, ...]{}
\setkeys{Gin}{width=\maxwidth,height=\maxheight,keepaspectratio}
\IfFileExists{parskip.sty}{%
\usepackage{parskip}
}{% else
\setlength{\parindent}{0pt}
\setlength{\parskip}{6pt plus 2pt minus 1pt}
}
\setlength{\emergencystretch}{3em}  % prevent overfull lines
\providecommand{\tightlist}{%
  \setlength{\itemsep}{0pt}\setlength{\parskip}{0pt}}
\setcounter{secnumdepth}{0}
% Redefines (sub)paragraphs to behave more like sections
\ifx\paragraph\undefined\else
\let\oldparagraph\paragraph
\renewcommand{\paragraph}[1]{\oldparagraph{#1}\mbox{}}
\fi
\ifx\subparagraph\undefined\else
\let\oldsubparagraph\subparagraph
\renewcommand{\subparagraph}[1]{\oldsubparagraph{#1}\mbox{}}
\fi

%%% Use protect on footnotes to avoid problems with footnotes in titles
\let\rmarkdownfootnote\footnote%
\def\footnote{\protect\rmarkdownfootnote}

%%% Change title format to be more compact
\usepackage{titling}

% Create subtitle command for use in maketitle
\newcommand{\subtitle}[1]{
  \posttitle{
    \begin{center}\large#1\end{center}
    }
}

\setlength{\droptitle}{-2em}
  \title{}
  \pretitle{\vspace{\droptitle}}
  \posttitle{}
  \author{}
  \preauthor{}\postauthor{}
  \date{}
  \predate{}\postdate{}

\usepackage{helvet} % Helvetica font
\renewcommand*\familydefault{\sfdefault} % Use the sans serif version of the font
\usepackage[T1]{fontenc}

\usepackage[none]{hyphenat}

\usepackage{setspace}
\doublespacing
\setlength{\parskip}{1em}

\usepackage{lineno}

\usepackage{pdfpages}
\usepackage{comment}

\begin{document}

\section{Differences in the fecal Microbiome Before and After Colorectal
Cancer
Treatment}\label{differences-in-the-fecal-microbiome-before-and-after-colorectal-cancer-treatment}

\vspace{25mm}

\begin{center}
Running Title: Human Microbiome and Colorectal Cancer

\vspace{10mm}

Marc A Sze${^1}$, Nielson T Baxter${^2}$, Mack T Ruffin IV${^3}$, Mary AM Rogers${^2}$, and Patrick D Schloss${^1}$${^\dagger}$

\vspace{20mm}

$\dagger$ To whom correspondence should be addressed: pschloss@umich.edu

$1$ Department of Microbiology and Immunology, University of Michigan, Ann Arbor, MI

$2$ Department of Internal Medicine, University of Michigan, Ann Arbor, MI   

$3$ Department of Family Medicine and Community Medicine, Penn State Hershey Medical Center, Hershey, PA    


\end{center}

\newpage

\linenumbers

\subsection{Abstract}\label{abstract}

\textbf{Background:} Colorectal cancer (CRC) continues to be a worldwide
health problem with early detection being used as a key component in
mitigating deaths due to the disease. Previous research suggests a link
between fecal bacterial microbiome and CRC. The overall objective of our
study was to investigate the changes in the bacterial microbiome after
surgery in patients with lesion (i.e.~adenoma or carcinoma).
Specifically, we wanted to identify what within the community was
different within those undergoing surgical removal of lesion. We also
wanted to investigate the use of the bacterial microbiome and Fecal
Immunoglobulin Test (FIT) to build models which could classify
individuals as having a lesion or as before surgery.\\
\textbf{Results:} Adenoma individual's bacterial microbiome were more
similar to their pre-surgery sample then those with carcinoma (P-value =
0.00198). Their change in FIT was also significantly different then
those with carcinoma (P-value = 2.15e-05). There was no significant
difference in any indivdiual OTU between samples before and after
surgery (P-value \textgreater{} 0.125). A model with a total of 37
variables was able to classify lesion (AUC = 0.847 - 0.791) while the
model to classify samples as before surgery had 33 variables (AUC = 0.79
- 0.651). There was a significant decrease in the model positive
probability for the after surgery sample being either a lesion or before
surgery sample (P-value = 1.91e-11 and 6.72e-12). In total there were 14
OTUs that were common to both models and the majority of these
classified to commensal bacteria (Bacteroides, Blautia, Streptococcus,
and Clostridiales).\\
\textbf{Conclusions:} Our data suggests that treatment not only
significantly reduces the probability of having a colonic lesion within
our models but also causes detectable changes in the bacterial
microbiome. Further surveillance of these individuals will enable us to
determine whether models such as the ones we present could be used to
predict recurrence of colorectal cancer.

\newpage

\subsubsection{Keywords}\label{keywords}

bacterial microbiome; colorectal cancer; polyps; FIT; detection; risk
factors

\newpage

\subsection{Background}\label{background}

Colorectal cancer (CRC) continues to be a leading cause of cancer
related deaths and is the second most common cancer death among men aged
40-79 years of age {[}1,2{]}. Over the last few years death due to the
disease has seen a significant decrease, thanks mainly to improvements
in screening {[}1{]}. However, despite this improvement there are still
approximately 50,000 deaths from the disease a year {[}2{]}. It is
estimated that around 5-10\% of all CRCs can be explained by autosomal
dominant inheritance {[}3{]}. The vast majority of CRCs are not
inherited and the exact etiology of the disease has not been well worked
out {[}2{]}. Although many risk factors have been identified {[}2{]} and
non-invasive screening techniques have started to be put into consistent
use {[}4,5{]} there has been an additional increase in the incidence of
CRC in the younger population.

This increased incidence of CRC in the younger population is concerning
since having either an adenoma or carcinoma increases ones risk for
future adenomas or carcinomas {[}6--8{]}. This increased risk can also
carry with it an increased chance of mortality due to this recurrence
{[}9,10{]}. Therefore, there has been a great amount of interest in
early risk stratification tools {[}11,12{]} that can help identify those
that may be at most susceptibility to reccurence. Concurrently, there
has also been a lot of interest in new areas that could have a role in
disease pathogenesis, such as the gut bacterial microbiome.

There has been promising work on the bacterial microbiome and it's
ability to be able to complement existing screening methods such as
Fecal Immunoglobulin Test (FIT) or act alone as a screening tool
{[}13,14{]}. There has also been research into how this microbiome could
be altered directly on tumor tissue itself {[}15{]}. A few studies have
now even shown how this microbiome {[}16{]} or specific members within
it {[}17{]} could be directly involved with the pathogenesis of CRC.
These studies have helped to provide a tantilzing link between the
bacterial microbiome and CRC. However, at this present time there
remains limited information on the bacterial microbiome before and after
successful surgery for removal of the adenoma or carcinoma and whether
it changes at all.

In this study we investigated what happened to the bacterial microbiome
before and after surgery for indivdiuals with either adenoma or
carcinoma. Our anlaysis includes both alpha and beta diversity analysis
along with investigation of individual operational taxonomic units
(OTUs). We also utilized Random Forest models and observed how these
models as well as specific OTUs within them performed before (initial)
and after (follow up) surgery. We also used these models to look for
similar important OTUs to identify the crucial OTUs for not only
classifying initial and follow up samples but also lesion or normal.

\newpage

\subsection{Results}\label{results}

\textbf{\emph{Bacterial Community and FIT}} Based on the thetayc
distance metric, comparing the initial to the follow up samples, there
was a significant difference between the adenoma and carcinoma groups
(P-value = 0.00198) {[}Figure 1a{]}. There was also a significant
difference in change in FIT between initial and follow up samples
(P-value = 2.15e-05) {[}Figure 1b{]}. The whole community structure
before and after surgery was visualized on NMDS graphs for both adenoma
{[}Figure1c{]} (PERMANOVA = 0.997) and carcinoma {[}Figure 1d{]}
(PERMANOVA = 0.002). When all initial and follow up samples were
compared to each other there was no significant overall difference
between them (PERMANOVA = 0.085). There was no significant difference
between initial and follow up samples for observed OTUs, Shannon
diversity, or evenness after correction for multiple comparisons
{[}Table S1{]}. There was also no significant difference between initial
and follow up samples for any single OTU {[}Figure S1{]}.

\textbf{\emph{Previously Associated Cancer Bacteria}} We next examined
whether there were differences in previously well described carcinoma
associated OTUs. These included the OTUs that aligned with
\emph{Porphyromonas asaccharolytica} (Otu000153), \emph{Fusobacterium
nucleatum} (Otu000226), \emph{Parvimonas micra} (Otu000460), and
\emph{Peptostreptococcus stomatis} (Otu000653). First, the carcinoma
samples showed a significant difference between initial and follow up
samples for \emph{Peptostreptococcus stomatis} (P-value = 0.0183) and
\emph{Porphyromonas asaccharolytica} (P-value = 0.0154) whereas there
were no signficant differences in any of these OTUs in the adenoma
samples {[}Table S4{]}. Second, when these OTUs were present, there was
a clear magnitude difference based on whether they were from adenoma or
carcinoma {[}Figure 2{]}. However, only a small percentage of those with
adenoma or carcinoma were positive for any of these OTUs.

\textbf{\emph{Full and Reduced Model}} Since differences were observed
between initial and follow up samples and only a small number of
individuals were positive for previously associated CRC bacteria; we
next investigated if we could create models that could adequately
classify and adjust either lesion or before sample probability based on
the bacterial community and FIT. The lesion model had an AUC range of
0.723 to 0.795 versus the before sample model which had an AUC range of
0.451 to 0.67 after 100 iterations of 20 repeated 10-fold cross
validations. Interestingly, identification of the most important
variables and reducing the models to only these factors increased the
AUC in the lesion model (0.791 - 0.847) and before sample model (0.651 -
0.79).

The test set AUC range for the full and reduced lesion model were
similar to that reported for the training set AUC ranges and the ROC
curve ranges overlap each other {[}Figure 3a{]}. The ROC curve for the
final lesion model used was within the range of both the full and
reduced lesion model {[}Figure 3a{]}. Interestingly, the test set AUC
range for the before sample model performed much better then the
training set AUCs. Both the full and reduced before sample models
overlapped with each other {[}Figure 3b{]} there was a marked decrease
in the ROC curve for the final model used.

\textbf{\emph{Most Important Variables}} The reduced models were built
based on the most important variables to the respective full model. For
the lesion model there were a total of 37 variables {[}Figure S2{]}
whereas for the before sample model there were a total of 33 variables
{[}Figure S3{]}. For both models FIT resulted in the largest decrease in
MDA {[}Figure S2a \& S3a{]}.

\textbf{\emph{Positive Probability after Lesion Removal}} Regardless of
model used there was a significant decrease in the positive probability
of either the sample being lesion or a before sample on follow up
{[}Figure 4 \& S4{]} (full lesion P-value = 1.11e-11, reduced lesion
P-value = 1.91e-11, initial and follow up P-value = 6.71e-12, and
reduced initial and follow up P-value = 6.72e-12).

For the full and reduced lesion model there was a significant difference
in the classification for the lesion model between predicted and actual
(P-value = 4.19e-10 and 6.98e-10, respectively) but not for the before
sample model (P-value = 1.00 and 1.00). However, the lesion model
correctly kept the one individual who still had a carcinoma on follow up
above the cut off threshold {[}Figure 4a \& S4a{]} for a positive call
while the before sample models did not {[}Figure 4b \& S4b{]}.

\textbf{\emph{Common OTUs to both Models}} There were a total of 14 OTUs
that were common to both models. Of these OTUs the most common taxonomic
identifications were to Blautia, Bacteroides, Streptococcus, and
Clostridiales. The majority of these OTUs had classifications to
bacteria typically thought of as commensal {[}Table S2{]}.

\textbf{\emph{Treatment and Time Differences}} There was no difference
in the amount of change in positive probability for either the full or
reduced lesion model for either chemotherapy (P-value = 0.821 and 0.821)
or radiation therapy (P-value = 0.69 and 0.981). Although the before
sample model was similar there was a significant decrease in positive
probability for those treated with chemotherapy (P-value = 7.04e-04 and
5.07e-03). Time elapsed between the collection of the follow up sample
from the initial sample, did not have a significant difference between
adenoma and carcinoma (uncorrected P-value = 0.784).

\newpage

\subsection{Discussion}\label{discussion}

There was no difference in alpha diversity metrics between the initial
and follow up samples {[}Table S1{]}. Yet based on thetayc there was a
significant difference between inital and follow up samples {[}Figure
1a{]} with carcinoma samples being more dissimilar to their initial
samples then adenoma samples. The change in FIT was also different
between initial and follow up for those with adenoma or carcinoma
{[}Figure 1b{]}. There was also a significant difference in the overall
bacterial community structure for those in the carcinoma group {[}Figure
1d{]} before and after surgery but not for the adenoma group {[}Figure
1c{]}. Investigation of the OTU relative abundance before and after
surgery found no single OTU to be significantly different {[}Figure
S1{]}.

Interestingly, when only previously associated cancer bacteria were
investigated only 2 out of the 4 had a significant decrease in relative
abundance between initial and follow up for the carcinoma group and 0
out of the 4 were significant for the adenoma group. This data suggests
that the changes that may be important in the bacterial microbiome after
surgery are not necessarily any one specific bacterium, a clear
depression of CRC related bacteria, or any addition of new or depressed
bacterium. Instead it may be due to a shift of the bacterial community
as a whole between existing members of the host's microbiome.

We next created a model that incorporated FIT and the bacterial
microbiome to either be able to classify lesions (adenoma or carcinoma)
or before samples in order to confirm that the community was what was
responding or changing due to surgery. We found that the OTUs that made
up the most important variables to the model overwhelmingly belonged to
comensal bacteria. With only the lesion model having a single OTU from a
previously associated cancer bacterium (\emph{Porphyromonas
asaccharolytica}). Using only these important OTUs and FIT both models
(lesion and before sample) had a significantly decreased positive
probability of either lesion or being a before sample on follow up
{[}Figure 4 \& S4{]}. Further confirmation of the importance of the
changes of comensal bacteria to these classifications was that a total
of 14 OTUs were common to both models and the vast majority belonged to
regular residents of our gut community.

There was no difference for the majority of models tested for
differences in positive probablity based on whether chemotherapy or
radiation was received. There was also no difference in the thetayc
distance metric based on length of time between initial and follow up
sample between adenoma and carcinoma. These results would indicate that
the findings described were specific to the surgical intervention and
that differences observed between carcinoma and adenoma samples can not
be simply attributed to collection time between samples.

This study builds upon previous work from numerous labs that have looked
into the bacterial microbiome as a potential screening tool {[}13,14{]}
by exploring what happens to the bacterial community after surgical
removal of a lesion. Based on previous work by Arthur, et al. {[}18{]}
it may not be surprising to have E.coli as one of the most important
OTUs and one that was common to both models. Interestingly, many of the
most important OTUs had taxonomic identification for resident gut
microbes. This could suggest that the bacterial community is one of the
first components that could change during the pathogenesis of disease.
These bacterial microbiome changes could be the first step in allowing
more inflammatory bacterium to gain a foothold within the colon
{[}19{]}.

Curiously, we observed that the typical CRC associated bacteria were not
predictive within our models. There are a number of reasons why this may
have occured. First, one potential explanation is that even with surgery
and a shift of the bacterial community these specific bacteria still
persisted within the colon. Second, the bacteria even if they were
reduced in relative abundance they were still present within the gut.
Third, is that they were not present in enough individuals to be able to
classify those with and without disease with a high degree of accuracy.
Fourth, is that it is possible that our Random Forest models were able
to get the same information from measures such as FIT or other OTUs.
Finally, it is also possible that all of these potential explanations
could have played a role. Regardless, our observations would suggest
that an individual's resident bacteria have a large role to play in
disease initiation and could change in a way that allows predictive
models to lower the positive probability of a lesion after surgery
{[}Figure 4{]}. It should be noted that our study does not argue against
the importance of these CRC associated bacteria in the pathogenesis of
disease but rather that the models do not utilize these specific
bacteria for classification purposes (lesion or before sample). In fact,
it is possible that these CRC associated bacteria are important in the
transition from adenoma to carcinoma and would be one explanation as to
why in our data we not only see high initial relative abundances, in
certain individuals, but also large decreases in relative abundance in
those with carcinoma but not in those with adenoma after surgery
{[}Figure 2{]}.

One limitation of our study is that we do not know whether individuals
who were still classified as positive by the lesion model eventually had
a subsequent CRC diagnosis. This information would help to strengthen
the case for our Random Forest based model to have kept a number of
individuals above the cutoff threshold even though at follow up they
were diagnosed as no longer having a lesion. Another limitation is that
we do not know if adding modern tests such as the stool DNA test
{[}20{]} could help improve our overall AUC. Another limitation is that
this study drew heavily from those with caucasian ancestry. The results
may not be immediately representative of those with either Asian or
African ancestry. Finally, although our training and test set are
relatively large we still run the risk of overfitting or having a model
that may not be immediately extrapolateable to other populations. We've
done our best to safeguard against this by not only running 10-fold
cross validation but also having over 100 different 80/20 splits to try
and mimic the type of variation that might be expected to occur.

Another interesting outcome was that within figure 3 the before sample
model showed better test AUC results then the training set AUC. This may
have occured because the training AUC that was determined from 20
repeated 10 fold cross validation removed samples at random and did not
take into account that they were matched samples. Another potential
reason is that the model itself may be overfit since the total number of
samples was not that large. However, the lesion model did not suffer
from these discrepencies and similar conclusions can be drawn solely
from this model. Regardless, further independent studies will need to be
carried out to verify our findings since not only are we dealing with
feces, which could be very different than the communities present on the
actual tissue, but also are dealing with correlations that may not be
representative of the true pathogensis of disease.

Despite these limitations we think that these findings significantly add
to the existing scientific knowledge on CRC and the bacterial
microbiome. The ability for machine learning algorithms to take
bacterial microbiome data and successfully lower positive probability
after either adenoma or carcinoma removal provides evidence that there
are specific signatures, mostly attributable to commensal organisms,
associated with these lesions. It also shows that these algorithms can
not only successfully react to successful treatment regimens but also
may be able to one day stratify CRC disease risk with a high level of
accuracy.

\newpage

\subsection{Methods}\label{methods}

\textbf{\emph{Study Design and Patient Sampling}} The sampling and
design of the study was similar to that reported in Baxter, et al
{[}13{]}. In brief, study exclusion involved those who had already
undergone surgery, radiation, or chemotherapy, had colorectal cancer
before a baseline fecal sample could be obtained, had IBD, a known
hereditary non-polyposis colorectal cancer, or Familial adenomatous
polyposis. Samples used to build the models for prediction were
collected either prior to a colonoscopy or between 1 - 2 weeks after.
The bacterial microbiome has been shown to normalize within this time
period {[}21{]}. Our follow up data set had a total of 67 individuals
that not only had a sample as described but also a follow up sample
between 188 - 546 days after surgery and treatment had been completed.
This study was approved by the University of Michigan Institutional
Review Board. All study participants provided informed consent and the
study itself conformed to the guidelines set out by the Helsinki
Declaration.

\textbf{\emph{FIT and 16S rRNA Gene Sequencing}} FIT was analyzed as
previously published using both OC FIT-CHEK and OC-Auto Micro 80
automated system (Polymedco Inc.) {[}22{]}. 16S rRNA gene sequencing was
completed as previously described by Kozich, et al. {[}23{]}. In brief,
DNA extraction used the 96 well Soil DNA isolation kit (MO BIO
Laboratories) and an epMotion 5075 automated pipetting system
(Eppendorf). The V4 variable region was amplified and the resulting
product was split between three sequencing runs with normal, adenoma,
and carcinoma evenly represented on each run. Each group was randomly
assigned to avoid biases based on sample collection location.

\textbf{\emph{Sequence Processing}} The mothur software package
(v1.37.5) was used to process the 16S rRNA gene sequences. This process
has been previously described {[}23{]}. The general processing workflow
using mothur was as follows: Paired-end reads were first merged into
contigs, quality filtered, aligned to the SILVA database, screened for
chimeras, classified with a naive Bayesian classifier using the
Ribosomal Database Project (RDP), and clustered into Operational
Taxonomic Units (OTUs) using a 97\% similarity cutoff with an average
neighbor clustering algorithm. The numer of sequences for each sample
was rarified to 10521 in an attempt to minimize uneven sampling.

\textbf{\emph{Lesion Model Creation}} The Random Forest {[}24{]}
algorithm was used to create the model used for prediction of lesion
(adenoma or carcinoma) with the main testing and training of the model
completed on a data set of 490 individuals. This model was then applied
to our follow up data set of 67 individuals. The model included data on
FIT and the bacterial microbiome. Non-binary data was checked for near
zero variance and OTUs that had near zero variance were removed. This
pre-processing was performed with the R package caret (v6.0.73).
Optimization of the mtry hyperparameter involved taking the samples and
making 100 different 80/20 (train/test) splits of the data where normal
and lesion were represented in the same proportion within both the whole
data set and the 80/20 split. Each of these splits were then run through
20 repeated 10-fold cross validations to optimize the mtry
hyperparameter by maximizing the AUC (Area Under the Curve of the
Receiver Operator Characteristic). This resulting model was then tested
on the 20\% of the data that was originally held out from this overall
process. Once the ideal mtry was found the entire 490 sample set was
used to create the final Random Forest model on which classifications on
the 67-person cohort was completed. The default cutoff of 0.5 was used
as the threshold to classify individuals as positive or negative for
lesion. The hyperparameter, mtry, defines the number of variables to
investigate at each split before a new division of the data is created
with the Random Forest model.

\textbf{\emph{Before Sample Model Creation}} We also investigated
whether a model could be created that could identify before and after
surgery samples. The main difference was that only the 67-person cohort
was used at all stages of model building and classification. Other than
this difference the creation of this model and optimization of the mtry
hyperparameter was completed using the same procedure that was used to
create the lesion model. Instead of classifying samples as positive or
negative of lesion this model classified samples as positive or negative
for being a before surgery sample.

\textbf{\emph{Selection of Important OTUs}} In order to assess which
variables were most important to all the models we counted the number of
times a variable was present in the top 10\% of mean decrease in
accuracy (MDA) for each of the 100 different 80/20 split models and then
filtered this list to variables that were only present more than 50\% of
the time. This final collated list of variables was what was considered
the most important for the lesion or before sample models.

\textbf{\emph{Statistical Analysis}} The R software package (v3.3.2) was
used for all statisitical analysis. Comparisons between bacterial
community structure utilized PERMANOVA {[}25{]} in the vegan package
(v2.4.1). Comparisons between probabilities as well as overall OTU
differences between initial and follow up samples utilized a paired
wilcoxson ranked sum test. Where multiple comparison testing was needed
a Benjamini-Hochberg (BH) correction was applied {[}26{]} and a
corrected P-value of less than 0.05 was considered significant. Unless
otherwise stated the P-values reported are those that were BH corrected.

\textbf{\emph{Analysis Overview}} Initial and follow up samples were
analyzed for differences in alpha and beta diversity. Next, differences
in FIT between initial and follow ups for either adenoma or carcinoma
were investigated. From here, all OTUs that were used in either model
were then analyzed using a paired wilcoxson test. We then investigated
the relative abundance of specific previously associated CRC bacteria,
specifically, OTUs that taxonomically classified to \emph{Fusobacterium
nucleatum}, \emph{Parvimonas micra}, \emph{Peptostreptococcus
assacharolytica}, and \emph{Porphyromonas stomatis}. We wanted to test
if there were any differences based on whether the individual had an
adenoma or carcinoma. From here the lesion model was then tested for
accuracy in prediction and whether it reduced the positive probability
of lesion after surgery. The most important OTUs for this were used to
build a reduced model and it was assessed for similarity to the original
model. We then used the before sample model to assess whether it could
classify samples better then the lesion model. The most important OTUs
were then identified from this model and used to create a reduced
feature before sample model. This reduced feature model, as was done
with the lesion model, was compared to the full model for loss of
accuracy. Finally, a list of common OTUs were found for the two
different models used.

\textbf{\emph{Reproducible Methods.}} A detailed and reproducible
description of how the data were processed and analyzed can be found at
\url{https://github.com/SchlossLab/Sze_followUps_2017}. Raw sequences
have been deposited into the NCBI Sequence Read Archive (SRP062005 and
SRP096978) and the necessary metadata can be found at
\url{https://www.ncbi.nlm.nih.gov/Traces/study/} and searching the
respective SRA study accession.

\newpage

\textbf{Figure 1: General Differences between the Adenoma or Carcinoma
Group.} A) A significant difference was found between the adenoma and
carcinoma group for thetayc (P-value = 0.00198). B) A significant
difference was found between the adenoma and carcinoma group for change
in FIT (P-value = 2.15e-05). C) NMDS of the intial and follow up samples
for the Adenoma group. D) NMDS of the initial and follow up samples for
the Carcinoma group. For C) and D) the teal represents initial samples
and the pink represents follow up samples.

\textbf{Figure 2: Previously Associated CRC Bacteria in Initial and
Follow up Samples.} A) Carcinoma initial and follow up samples. There
was a significant difference in initial and follow up sample for the
OTUs classfied as \emph{Peptostreptococcus stomatis} (P-value = 0.0183)
and \emph{Porphyromonas asaccharolytica} (P-value = 0.0154). B) Adenoma
initial and follow up samples. There were no significant differences
between initial and follow up.

\textbf{Figure 3: Graph of the Receiver Operating Characteristic Curve
for lesion and Before Sample Models.} The shaded areas represents the
range of values of a 100 different 80/20 splits of the test set data
using either all variables (grey) or reduced variable (red) models. The
blue line represents the reduced variable model using 100\% of the data
set. A) Lesion model. B) Defore sample model

\textbf{Figure 4: Breakdown by Carcinoma and Adenoma of Prediction
Results for Lesion and Before Sample Reduced Variable Models} A) Lesion
positive probability adjustment of those with carcinoma from initial to
follow up sample B) Initial follow up positive probability adjustment of
those with carcinoma from initial to follow up sample C) Lesion positive
probability adjustment of those with adenoma as well as those with SRN
and the probability adjustment from initial to follow up sample. D)
Initial follow up positive probability adjustment of those with adenoma
as well as those with SRN and the probability adjustment from initial to
follow up sample. The dotted line represents the threshold used to make
the decision of whether a sample was positive or not.

\newpage

\textbf{Figure S1: Distribution of P-values from Paired Wilcoxson
Analysis of OTUs in Initial versus Follow Up}

\textbf{Figure S2: Summary of Important Variables in the Lesion Model}
A) MDA of the most important variables in the lesion model. The black
point represents the median and the different colors are the different
runs up to 100. B) The total number of appearances of each variable in
the 100 different lesion models. The cutoff of 50\% was used to assess
importance.

\textbf{Figure S3: Summary of Important Variables in Before Sample
Model} A) MDA of the most important variables in the lesion model. The
black point represents the median and the different colors are the
different runs up to 100. B) The total number of appearances of each
variable in the 100 different lesion models. The cutoff of 50\% was used
to assess importance.

\textbf{Figure S4: Breakdown by Carcinoma and Adenoma of Prediction
Results for Lesion and Before Sample Full Variable Models} A) Lesion
positive probability adjustment of those with carcinoma from initial to
follow up sample B) Initial follow up positive probability adjustment of
those with carcinoma from initial to follow up sample C) Lesion positive
probability adjustment of those with adenoma as well as those with SRN
and the probability adjustment from initial to follow up sample. D)
Initial follow up positive probability adjustment of those with adenoma
as well as those with SRN and the probability adjustment from initial to
follow up sample. The dotted line represents the threshold used to make
the decision of whether a sample was positive or not.

\textbf{Figure S5: Thetayc Graphed Against Time of Follow up Sample from
Initial}

\newpage

\subsection{Declarations}\label{declarations}

\subsubsection{Ethics approval and consent to
participate}\label{ethics-approval-and-consent-to-participate}

\subsubsection{Consent for publication}\label{consent-for-publication}

\subsubsection{Availability of data and
material}\label{availability-of-data-and-material}

\subsubsection{Competing Interests}\label{competing-interests}

All authors declare that they do not have any relevent competing
interests to report.

\subsubsection{Funding}\label{funding}

This study was supported by funding from the National Institutes of
Health to P. Schloss (R01GM099514, P30DK034933) and to the Early
Detection Research Network (U01CA86400).

\subsubsection{Authors' contributions}\label{authors-contributions}

All authors were involved in the conception and design of the study. MAS
analyzed the data. NTB processed samples and analyzed the data. All
authors interpreted the data. MAS and PDS wrote the manuscript. All
authors reviewed and revised the manuscript. All authors read and
approved the final manuscript.

\subsubsection{Acknowledgements}\label{acknowledgements}

The authors thank the Great Lakes-New England Early Detection Research
Network for providing the fecal samples that were used in this study.

\newpage

\subsection*{References}\label{references}
\addcontentsline{toc}{subsection}{References}

\hypertarget{refs}{}
\hypertarget{ref-jemal_cancer_2010}{}
1. Jemal A, Siegel R, Xu J, Ward E. Cancer statistics, 2010. CA: a
cancer journal for clinicians. 2010;60:277--300.

\hypertarget{ref-haggar_colorectal_2009}{}
2. Haggar FA, Boushey RP. Colorectal cancer epidemiology: Incidence,
mortality, survival, and risk factors. Clinics in Colon and Rectal
Surgery. 2009;22:191--7.

\hypertarget{ref-green_very_2007}{}
3. Green RC, Green JS, Buehler SK, Robb JD, Daftary D, Gallinger S, et
al. Very high incidence of familial colorectal cancer in Newfoundland: A
comparison with Ontario and 13 other population-based studies. Familial
Cancer. 2007;6:53--62.

\hypertarget{ref-liao_application_2013}{}
4. Liao C-S, Lin Y-M, Chang H-C, Chen Y-H, Chong L-W, Chen C-H, et al.
Application of quantitative estimates of fecal hemoglobin concentration
for risk prediction of colorectal neoplasia. World Journal of
Gastroenterology. 2013;19:8366--72.

\hypertarget{ref-johnson_multi-target_2016}{}
5. Johnson DH, Kisiel JB, Burger KN, Mahoney DW, Devens ME, Ahlquist DA,
et al. Multi-target stool DNA test: Clinical performance and impact on
yield and quality of colonoscopy for colorectal cancer screening.
Gastrointestinal Endoscopy. 2016;

\hypertarget{ref-laiyemo_short-_2013}{}
6. Laiyemo AO, Doubeni C, Brim H, Ashktorab H, Schoen RE, Gupta S, et
al. Short- and long-term risk of colorectal adenoma recurrence among
whites and blacks. Gastrointestinal Endoscopy. 2013;77:447--54.

\hypertarget{ref-matsuda_five-year_2009}{}
7. Matsuda T, Fujii T, Sano Y, Kudo S-e, Oda Y, Igarashi M, et al.
Five-year incidence of advanced neoplasia after initial colonoscopy in
Japan: A multicenter retrospective cohort study. Japanese Journal of
Clinical Oncology. 2009;39:435--42.

\hypertarget{ref-ren_long-term_2016}{}
8. Ren J, Kirkness CS, Kim M, Asche CV, Puli S. Long-term risk of
colorectal cancer by gender after positive colonoscopy: Population-based
cohort study. Current Medical Research and Opinion. 2016;32:1367--74.

\hypertarget{ref-loberg_long-term_2014}{}
9. Løberg M, Kalager M, Holme Ø, Hoff G, Adami H-O, Bretthauer M.
Long-term colorectal-cancer mortality after adenoma removal. The New
England Journal of Medicine. 2014;371:799--807.

\hypertarget{ref-freeman_natural_2013}{}
10. Freeman HJ. Natural history and long-term outcomes of patients
treated for early stage colorectal cancer. Canadian Journal of
Gastroenterology = Journal Canadien De Gastroenterologie.
2013;27:409--13.

\hypertarget{ref-lee_identification_2016}{}
11. Lee JH, Lee JL, Park IJ, Lim S-B, Yu CS, Kim JC. Identification of
Recurrence-Predictive Indicators in Stage I Colorectal Cancer. World
Journal of Surgery. 2016;

\hypertarget{ref-richards_evidence-based_2016}{}
12. Richards CH, Ventham NT, Mansouri D, Wilson M, Ramsay G, Mackay CD,
et al. An evidence-based treatment algorithm for colorectal polyp
cancers: Results from the Scottish Screen-detected Polyp Cancer Study
(SSPoCS). Gut. 2016;

\hypertarget{ref-baxter_microbiota-based_2016}{}
13. Baxter NT, Ruffin MT, Rogers MAM, Schloss PD. Microbiota-based model
improves the sensitivity of fecal immunochemical test for detecting
colonic lesions. Genome Medicine. 2016;8:37.

\hypertarget{ref-zeller_potential_2014}{}
14. Zeller G, Tap J, Voigt AY, Sunagawa S, Kultima JR, Costea PI, et al.
Potential of fecal microbiota for early-stage detection of colorectal
cancer. Molecular Systems Biology. 2014;10:766.

\hypertarget{ref-dejea_microbiota_2014}{}
15. Dejea CM, Wick EC, Hechenbleikner EM, White JR, Mark Welch JL,
Rossetti BJ, et al. Microbiota organization is a distinct feature of
proximal colorectal cancers. Proceedings of the National Academy of
Sciences of the United States of America. 2014;111:18321--6.

\hypertarget{ref-zackular_manipulation_2016}{}
16. Zackular JP, Baxter NT, Chen GY, Schloss PD. Manipulation of the Gut
Microbiota Reveals Role in Colon Tumorigenesis. mSphere. 2016;1.

\hypertarget{ref-arthur_microbial_2014}{}
17. Arthur JC, Gharaibeh RZ, Mühlbauer M, Perez-Chanona E, Uronis JM,
McCafferty J, et al. Microbial genomic analysis reveals the essential
role of inflammation in bacteria-induced colorectal cancer. Nature
Communications. 2014;5:4724.

\hypertarget{ref-arthur_intestinal_2012}{}
18. Arthur JC, Perez-Chanona E, Mühlbauer M, Tomkovich S, Uronis JM, Fan
T-J, et al. Intestinal inflammation targets cancer-inducing activity of
the microbiota. Science (New York, N.Y.). 2012;338:120--3.

\hypertarget{ref-flynn_metabolic_2016}{}
19. Flynn KJ, Baxter NT, Schloss PD. Metabolic and Community Synergy of
Oral Bacteria in Colorectal Cancer. mSphere. 2016;1.

\hypertarget{ref-cotter_long-term_2016}{}
20. Cotter TG, Burger KN, Devens ME, Simonson JA, Lowrie KL, Heigh RI,
et al. Long-Term Follow-up of Patients Having False Positive
Multi-target Stool DNA Tests after Negative Screening Colonoscopy: The
LONG-HAUL Cohort Study. Cancer Epidemiology, Biomarkers \& Prevention: A
Publication of the American Association for Cancer Research, Cosponsored
by the American Society of Preventive Oncology. 2016;

\hypertarget{ref-obrien_impact_2013}{}
21. O'Brien CL, Allison GE, Grimpen F, Pavli P. Impact of colonoscopy
bowel preparation on intestinal microbiota. PloS One. 2013;8:e62815.

\hypertarget{ref-daly_evaluation_2013}{}
22. Daly JM, Bay CP, Levy BT. Evaluation of fecal immunochemical tests
for colorectal cancer screening. Journal of Primary Care \& Community
Health. 2013;4:245--50.

\hypertarget{ref-kozich_development_2013}{}
23. Kozich JJ, Westcott SL, Baxter NT, Highlander SK, Schloss PD.
Development of a dual-index sequencing strategy and curation pipeline
for analyzing amplicon sequence data on the MiSeq Illumina sequencing
platform. Applied and Environmental Microbiology. 2013;79:5112--20.

\hypertarget{ref-breiman_random_2001}{}
24. Breiman L. Random Forests. Machine Learning {[}Internet{]}. 2001
{[}cited 2013 Feb 7{]};45:5--32. Available from:
\href{http://link.springer.com/article/10.1023/A\%3A1010933404324\%20http://link.springer.com/article/10.1023\%2FA\%3A1010933404324?LI=true}{http://link.springer.com/article/10.1023/A\%3A1010933404324 http://link.springer.com/article/10.1023\%2FA\%3A1010933404324?LI=true}

\hypertarget{ref-anderson_permanova_2013}{}
25. Anderson MJ, Walsh DCI. PERMANOVA, ANOSIM, and the Mantel test in
the face of heterogeneous dispersions: What null hypothesis are you
testing? Ecological Monographs {[}Internet{]}. 2013 {[}cited 2017 Jan
5{]};83:557--74. Available from:
\url{http://doi.wiley.com/10.1890/12-2010.1}

\hypertarget{ref-benjamini_controlling_1995}{}
26. Benjamini Y, Hochberg Y. Controlling the false discovery rate: A
practical and powerful approach to multiple testing. Journal of the
Royal Statistical Society. Series B (Methodological). 1995;57:289--300.


\end{document}
