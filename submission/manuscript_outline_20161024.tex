\documentclass[12pt,]{article}
\usepackage{lmodern}
\usepackage{amssymb,amsmath}
\usepackage{ifxetex,ifluatex}
\usepackage{fixltx2e} % provides \textsubscript
\ifnum 0\ifxetex 1\fi\ifluatex 1\fi=0 % if pdftex
  \usepackage[T1]{fontenc}
  \usepackage[utf8]{inputenc}
\else % if luatex or xelatex
  \ifxetex
    \usepackage{mathspec}
  \else
    \usepackage{fontspec}
  \fi
  \defaultfontfeatures{Ligatures=TeX,Scale=MatchLowercase}
\fi
% use upquote if available, for straight quotes in verbatim environments
\IfFileExists{upquote.sty}{\usepackage{upquote}}{}
% use microtype if available
\IfFileExists{microtype.sty}{%
\usepackage{microtype}
\UseMicrotypeSet[protrusion]{basicmath} % disable protrusion for tt fonts
}{}
\usepackage[margin=1.0in]{geometry}
\usepackage{hyperref}
\hypersetup{unicode=true,
            pdfborder={0 0 0},
            breaklinks=true}
\urlstyle{same}  % don't use monospace font for urls
\usepackage{graphicx,grffile}
\makeatletter
\def\maxwidth{\ifdim\Gin@nat@width>\linewidth\linewidth\else\Gin@nat@width\fi}
\def\maxheight{\ifdim\Gin@nat@height>\textheight\textheight\else\Gin@nat@height\fi}
\makeatother
% Scale images if necessary, so that they will not overflow the page
% margins by default, and it is still possible to overwrite the defaults
% using explicit options in \includegraphics[width, height, ...]{}
\setkeys{Gin}{width=\maxwidth,height=\maxheight,keepaspectratio}
\IfFileExists{parskip.sty}{%
\usepackage{parskip}
}{% else
\setlength{\parindent}{0pt}
\setlength{\parskip}{6pt plus 2pt minus 1pt}
}
\setlength{\emergencystretch}{3em}  % prevent overfull lines
\providecommand{\tightlist}{%
  \setlength{\itemsep}{0pt}\setlength{\parskip}{0pt}}
\setcounter{secnumdepth}{0}
% Redefines (sub)paragraphs to behave more like sections
\ifx\paragraph\undefined\else
\let\oldparagraph\paragraph
\renewcommand{\paragraph}[1]{\oldparagraph{#1}\mbox{}}
\fi
\ifx\subparagraph\undefined\else
\let\oldsubparagraph\subparagraph
\renewcommand{\subparagraph}[1]{\oldsubparagraph{#1}\mbox{}}
\fi

%%% Use protect on footnotes to avoid problems with footnotes in titles
\let\rmarkdownfootnote\footnote%
\def\footnote{\protect\rmarkdownfootnote}

%%% Change title format to be more compact
\usepackage{titling}

% Create subtitle command for use in maketitle
\newcommand{\subtitle}[1]{
  \posttitle{
    \begin{center}\large#1\end{center}
    }
}

\setlength{\droptitle}{-2em}
  \title{}
  \pretitle{\vspace{\droptitle}}
  \posttitle{}
  \author{}
  \preauthor{}\postauthor{}
  \date{}
  \predate{}\postdate{}

\usepackage{helvet} % Helvetica font
\renewcommand*\familydefault{\sfdefault} % Use the sans serif version of the font
\usepackage[T1]{fontenc}

\usepackage[none]{hyphenat}

\usepackage{setspace}
\doublespacing
\setlength{\parskip}{1em}

\usepackage{lineno}

\usepackage{pdfpages}
\usepackage{comment}
\usepackage{lscape}

\begin{document}

\section{The Fecal Microbiome Before and After Treatment for Colorectal
Adenoma or
Carcinoma}\label{the-fecal-microbiome-before-and-after-treatment-for-colorectal-adenoma-or-carcinoma}

\vspace{25mm}

\begin{center}
Running Title: Human Microbiome before and after Colorectal Cancer

\vspace{10mm}

Marc A Sze${^1}$, Nielson T Baxter${^2}$, Mack T Ruffin IV${^3}$, Mary AM Rogers${^2}$, and Patrick D Schloss${^1}$${^\dagger}$

\vspace{20mm}

$\dagger$ To whom correspondence should be addressed: pschloss@umich.edu

$1$ Department of Microbiology and Immunology, University of Michigan, Ann Arbor, MI

$2$ Department of Internal Medicine, University of Michigan, Ann Arbor, MI   

$3$ Department of Family Medicine and Community Medicine, Penn State Hershey Medical Center, Hershey, PA    


\end{center}

\newpage

\linenumbers

\subsection{Abstract}\label{abstract}

\textbf{Background:} Colorectal cancer (CRC) is a worldwide health
problem and research suggests a correlation between the fecal bacterial
microbiome and CRC. This study tested the hypothesis that treatment for
adenoma or carcinoma results in changes to the bacterial community.
Specifically, we tried to identify components within the community that
were different before and after treatment of adenoma, advanced adenoma
(Screen Relevant Neoplasia (SRN)), and carcinoma.

\textbf{Results:} There was a larger change in the bacterial community
in response to treatment for carcinoma versus adenoma (P-value
\textless{} 0.05) but not carcinoma versus SRN cases (P-value
\textgreater{} 0.05). But there was a trend for increasingly less
community similarity between initial and follow up samples from adenoma
to SRN to carcinoma. Despite this, no difference was found in the
relative abundance of any OTU before and after treatment for adenoma,
SRN, or carcinoma groups (P-value \textgreater{} 0.05). Using Random
Forest models to assess whether changes in follow up samples were
towards a normal community showed that only those with carcinoma had a
significant decrease in positive probability (P-value \textless{} 0.05);
providing further evidence that treatment has the greatest effect in
those with carcinoma. The adenoma model used a total of 62 OTUs, the SRN
model used a total of 61 OTUs, and the carcinoma model used a total of
59 OTUs. A total of 26 OTUs were common to both models with many
classifying to commensal bacteria (e.g. \emph{Lachnospiraceae},
\emph{Bacteroides}, \emph{Anaerostipes}, \emph{Blautia}, and
\emph{Dorea}). Both chemotherapy and radiation did not provide any
additional changes to the bacterial community in those treated for
carcinoma (Pvalue \textgreater{} 0.05).

\textbf{Conclusions:} Our data partially supports the hypothesis that
the bacterial community changes after treatment. Individuals with
carcinoma had more drastic differences to the overall community then
those with adenoma. Common OTUs to all models were overwhelmingly from
commensal bacteria, suggesting that these bacteria may be important to
initial polyp formation, development of advanced adenoma, and transition
to carcinoma.

\newpage

\subsubsection{Keywords}\label{keywords}

bacterial microbiome; colorectal cancer; polyps; FIT; post-surgery; risk
factors

\newpage

\subsection{Background}\label{background}

Colorectal cancer (CRC) is currently the third most common cause of
cancer deaths {[}1,2{]}. The rate of disease mortality has seen a
significant decrease, thanks mainly to improvements in screening
{[}1{]}. However, despite this improvement there are still approximately
50,000 deaths from the disease per year {[}2{]}.

Recent studies in humans have shown that both the microbiome and
specific members within it correlate with CRC pathogenesis {[}3,4{]}.
Further, bacterial communities have been observed to be altered between
normal and tumor tissue {[}5{]}. Mouse models of CRC have further
demonstrated the importance of the microbiome, both on a community
{[}3,6{]} and species level {[}4{]}, for tumorgenesis. Collectively,
these studies provide a tantalizing link between our gut bacteria and
CRC and suggest that biomarkers using our microbes could be developed.
Indeed, builidng models using 16S rRNA gene sequencing along with
clinical tests such as Fecal Immunoglobulin Test (FIT) result in good
predictions of CRC {[}7,8{]}. Although these studies show how our gut
bacteria can impact CRC progression via a changed community or invasion
by more inflammatory bacteria {[}9{]}. They provide very little
information as to whether these communities change and rebound towards
normal after successful treatment of adenoma, advanced adenoma (Screen
Relevent Neoplasia (SRN)), or carcinoma.

Providing an answer to this question is important because it has far
reaching implications on both how the bacterial community causes the
formation of more polyps {[}3,4{]} and the ability to be able to use the
microbiome as a predictive screening tool {[}7,8{]}. Understanding polyp
formation and transition to SRN and then carinoma is crucial to being
able to understand how to prevent CRC occurence. Response of the
community to treatment is also eqaully important to predictive models
designed for screening purposes since an unresponsive community would
provide little additional information for important events, such as
recurrence {[}10{]}.

Using pre- (initial) and post- (follow up) treatment samples we tested
the hypothesis that treatment causes detectable changes to the
microbiome in those with adenoma, SRN, and carincoma. First, we assessed
differences between initial and follow up samples in adenoma, SRN, or
carinoma using alpha or beta diversity metrics. Second, we explored
whether models built to classify adenoma, SRN, or carcinoma versus
normal were able to identify specific community members that differed
between initial and follow up. We also used these models to assess
whether changes in the community were toward a more normal micorbiome.
Finally, we assessed both whether surgery for adenomas and SRN provided
larger community changes or whether chemotherapy or radiation provided
additive changes to the microbiome over surgerical resection. This study
helps to provide evidence as to whether treatment can influence the
community and if the CRC microbiome, identified in previous studies,
persists after such interventions.

\newpage

\subsection{Results}\label{results}

\textbf{\emph{The Bacterial Community:}} Within our 67-person cohort we
tested whether those with adenoma (n = 22), SRN (n = 19), or carcinoma
(n = 26) had any broad differences between their initial and follow up
samples. We found that those with carcinoma had a more dissimilar
bacterial community between their initial and follow up sample than
those with adenoma (P-value \textless{} 0.001) {[}Figure 1A{]}. Although
no significant differences were observed between SRN and carcinoma there
was an increase in the dissimilarity of the initial and follow up
samples from adenoma (0.55 ± 0.21 (mean ± SD)) to SRN (0.65 ± 0.25) to
carcinoma (0.78 ± 0.15) {[}Figure 1A{]}. The bacterial community
structure before and after surgery was visualized using NMDS for adenoma
{[}Figure 1B{]} (PERMANOVA \textgreater{} 0.05), SRN {[}Figure 1C{]}
(PERMANOVA \textgreater{} 0.05), and carcinoma {[}Figure 1D{]}
(PERMANOVA \textless{} 0.05). Interestingly, when initial and follow up
samples were compared, regardless of whether the lesions were adenoma or
carcinoma, there was no significant overall difference in beta diversity
(PERMANOVA \textgreater{} 0.05). There was no difference between initial
and follow up samples when investigating alpha diversity metrics for
adenoma, SRN, or carcinoma for any metric tested {[}Table S1{]}.
Additionally, there was also no difference in the relative abundance of
any OTU between initial and follow up samples for adenoma, SRN, or
carcinoma only {[}Figure S1{]}.

\textbf{\emph{Adenoma Model}} The range of model AUC's from 100 runs of
20 repeated 10 fold cross-validation was 0.62 - 0.72 with the AUC of the
model used for classification havin an AUC of 0.65 {[}Figure S2A{]}.
There was a total of 62 OTUs in this model with the vast majority
classifying to bacteria typically thought of as commensal {[}Figure
S3A{]}. There was a significant difference between the actual and model
predicted group stratification calls (P-value \textless{} 0.05). There
was no significant decrease in the positive probability of adenoma
between initial and follow up samples (P-value \textgreater{} 0.05)
{[}Figure 2A{]}.

\textbf{\emph{SRN Model}} The range of model AUC's from 100 runs of 20
repeated 10 fold cross-validation was 0.68 - 0.77 with the AUC of the
model used for classification having an AUC of 0.73 {[}Figure S2B{]}.
There was a total of 61 OTUs in the SRN model {[}Figure S3B{]}. Similar
to the adenoma model the vast majority of OTUs classified to bacteria
typically thought of as commensal. Also similar to the adenoma model
there was a significant difference between the actual and model
predicted group stratification calls (P-value \textless{} 0.05). There
was no significant decrease in the positive probability of SRN between
initial and follow up samples (P-value \textgreater{} 0.05) {[}Figure
2B{]}.

\textbf{\emph{Carcinoma Model}} The range of model AUC's from 100 runs
of 20 repeated 10 fold cross-validation was 0.84 - 0.9 with the AUC of
the model used for classification being 0.88 {[}Figure S2C{]}.
Interestingly, the AUCs improved from adenoma to SRN to carcinoma
{[}Figure 2{]}. There was a total of 59 OTUs in the carinoma model
{[}Figure S3C{]}. Similar to the adenoma and SRN models the vast
majority of OTUs classified to bacteria typically thought of as
commensal but OTUs that also classified to \emph{Fusobacterium},
\emph{Porphyromonas}, and \emph{Parvimonas} appear to be important for
carcinoma classification {[}Figure S3C{]}. Also similar to the adenoma
and SRN models there was a significant difference between the actual and
model predicted group stratification calls (P-value \textless{} 0.05).
There was a significant decrease in the positive probability of
carcinoma between initial and follow up samples (P-value \textless{}
0.05) {[}Figure 2C{]}; suggesting that the carcinoma samples changed
towards normal after treatment, unlike either adenoma or SRN. The one
indivdiual still positive for carcinoma after treatment had an increase
in carcinoma positive proability on follow up {[}Figure 2C{]}.

\textbf{\emph{Adenoma, SRN, and Carcinoma Common OTUs}} We next wanted
to know what predictors within the adenoma, SRN, and carcinoma models
were similar to each other. The main purpose was to identify which OTUs
could be important at all three stages of disease. When we compared the
three different models with each other there were a total of 26 common
OTUs. Some of the most common taxonomic identifications belonged to
\emph{Bacteroides}, \emph{Blautia}, \emph{Anaerostipes},
\emph{Lachnospiraceae}, and \emph{Dorea}. These along with the vast
majority of the OTUs that were common between these models had
classifications to bacteria typically thought of as commensal {[}Table
S2{]}.

\textbf{\emph{Treatment Affects on Community}} After observing these
these changes from treatment we assessed whether chemotherapy or
radiation, in the carcinoma group, and surgery, in the adenoma group,
impacted the observed results. In the carcinoma group neither
chemotherapy or radiation provided any additive change from initial
sample over surgery alone (P-value \textgreater{} 0.05) {[}Table S3{]}.
For the adenoma group there was a single difference in observed OTUs
(sobs) between those that received surgery and those that did not
(P-value \textless{} 0.05) {[}Table S4{]}. For the surgery comparison,
adenoma and SRN were combined due to the low number of surgery
occurances in these two groups. There was no difference in the
proportion of those receiving surgery between the adenoma and SRN groups
(P-value \textgreater{} 0.05). This data suggests that microbiome
changes observed in the carcinoma group were mostly a result from
surgery and not from chemotherapy or radiation.

\newpage

\subsection{Discussion}\label{discussion}

This study builds upon previous work from numerous labs that have
considered both how the bacterial community between those with and
without CRC differ and how it might be used as an early screening tool
{[}7,8,11--13{]}. Here we show that the bacterial community changes
towards normal after treatment for carcinoma and that chemotherapy and
radiation did not provide an additive change. Although some of the
important OTUs classified to genera from bacteria considered the usual
suspects (e.g. \emph{Fusobacterium}, \emph{Porphyromonas}, and
\emph{Parvimonas}) many did not. The majority of important OTUs had
taxonomic classifications for resident gut microbes and were common for
the adenoma, SRN, and carcinoma models. This suggests that members
within the commensal community may be the first that change during CRC
pathogenesis. These subtle changes, in turn, could be the first step in
allowing more inflammatory bacteria to gain a foothold within the colon
{[}9{]}.

Unlike previous studies on the microbiome and CRC, ours focuses both on
identifying commonalities and differences within adenoma, SRN, and
carcinoma groups before and after treatment. Although there were
differences for genera associated with specific bacterium linked with
CRC {[}Figure S3{]}. The majority of important OTUs taxonmically
classified to commensal bacteria {[}Figure S3{]}. Although these changes
may be subtle, due to the lack of signifcant difference in the bacterial
community before and after treatment in adenoma and SRN {[}Figure 1{]},
they support the hypothesis that the first members of the community to
change and potentially stay changed even after treatment are those that
are commensal bacteria.

Many of the common OTUs that we identified taxnomically classified to
potential butyrate producers (e.g. \emph{Clostridiales},
\emph{Roseburia}, and \emph{Anaerostipes}) {[}Table S2{]}. Other OTUs
classified to bacteria that are inhibited by polyphenols (e.g.
\emph{Bacteroides}). Both butyrate and polyphenols are thought to be
protective against cancer, in part by reducing inflammation {[}14{]}.
These protective compounds are derived from the breakdown of fiber,
fruits, and vegetables by resident gut microbes. One example of this
potential diet-microbiome-inflammation-polyp axis is that
\emph{Bacteroides}, which was highly prevalent in our models, are known
to be increased in those with high non-meat based protein consumption
{[}15{]}. High protein consumption in general has been linked with an
increased CRC risk {[}16{]}. Conversely, \emph{Bacteroides} are
inhibited by polyphenols which are derived from fruits and vegetables
{[}17{]}. Our data fits with the hypothesis that the microbial
metabolites from breakdown products within our own diet could not only
help to shape the existing community but also have an effect on CRC risk
and disease progression.

A limitation, in our study, was that there was a significant difference
in the time elapsed in the collection of the follow up sample between
adenoma or SRN versus carcinoma (P-value \textless{} 0.05), with time
passed being less for adenoma (255 ± 42 days) and SRN (250 ± 41) than
carcinoma (351 ± 102). These results would indicate that some of the
differences observed between the carcinoma and adenoma groups could be
due to differences in collection time. Specifically, it could confound
the observation that carcinomas changed more than adenomas {[}Figure
1{]}. However, there are two reasons that this may not be the case.
First, the SRN group did not have a significant difference and the
collection time of their follow up sample was less than the adenoma
group. Second, this confounding would not affect the observations where
models were used since they were built using a different cohort
{[}Figures 2 \& S2-S3{]}.

Another limitation was that we do not know whether individuals who were
still classified as positive by the carcinoma model eventually had a
subsequent CRC recurrence. This information would help to strengthen the
case for this model keeping numerous individuals above the cutoff
threshold even though at follow up they were diagnosed as no longer
having carcinoma. This study also drew heavily from those with Caucasian
ancestry making it possible that the observations may not be
representative of those with either Asian or African ancestry. Although
our training and test set are relatively large we still run the risk of
over-fitting or having a model that may not be representative of other
populations. We've done our best to safeguard against this by not only
running 10-fold cross validation but also having over 100 different
80/20 splits to try and mimic the type of variation that might be
expected to occur.

Despite these shortcomings our findings add to the existing scientific
knowledge on CRC and the microbiome: That there is a measurable
difference in the bacterial community after adenoma, SRN or carcinoma
treatment. Further, the ability for machine learning algorithms to take
OTU data and successfully lower positive probability of carcinoma after
treatment provides evidence that there are specific signatures,
attributable to both inflammtory and resident commensal organisms,
associated with treatment. Our data provides evidence that commensal
bacteria may be important in the development of polyps and potentially
the transition from adenoma to carcinoma.

\newpage

\subsection{Methods}\label{methods}

\textbf{\emph{Study Design and Patient Sampling:}} Sampling and design
have been previously reported in Baxter, et al {[}7{]}. Briefly, study
exclusion involved those who had already undergone surgery, radiation,
or chemotherapy, had colorectal cancer before a baseline fecal sample
could be obtained, had IBD, a known hereditary non-polyposis colorectal
cancer, or familial adenomatous polyposis. Samples used to build the
models for prediction were collected either prior to a colonoscopy or
between 1 - 2 weeks after. The bacterial community has been shown to
normalize back to a pre-colonscopy community within this time period
{[}18{]}. Our training cohort consisted of a total of 423 individuals
{[}Table 1{]}. Our study cohort consisted of 67 individuals with an
initial sample as described and a follow up sample obtained between 188
- 546 days after treatment of lesion {[}Table 2{]}. This study was
approved by the University of Michigan Institutional Review Board. All
study participants provided informed consent and the study itself
conformed to the guidelines set out by the Helsinki Declaration.

\textbf{\emph{16S rRNA Gene Sequencing:}} Sequencing was completed as
described by Kozich, et al. {[}19{]}. DNA extraction used the 96-well
Soil DNA isolation kit (MO BIO Laboratories) and an epMotion 5075
automated pipetting system (Eppendorf). The V4 variable region was
amplified and the resulting product was split between three sequencing
runs with normal, adenoma, and carcinoma evenly represented on each run.
Each group was randomly assigned to avoid biases based on sample
collection location. The initial and follow up samples were sequenced on
the same run.

\textbf{\emph{Sequence Processing:}} The mothur software package
(v1.37.5) was used to process the 16S rRNA gene sequences and has been
previously described {[}19{]}. The general workflow using mothur was:
Paired-end reads were first merged into contigs, quality filtered,
aligned to the SILVA database, screened for chimeras, classified with a
naive Bayesian classifier using the Ribosomal Database Project (RDP),
and clustered into Operational Taxonomic Units (OTUs) using a 97\%
similarity cutoff with an average neighbor clustering algorithm. The
number of sequences for each sample was rarefied to 10523 to minimize
uneven sampling.

\textbf{\emph{Lesion Model Creation:}} The Random Forest {[}20{]}
algorithm was used to create the model used for prediction of lesion
(adenoma or carcinoma) with the main training and testing of the model
completed on an independent data set of 423 individuals. This model was
then applied to our 67-person cohort. It should be noted that all
individuals with an adenoma or carcinoma were grouped together to form
the lesion group and the model was not created to find differences
between normal, adenoma, and carcinoma but rather differences between
both adenoma and carcinoma versus normal.

The model included only OTU data obtained from 16S rRNA sequencing.
Non-binary data was checked for near zero variance and OTUs that had
near zero variance were removed. This pre-processing was performed with
the R package caret (v6.0.73). Optimization of the mtry hyper-parameter
involved making 100 different 80/20 (train/test) splits of the data
where normal and lesion were represented in the same proportion within
both the whole data set and the 80/20 split. For each a 20 repeated
10-fold cross validation was performed on 80\% component to optimize the
mtry hyper-parameter by maximizing the AUC (Area Under the Curve of the
Receiver Operator Characteristic). The resulting model was then tested
on the hold out data obtained from the 20\% component. Assessment of the
most important OTUs to the model involved counting the number of times
an OTU was present in the top 10\% of mean decrease in accuracy (MDA)
for each of the 100 different splits run. This was then followed with
filtering of this list to variables that were only present in more than
50\% of these 100 runs. The final collated list of variables was then
run through the mtry optimization again. Once the ideal mtry was found
the entire 423 sample set was used to create the final Random Forest
model on which classifications on the 67-person cohort was completed.

The default cutoff of 0.5 was used as the threshold to classify
individuals as positive or negative for lesion. The hyper-parameter,
mtry, defines the number of variables to investigate at each split
before a new division of the data was created with the Random Forest
model.

\textbf{\emph{Treatment Model Creation:}} We also investigated whether a
model could be created that could identify pre- (initial) and post-
(follow up) treatment samples. The main difference was that only the
67-person cohort was used at all stages of model building and
classification. Other than this difference the creation of this model
and optimization of the mtry hyper-parameter was completed using the
same procedure as was used for the lesion model. Instead of classifying
samples as positive or negative of lesion this model classified samples
as positive or negative for being an initial sample prior to treatment.

\textbf{\emph{Statistical Analysis:}} The R software package (v3.3.2)
was used for all statistical analysis. Comparisons between bacterial
community structure utilized PERMANOVA {[}21{]} in the vegan package
(v2.4.1). Comparisons between probabilities as well as overall OTU
differences between initial and follow up samples utilized a paired
Wilcoxson ranked sum test. Where multiple comparison testing was
appropriate, a Benjamini-Hochberg (BH) correction was applied {[}22{]}
and a corrected P-value of less than 0.05 was considered significant.
Unless otherwise stated the P-values reported are those that were BH
corrected.

\textbf{\emph{Analysis Overview:}} We first tested for any differences
based on whether the individual had an adenoma or carcinoma. This was
done by testing initial and follow up samples for differences in alpha
and beta diversity, testing all OTUs, and investigating the relative
abundance of genera from previously associated CRC bacteria
(\emph{Fusobacterium}, \emph{Parvimonas}, \emph{Peptostreptococcus}, and
\emph{Porphyromonas}). Next, the lesion model was tested for accuracy in
prediction and whether it reduced the positive probability of lesion in
follow up samples. We then used the treatment model to assess whether it
could classify samples better than the lesion model and whether it could
reduce the positive probability of an initial sample in the follow up
samples. Common OTUs were found for the two different models used to
assess which were important for both models. Finally, differences
between those receiving chemotherapy and radiation versus those who
received neither were tested.

\textbf{\emph{Reproducible Methods:}} A detailed and reproducible
description of how the data were processed and analyzed can be found at
\url{https://github.com/SchlossLab/Sze_followUps_2017}. Raw sequences
have been deposited into the NCBI Sequence Read Archive (SRP062005 and
SRP096978) and the necessary metadata can be found at
\url{https://www.ncbi.nlm.nih.gov/Traces/study/} and searching the
respective SRA study accession.

\newpage

\textbf{Figure 1: General Differences between Adenoma, SRN, and
Carcinoma Groups After Treatment.} A) A significant difference was found
between the adenoma and carcinoma group for thetayc (P-value = NULL).
Advanced adenomas are denoted as Screen Relevant Neoplasia (SRN). B)
NMDS of the initial and follow up samples for the adenoma group. C) NMDS
of the initial and follow up samples for the carcinoma group.

\textbf{Figure 2: Treatment Response Based on Models Built for Adenoma,
SRN, or Carcinoma.} A) Positive probability change from initial to
follow up sample in those with adenoma. B) Positive probability change
from initial to follow up sample in those with SRN. C) Positive
probability change from initial to follow up sample in those with
carcinoma..

\textbf{Figure 3: The Treatment Model.} B) C) Positive probability
change from initial to follow up sample in those with carcinoma. D)
Positive probability change from initial to follow up sample of those
with adenoma or advanced adenoma (Screen Relevant Neoplasia (SRN)).

\newpage

\textbf{Table 1: Demographic Data of Training Cohort}

\textbf{Table 2: Demographic Data of Pre and Post Treatment Cohort}

\newpage

\textbf{Figure S1: Distribution of P-values from Paired Wilcoxson
Analysis of All OTUs Before and After Treatment}

\textbf{Figure S2: ROC Curves of the Adenoma, SRN, and Carcinoma
Models.} A) Adenoma ROC curve: The light greenshaded areas represent the
range of values of a 100 different 80/20 splits of the test set data and
the dark green line represents the model using 100\% of the data set and
what was used for subsequent classification. B) SRN ROC curve: The light
yellow shaded areas represent the range of values of a 100 different
80/20 splits of the test set data and the dark yellow line represents
the model using 100\% of the data set and what was used for subsequent
classification. C) Carcinoma ROC curve: The light red shaded areas
represent the range of values of a 100 different 80/20 splits of the
test set data and the dark red line represents the model using 100\% of
the data set and what was used for subsequent classification.

\textbf{Figure S3: Summary of Important Variables for the Adenoma, SRN,
and Carcinoma Models.} A) MDA of the most important variables in the
adenoma model. The dark green point represents the mean and the lighter
green points are the value of each of the 100 different runs. B) Summary
of Important Variables in the SRN model. MDA of the most important
variables in the SRN model. The dark yellow point represents the mean
and the lighter yellow points are the value of each of the 100 different
runs. C) MDA of the most important variables in the carcinoma model. The
dark red point represents the mean and the lighter redpoints are the
value of each of the 100 different runs.

\newpage

\subsection{Declarations}\label{declarations}

\subsubsection{Ethics approval and consent to
participate}\label{ethics-approval-and-consent-to-participate}

The University of Michigan Institutional Review Board approved this
study, and all subjects provided informed consent. This study conformed
to the guidelines of the Helsinki Declaration.

\subsubsection{Consent for publication}\label{consent-for-publication}

Not applicable.

\subsubsection{Availability of data and
material}\label{availability-of-data-and-material}

A detailed and reproducible description of how the data were processed
and analyzed can be found at
\url{https://github.com/SchlossLab/Sze_followUps_2017}. Raw sequences
have been deposited into the NCBI Sequence Read Archive (SRP062005 and
SRP096978) and the necessary metadata can be found at
\url{https://www.ncbi.nlm.nih.gov/Traces/study/} and searching the
respective SRA study accession.

\subsubsection{Competing Interests}\label{competing-interests}

All authors declare that they do not have any relevant competing
interests to report.

\subsubsection{Funding}\label{funding}

This study was supported by funding from the National Institutes of
Health to P. Schloss (R01GM099514, P30DK034933) and to the Early
Detection Research Network (U01CA86400).

\subsubsection{Authors' contributions}\label{authors-contributions}

All authors were involved in the conception and design of the study. MAS
analyzed the data. NTB processed samples and analyzed the data. All
authors interpreted the data. MAS and PDS wrote the manuscript. All
authors reviewed and revised the manuscript. All authors read and
approved the final manuscript.

\subsubsection{Acknowledgements}\label{acknowledgements}

The authors thank the Great Lakes-New England Early Detection Research
Network for providing the fecal samples that were used in this study. We
would also like to thank Amanda Elmore for reviewing and correcting code
error and providing feedback on manuscript drafts. We would also like to
thank Nicholas Lesniak for providing feedback on manuscript drafts.

\newpage

\subsection*{References}\label{references}
\addcontentsline{toc}{subsection}{References}

\hypertarget{refs}{}
\hypertarget{ref-jemal_cancer_2010}{}
1. Jemal A, Siegel R, Xu J, Ward E. Cancer statistics, 2010. CA: a
cancer journal for clinicians. 2010;60:277--300.

\hypertarget{ref-haggar_colorectal_2009}{}
2. Haggar FA, Boushey RP. Colorectal cancer epidemiology: Incidence,
mortality, survival, and risk factors. Clinics in Colon and Rectal
Surgery. 2009;22:191--7.

\hypertarget{ref-zackular_manipulation_2016}{}
3. Zackular JP, Baxter NT, Chen GY, Schloss PD. Manipulation of the Gut
Microbiota Reveals Role in Colon Tumorigenesis. mSphere. 2016;1.

\hypertarget{ref-arthur_microbial_2014}{}
4. Arthur JC, Gharaibeh RZ, Mühlbauer M, Perez-Chanona E, Uronis JM,
McCafferty J, et al. Microbial genomic analysis reveals the essential
role of inflammation in bacteria-induced colorectal cancer. Nature
Communications. 2014;5:4724.

\hypertarget{ref-dejea_microbiota_2014}{}
5. Dejea CM, Wick EC, Hechenbleikner EM, White JR, Mark Welch JL,
Rossetti BJ, et al. Microbiota organization is a distinct feature of
proximal colorectal cancers. Proceedings of the National Academy of
Sciences of the United States of America. 2014;111:18321--6.

\hypertarget{ref-zackular_gut_2013}{}
6. Zackular JP, Baxter NT, Iverson KD, Sadler WD, Petrosino JF, Chen GY,
et al. The gut microbiome modulates colon tumorigenesis. mBio.
2013;4:e00692--00613.

\hypertarget{ref-baxter_microbiota-based_2016}{}
7. Baxter NT, Ruffin MT, Rogers MAM, Schloss PD. Microbiota-based model
improves the sensitivity of fecal immunochemical test for detecting
colonic lesions. Genome Medicine. 2016;8:37.

\hypertarget{ref-zeller_potential_2014}{}
8. Zeller G, Tap J, Voigt AY, Sunagawa S, Kultima JR, Costea PI, et al.
Potential of fecal microbiota for early-stage detection of colorectal
cancer. Molecular Systems Biology. 2014;10:766.

\hypertarget{ref-flynn_metabolic_2016}{}
9. Flynn KJ, Baxter NT, Schloss PD. Metabolic and Community Synergy of
Oral Bacteria in Colorectal Cancer. mSphere. 2016;1.

\hypertarget{ref-hassan_efficacy_2016}{}
10. Hassan C, Repici A, Sharma P, Correale L, Zullo A, Bretthauer M, et
al. Efficacy and safety of endoscopic resection of large colorectal
polyps: A systematic review and meta-analysis. Gut. 2016;65:806--20.

\hypertarget{ref-yu_metagenomic_2017}{}
11. Yu J, Feng Q, Wong SH, Zhang D, Liang QY, Qin Y, et al. Metagenomic
analysis of faecal microbiome as a tool towards targeted non-invasive
biomarkers for colorectal cancer. Gut. 2017;66:70--8.

\hypertarget{ref-zackular_human_2014}{}
12. Zackular JP, Rogers MAM, Ruffin MT, Schloss PD. The human gut
microbiome as a screening tool for colorectal cancer. Cancer Prevention
Research (Philadelphia, Pa.). 2014;7:1112--21.

\hypertarget{ref-warren_co-occurrence_2013}{}
13. Warren RL, Freeman DJ, Pleasance S, Watson P, Moore RA, Cochrane K,
et al. Co-occurrence of anaerobic bacteria in colorectal carcinomas.
Microbiome. 2013;1:16.

\hypertarget{ref-louis_gut_2014}{}
14. Louis P, Hold GL, Flint HJ. The gut microbiota, bacterial
metabolites and colorectal cancer. Nature Reviews Microbiology
{[}Internet{]}. 2014 {[}cited 2017 Feb 14{]};12:661--72. Available from:
\url{http://www.nature.com/doifinder/10.1038/nrmicro3344}

\hypertarget{ref-zhu_intake_2016}{}
15. Zhu Y, Lin X, Li H, Li Y, Shi X, Zhao F, et al. Intake of Meat
Proteins Substantially Increased the Relative Abundance of Genus
Lactobacillus in Rat Feces. PloS One. 2016;11:e0152678.

\hypertarget{ref-mu_colonic_2016}{}
16. Mu C, Yang Y, Luo Z, Guan L, Zhu W. The Colonic Microbiome and
Epithelial Transcriptome Are Altered in Rats Fed a High-Protein Diet
Compared with a Normal-Protein Diet. The Journal of Nutrition.
2016;146:474--83.

\hypertarget{ref-ozdal_reciprocal_2016}{}
17. Ozdal T, Sela DA, Xiao J, Boyacioglu D, Chen F, Capanoglu E. The
Reciprocal Interactions between Polyphenols and Gut Microbiota and
Effects on Bioaccessibility. Nutrients {[}Internet{]}. 2016 {[}cited
2017 Feb 14{]};8:78. Available from:
\url{http://www.mdpi.com/2072-6643/8/2/78}

\hypertarget{ref-obrien_impact_2013}{}
18. O'Brien CL, Allison GE, Grimpen F, Pavli P. Impact of colonoscopy
bowel preparation on intestinal microbiota. PloS One. 2013;8:e62815.

\hypertarget{ref-kozich_development_2013}{}
19. Kozich JJ, Westcott SL, Baxter NT, Highlander SK, Schloss PD.
Development of a dual-index sequencing strategy and curation pipeline
for analyzing amplicon sequence data on the MiSeq Illumina sequencing
platform. Applied and Environmental Microbiology. 2013;79:5112--20.

\hypertarget{ref-breiman_random_2001}{}
20. Breiman L. Random Forests. Machine Learning {[}Internet{]}. 2001
{[}cited 2013 Feb 7{]};45:5--32. Available from:
\href{http://link.springer.com/article/10.1023/A\%3A1010933404324\%20http://link.springer.com/article/10.1023\%2FA\%3A1010933404324?LI=true}{http://link.springer.com/article/10.1023/A\%3A1010933404324 http://link.springer.com/article/10.1023\%2FA\%3A1010933404324?LI=true}

\hypertarget{ref-anderson_permanova_2013}{}
21. Anderson MJ, Walsh DCI. PERMANOVA, ANOSIM, and the Mantel test in
the face of heterogeneous dispersions: What null hypothesis are you
testing? Ecological Monographs {[}Internet{]}. 2013 {[}cited 2017 Jan
5{]};83:557--74. Available from:
\url{http://doi.wiley.com/10.1890/12-2010.1}

\hypertarget{ref-benjamini_controlling_1995}{}
22. Benjamini Y, Hochberg Y. Controlling the false discovery rate: A
practical and powerful approach to multiple testing. Journal of the
Royal Statistical Society. Series B (Methodological). 1995;57:289--300.


\end{document}
