\documentclass[12pt,]{article}
\usepackage{lmodern}
\usepackage{amssymb,amsmath}
\usepackage{ifxetex,ifluatex}
\usepackage{fixltx2e} % provides \textsubscript
\ifnum 0\ifxetex 1\fi\ifluatex 1\fi=0 % if pdftex
  \usepackage[T1]{fontenc}
  \usepackage[utf8]{inputenc}
\else % if luatex or xelatex
  \ifxetex
    \usepackage{mathspec}
  \else
    \usepackage{fontspec}
  \fi
  \defaultfontfeatures{Ligatures=TeX,Scale=MatchLowercase}
\fi
% use upquote if available, for straight quotes in verbatim environments
\IfFileExists{upquote.sty}{\usepackage{upquote}}{}
% use microtype if available
\IfFileExists{microtype.sty}{%
\usepackage{microtype}
\UseMicrotypeSet[protrusion]{basicmath} % disable protrusion for tt fonts
}{}
\usepackage[margin=1.0in]{geometry}
\usepackage{hyperref}
\hypersetup{unicode=true,
            pdfborder={0 0 0},
            breaklinks=true}
\urlstyle{same}  % don't use monospace font for urls
\usepackage{graphicx,grffile}
\makeatletter
\def\maxwidth{\ifdim\Gin@nat@width>\linewidth\linewidth\else\Gin@nat@width\fi}
\def\maxheight{\ifdim\Gin@nat@height>\textheight\textheight\else\Gin@nat@height\fi}
\makeatother
% Scale images if necessary, so that they will not overflow the page
% margins by default, and it is still possible to overwrite the defaults
% using explicit options in \includegraphics[width, height, ...]{}
\setkeys{Gin}{width=\maxwidth,height=\maxheight,keepaspectratio}
\IfFileExists{parskip.sty}{%
\usepackage{parskip}
}{% else
\setlength{\parindent}{0pt}
\setlength{\parskip}{6pt plus 2pt minus 1pt}
}
\setlength{\emergencystretch}{3em}  % prevent overfull lines
\providecommand{\tightlist}{%
  \setlength{\itemsep}{0pt}\setlength{\parskip}{0pt}}
\setcounter{secnumdepth}{0}
% Redefines (sub)paragraphs to behave more like sections
\ifx\paragraph\undefined\else
\let\oldparagraph\paragraph
\renewcommand{\paragraph}[1]{\oldparagraph{#1}\mbox{}}
\fi
\ifx\subparagraph\undefined\else
\let\oldsubparagraph\subparagraph
\renewcommand{\subparagraph}[1]{\oldsubparagraph{#1}\mbox{}}
\fi

%%% Use protect on footnotes to avoid problems with footnotes in titles
\let\rmarkdownfootnote\footnote%
\def\footnote{\protect\rmarkdownfootnote}

%%% Change title format to be more compact
\usepackage{titling}

% Create subtitle command for use in maketitle
\newcommand{\subtitle}[1]{
  \posttitle{
    \begin{center}\large#1\end{center}
    }
}

\setlength{\droptitle}{-2em}
  \title{}
  \pretitle{\vspace{\droptitle}}
  \posttitle{}
  \author{}
  \preauthor{}\postauthor{}
  \date{}
  \predate{}\postdate{}

\usepackage{helvet} % Helvetica font
\renewcommand*\familydefault{\sfdefault} % Use the sans serif version of the font
\usepackage[T1]{fontenc}

\usepackage[none]{hyphenat}

\usepackage{setspace}
\doublespacing
\setlength{\parskip}{1em}

\usepackage{lineno}

\usepackage{pdfpages}
\usepackage{comment}
\usepackage{lscape}

\begin{document}

\section{The Fecal Microbiome Before and After Treatment for Colorectal
Adenoma or
Carcinoma}\label{the-fecal-microbiome-before-and-after-treatment-for-colorectal-adenoma-or-carcinoma}

\vspace{25mm}

\begin{center}
Running Title: Human Microbiome before and after Colorectal Cancer

\vspace{10mm}

Marc A Sze${^1}$, Nielson T Baxter${^2}$, Mack T Ruffin IV${^3}$, Mary AM Rogers${^2}$, and Patrick D Schloss${^1}$${^\dagger}$

\vspace{20mm}

$\dagger$ To whom correspondence should be addressed: pschloss@umich.edu

$1$ Department of Microbiology and Immunology, University of Michigan, Ann Arbor, MI

$2$ Department of Internal Medicine, University of Michigan, Ann Arbor, MI   

$3$ Department of Family Medicine and Community Medicine, Penn State Hershey Medical Center, Hershey, PA    


\end{center}

\newpage

\linenumbers

\subsection{Abstract}\label{abstract}

\textbf{Background:} Colorectal cancer (CRC) is a worldwide health
problem and research suggests a correlation between the fecal bacterial
microbiome and CRC. Despite this, very little is known about what
happens to the microbiome after treatment for an adenoma or carcinoma.
This study tested the hypothesis that treatment for adenoma or carcinoma
results in changes towards a normal bacterial community. Specifically,
we tried to identify components within the community that were different
before and after treatment of adenoma, advanced adenoma, and carcinoma.

\textbf{Results:} There was a larger change in the bacterial community
in response to treatment for carcinoma versus adenoma (P-value
\textless{} 0.05) but not carcinoma versus advanced adenoma (P-value
\textgreater{} 0.05). There was a trend for increasingly less community
similarity, between samples pre- and post-treatment from adenoma to
advanced adenoma to carcinoma. Despite this, no difference was found in
the relative abundance of any specific OTU before and after treatment
for adenoma, advanced adenoma, or carcinoma groups (P-value
\textgreater{} 0.05). Using Random Forest models to assess whether
changes in post-treatment samples were towards a normal community, only
those with carcinoma had a significant decrease in positive probability
(P-value \textless{} 0.05); providing further evidence that treatment
has the greatest effect in those with carcinoma. The adenoma model used
a total of 62 OTUs, the SRN model used a total of 61 OTUs, and the
carcinoma model used a total of 59 OTUs. A total of 26 OTUs were common
to all three models with many classifying to commensal bacteria (e.g.
\emph{Lachnospiraceae}, \emph{Bacteroides}, \emph{Anaerostipes},
\emph{Blautia}, and \emph{Dorea}). Both chemotherapy and radiation did
not provide any additional changes to the bacterial community in those
treated for carcinoma (Pvalue \textgreater{} 0.05).

\textbf{Conclusions:} Our data partially supports the hypothesis that
the microbiome changes after treatment towards a normal community.
Individuals with carcinoma had more drastic differences to the overall
community than those with adenoma. Common OTUs to all models were
overwhelmingly from commensal bacteria, suggesting that these bacteria
may be important in initial polyp formation, development of advanced
adenoma, and transition to carcinoma.

\newpage

\subsubsection{Keywords}\label{keywords}

bacterial microbiome; colorectal cancer; polyps; FIT; post-surgery; risk
factors

\newpage

\subsection{Background}\label{background}

Colorectal cancer (CRC) is currently the third most common cause of
cancer deaths {[}1,2{]}. The rate of disease mortality has seen a
significant decrease, thanks mainly to improvements in screening
{[}1{]}. However, despite this improvement there are still approximately
50,000 deaths from the disease per year {[}2{]}. Some of these deaths
are due to disease recurrence, with current estimates that 20-30\% of
those undergoing resection having a CRC recurrence {[}3{]}. This is
important since approximately 30-50\% of those who have a recurrence
will die of CRC {[}4{]}. To reduce the burden on the healthcare system,
finding ways to better stratify those at highest risk of recurrence is
of great importance.

Recent studies in humans and mouse models have shown that altered
membership and structure of the gut microbiome correlate with CRC
pathogenesis {[}5--7{]}. Further, bacterial communities on the mucosa
are altered between normal and tumor tissue {[}8{]}. Collectively, these
studies provide a tantalizing link between our gut bacteria and CRC and
suggest the possibility of using biomarkers to diagnose disease. Indeed,
models created with microbiome data and clinical tests, such as Fecal
Immunoglobulin Test (FIT), result in good predictions of CRC {[}9,10{]}.
While these studies show how changed microbial communities or invasion
by inflammatory species can impact CRC progression {[}11{]}, they
provide very little information as to whether these communities change
and rebound towards normal after successful treatment of adenoma,
advanced adenoma, or carcinoma.

Providing an answer to this question is important because it has far
reaching implications on both how the bacterial community causes the
formation of more polyps {[}5,6{]} and the ability to use the microbiome
as a predictive screening tool {[}9,10{]}. Understanding polyp formation
and transition to advanced adenoma and then carcinoma is crucial to
being able to understand how to prevent CRC occurrence. Response of the
community to treatment is also equally important to predictive models
designed for screening purposes since an unresponsive community would
provide little additional information for important events, such as
recurrence {[}12{]}.

Using pre- (initial) and post- (follow up) treatment samples we tested
the hypothesis that treatment causes detectable changes to the
microbiome in those with adenoma, advanced adenoma, and carincoma.
First, we assessed differences between samples pre- and post-treatment
in adenoma, advanced adenoma, or carcinoma using alpha or beta diversity
metrics. Second, we explored whether models built to classify adenoma,
advanced adenoma, or carcinoma versus normal were able to identify
specific community members that differed between initial and follow up.
We also used these models to assess whether changes in the community
were toward a more normal microbiome. Finally, we assessed both whether
surgery for adenomas and SRN provided larger community changes or
whether chemotherapy or radiation provided additive changes to the
microbiome over surgical resection. This study aims to provide evidence
as to whether the altered CRC microbiome persists or shifts back towards
normal after such interventions.

\newpage

\subsection{Results}\label{results}

\textbf{\emph{The Bacterial Community:}} Within our 67-person cohort we
tested whether the microbiome in patients with adenoma (n = 22),
advanced adenoma (n = 19), or carcinoma (n = 26) had any broad
differences between pre- or post-treatment. We found that carcinoma
patients had a more dissimilar bacterial community between their initial
and follow up sample than those with adenoma (P-value \textless{} 0.001)
{[}Figure 1A{]}. Although no significant differences were observed
between advanced adenoma and carcinoma there was an increase in the
dissimilarity of the samples pre- and post-treatment from adenoma (0.55
± 0.21 (mean ± SD)) to advanced adenoma (0.65 ± 0.25) to carcinoma (0.78
± 0.15) {[}Figure 1A{]}. The bacterial community structure before and
after surgery was visualized using NMDS for adenoma {[}Figure 1B{]}
(PERMANOVA \textgreater{} 0.05), advanced adenoma {[}Figure 1C{]}
(PERMANOVA \textgreater{} 0.05), and carcinoma {[}Figure 1D{]}
(PERMANOVA \textless{} 0.05). Interestingly, when pre- and
post-treatment samples were compared, regardless of whether they were
adenoma or carcinoma, there was no significant overall difference in
beta diversity (PERMANOVA \textgreater{} 0.05). There was no difference
between pre- and post-treatment samples when investigating alpha
diversity metrics for adenoma, advanced adenoma, or carcinoma {[}Table
S1{]}. Additionally, there was also no difference in the relative
abundance of any specific OTU between pre- and post-treatment samples
for adenoma, advanced adenoma, or carcinoma only {[}Figure S1{]}.

\textbf{\emph{Adenoma Model:}} Using normal and adenoma individauls from
a separate cohort we created an adenoma model to help assess whether
those with adenoma in this cohort changed towards normal. The range of
model AUC's from 100 runs of 20 repeated 10-fold cross-validation was
0.62 - 0.72 with the AUC of the model used for classification having an
AUC of 0.65 {[}Figure S2A{]}. There was a total of 62 OTUs in this model
with the vast majority classifying to bacteria typically thought of as
commensal {[}Figure S3A{]}. There was a significant difference between
the actual and predicted disease classification (P-value \textless{}
0.05). There was also no significant decrease in the positive
probability of adenoma between pre- and post-treatment samples (P-value
\textgreater{} 0.05) {[}Figure 2A{]}.

\textbf{\emph{Advanced Adenoma Model:}} Using normal and advanced
adenoma individuals from a separate cohort we created an advanced
adenoma model to help assess whether those with advanced adenoma in this
cohort changed towards normal. The range of model AUC's from 100 runs of
20 repeated 10 fold cross-validation was 0.68 - 0.77 with the AUC of the
model used for classification having an AUC of 0.73 {[}Figure S2B{]}.
There was a total of 61 OTUs in the advanced adenoma model {[}Figure
S3B{]}. Similar to the adenoma model the vast majority of OTUs
classified to bacteria typically thought of as commensal. Also similar
to the adenoma model there was a significant difference between the
actual and predicted disease classification (P-value \textless{} 0.05)
and no significant decrease in the positive probability of advanced
adenoma between pre- and post-treatment samples (P-value \textgreater{}
0.05) {[}Figure 2B{]}.

\textbf{\emph{Carcinoma Model:}} Using normal and carcinoma individuals
from a separate cohort we created a carcinoma model to help assess
whether those with carcinoma in this cohort changed towards normal. The
range of model AUC's from 100 runs of 20 repeated 10 fold
cross-validation was 0.84 - 0.9 with the AUC of the model used for
classification being 0.88 {[}Figure S2C{]}. Interestingly, the AUCs
improved from adenoma to advanced adenoma to carcinoma {[}Figure 2{]}.
There was a total of 59 OTUs in the carcinoma model {[}Figure S3C{]}.
Similar to the adenoma and SRN models the vast majority of OTUs
classified to bacteria typically thought of as commensal but OTUs that
also classified to \emph{Fusobacterium}, \emph{Porphyromonas}, and
\emph{Parvimonas} are also important for carcinoma classification
{[}Figure S3C{]}. Also, like the adenoma and advanced adenoma models
there was a significant difference between the actual and predicted
disease classification (P-value \textless{} 0.05). There was a
significant decrease in the positive probability of carcinoma between
pre- and post-treatment samples (P-value \textless{} 0.05) {[}Figure
2C{]}; suggesting that the carcinoma samples changed towards normal
after treatment, unlike either adenoma or advanced adenoma. The one
individual still positive for carcinoma after treatment had an increase
in carcinoma positive proability between their pre- and post-treatment
sample {[}Figure 2C{]}.

\textbf{\emph{Adenoma, Advanced Adenoma, and Carcinoma Common OTUs:}}
Next, to identify which OTUs are important at all stages of disease, we
identified common predictive OTUs within the adenoma, advanced adenoma,
and carcinoma models. When we compared the three different models with
each other there were a total of 26 common OTUs. Some of the most common
taxonomic identifications belonged to \emph{Bacteroides},
\emph{Blautia}, \emph{Anaerostipes}, \emph{Lachnospiraceae}, and
\emph{Dorea}. The vast majority of the OTUs that were common between
these models had classifications to bacteria typically thought of as
commensal {[}Table S2{]}.

\textbf{\emph{Treatment Affects on Community:}} After observing these
changes from treatment we assessed the possible confounding impact of
chemotherapy or radiation, in the carcinoma group, and surgical
resection, in the adenoma group, on the observed results. In the
carcinoma group neither chemotherapy nor radiation provided any
additional change from initial sample over surgery alone (P-value
\textgreater{} 0.05) {[}Table S3{]}. For the adenoma group there was a
single difference in observed OTUs (sobs) between those that received
surgerical resection and those that had regular polyp removal (P-value
\textless{} 0.05) {[}Table S4{]}. For the surgical resection comparison,
adenoma and advanced adenoma were combined due to the low number of
resection occurances in these two groups. Using a fisher exact test
there was no difference in the proportion of those receiving surgerical
resection between the adenoma and advanced adenoma groups (P-value
\textgreater{} 0.05). This data suggests that the microbiome changes
observed in the carcinoma group were mostly a result from surgerical
resection and not from chemotherapy or radiation.

\newpage

\subsection{Discussion}\label{discussion}

This study builds upon previous work from numerous labs that have
considered both how the bacterial community between those with and
without CRC differs and how it might be used as an early screening tool
{[}9,10,13--15{]}. Here we show that the microbiome changes towards
normal after treatment for carcinoma and that chemotherapy and radiation
did not provide an additive change. Although some of the important OTUs
classified to genera from bacteria considered the ``usual suspects''"
(e.g. \emph{Fusobacterium}, \emph{Porphyromonas}, and \emph{Parvimonas})
many did not. The majority of important OTUs had taxonomic
classifications for resident gut microbes and were common for the
adenoma, advanced adenoma, and carcinoma models. This suggests that
members within the commensal community may be the first that change
during CRC pathogenesis. These subtle changes, in turn, could be the
first step in allowing more inflammatory bacteria to gain a foothold
within the colon {[}11{]}.

Unlike previous studies on the microbiome and CRC, ours focuses on
comparing the microbiome during recovery from treatment in adenoma,
advanced adenoma, and carcinoma groups. Although there were differences
for genera associated with specific bacterium linked with CRC {[}Figure
S3{]}, the majority of important OTUs taxonmically classified to
commensal bacteria {[}Figure S3{]}. Although these changes may be
subtle, due to the lack of significant differences in the bacterial
community pre- and post-treatment in adenoma and advanced adenoma
{[}Figure 1{]}; they support the hypothesis that the first members of
the community to change and potentially stay changed even after
treatment are those that are commensal bacteria.

Many of the common OTUs that we identified taxonomically classified to
potential butyrate producers (e.g. \emph{Clostridiales},
\emph{Roseburia}, and \emph{Anaerostipes}) {[}Table S2{]}. Other OTUs
classified to bacteria that are inhibited by polyphenols (e.g.
\emph{Bacteroides}). Both butyrate and polyphenols are thought to be
protective against cancer, in part by reducing inflammation {[}16{]}.
These protective compounds are derived from the breakdown of fiber,
fruits, and vegetables by resident gut microbes. One example of this
potential diet-microbiome-inflammation-polyp axis is that
\emph{Bacteroides}, which was highly prevalent in our models, are known
to be increased in those with high non-meat based protein consumption
{[}17{]}. High protein consumption in general has been linked with an
increased CRC risk {[}18{]}. Conversely, \emph{Bacteroides} are
inhibited by polyphenols which are derived from fruits and vegetables
{[}19{]}. Our data fits with the hypothesis that the microbial
metabolites from breakdown products within our own diet could not only
help to shape the existing community but also have an effect on CRC risk
and disease progression. Within this context the commensal community may
be an important modifiable risk factor for monitoring and preventing CRC
recurrence after treatment.

A limitation, in our study, was that there was a significant difference
in the time elapsed in the collection of the follow up sample between
adenoma or advanced adenoma versus carcinoma (P-value \textless{} 0.05),
with time passed being less for adenoma (255 ± 42 days) and advanced
adenoma (250 ± 41) than carcinoma (351 ± 102). These results would
indicate that some of the differences observed between the carcinoma and
adenoma groups could be due to differences in collection time.
Specifically, it could confound the observation that carcinomas changed
more than adenomas {[}Figure 1{]}. However, there are two reasons that
this may not be the case. First, the advanced adenoma group had a higher
dissimilarity than adenoma but lower dissimilarity then carcinoma and
the collection time to their follow up samples was less than the adenoma
group. Second, this confounding would not affect the observations where
models were used since they were built using a different cohort
{[}Figures 2 \& S2-S3{]}.

Another limitation of this study was that it drew heavily from those
with Caucasian ancestry making it possible that the observations may not
be representative of those with either Asian or African ancestry.
Although our training and test set are relatively large we still run the
risk of over-fitting or having a model that may not be representative of
other populations. We have done our best to safeguard against this by
not only running 10-fold cross validation but also having over 100
different 80/20 splits to try and mimic the type of variation that might
be expected to occur.

Building off of our results, an area for future research, is that within
our study we do not know whether individuals who were still classified
as positive by the carcinoma model eventually had a subsequent CRC
recurrence. This information would help to strengthen the case for this
model keeping numerous individuals above the cutoff threshold even
though at follow up they were diagnosed as no longer having carcinoma.
It would also provide additional evidence that the microbiome could be
used as a risk stratification tool in monitoring recurrence risk and
whether different interventions could potentially change this community
and lower the probability of future reccurence.

Despite the stated shortcomings our findings add to the existing
scientific knowledge on CRC and the microbiome: That there is a
measurable difference in the bacterial community after adenoma, advanced
adenoma, or carcinoma treatment. Further, the ability for machine
learning algorithms to take OTU data and successfully lower positive
probability of carcinoma after treatment provides evidence that there
are specific signatures, attributable to both inflammatory and resident
commensal organisms, associated with treatment. Our data provides
evidence that commensal bacteria may be important in the development of
polyps, potential transition of advanced adenoma to carcinoma, and
recovery of the microbiome in CRC following treatment.

\newpage

\subsection{Methods}\label{methods}

\textbf{\emph{Study Design and Patient Sampling:}} Sampling and design
have been previously reported in Baxter, et al {[}9{]}. Briefly, study
exclusion involved those who had already undergone surgery, radiation,
or chemotherapy, had colorectal cancer before a baseline fecal sample
could be obtained, had IBD, a known hereditary non-polyposis colorectal
cancer, or familial adenomatous polyposis. Samples used to build the
models for prediction were collected either prior to a colonoscopy or
between 1 - 2 weeks after. The bacterial community has been shown to
normalize back to a pre-colonoscopy community within this time period
{[}20{]}. Our training cohort consisted of a total of 423 individuals
{[}Table 1{]}. Our study cohort consisted of 67 individuals with an
initial sample as described and a follow up sample obtained between 188
- 546 days after treatment of lesion {[}Table 2{]}. This study was
approved by the University of Michigan Institutional Review Board. All
study participants provided informed consent and the study itself
conformed to the guidelines set out by the Helsinki Declaration.

\textbf{\emph{16S rRNA Gene Sequencing:}} Sequencing was completed as
described by Kozich, et al. {[}21{]}. DNA extraction used the 96-well
Soil DNA isolation kit (MO BIO Laboratories) and an epMotion 5075
automated pipetting system (Eppendorf). The V4 variable region was
amplified and the resulting product was split between three sequencing
runs with normal, adenoma, and carcinoma evenly represented on each run.
Each group was randomly assigned to avoid biases based on sample
collection location. The pre- and post-treatment samples were sequenced
on the same run.

\textbf{\emph{Sequence Processing:}} The mothur software package
(v1.37.5) was used to process the 16S rRNA gene sequences and has been
previously described {[}21{]}. The general workflow using mothur was:
Paired-end reads were first merged into contigs, quality filtered,
aligned to the SILVA database, screened for chimeras, classified with a
naive Bayesian classifier using the Ribosomal Database Project (RDP),
and clustered into Operational Taxonomic Units (OTUs) using a 97\%
similarity cutoff with an average neighbor clustering algorithm. The
number of sequences for each sample was rarefied to 10523 to minimize
uneven sampling.

\textbf{\emph{Model Building:}} The Random Forest {[}22{]} algorithm was
used to create the model used to create the three models used. The
adenoma model classified normal versus adenoma, advanced adenoma was
normal versus advanced adenoma, and carcinoma was normal versus
carcinoma. The toal number of individuals in this data set was 423
individuals. There were a total of 239 individuals in the adenoma model,
262 individuals in the advanced adenoma model, and 266 indivdiuals in
the carcinoma model {[}Table 1{]}. Each model was then applied to our
67-person cohort testing prediction of adenoma intial (adenoma n = 22)
versus adenoma follow up (adenoma n = 0), advanced adenoma initial
(advanced adenoma n = 19) versus advanced adenoma follow up (advanced
adenoma n = 0), carcinoma initial (carinoma n = 26) versus carcinoma
follow up (carcinoma n = 1).

The model included only OTU data obtained from 16S rRNA sequencing.
Non-binary data was checked for near zero variance and OTUs that had
near zero variance were removed. This pre-processing was performed with
the R package caret (v6.0.73). Optimization of the mtry hyper-parameter
involved making 100 different 80/20 (train/test) splits of the data
where normal and adenoma, normal and advanced adenoma, or normal and
carcinoma were represented in the same proportion within both the whole
data set and the 80/20 split. For each of the different splits, 20
repeated 10-fold cross validation was performed on the 80\% component to
optimize the mtry hyper-parameter by maximizing the AUC (Area Under the
Curve of the Receiver Operator Characteristic). The resulting model was
then tested on the hold out data obtained from the 20\% component. All
three models had an optimized mtry of 2.

Assessment of the most important OTUs to the model involved counting the
number of times an OTU was present in the top 10\% of mean decrease in
accuracy (MDA) for each of the 100 different splits run. This was then
followed with filtering of this list to variables that were only present
in more than 50\% of these 100 runs. The final collated list of
variables was then run through the mtry optimization again. Once the
ideal mtry was found the entire sample set specific to normal versus
adenoma, normal versus advanced adenoma, or normal versus carcinoma was
used to create the final Random Forest model on which classifications on
the 67-person cohort was completed. For all three models the final
optimized mtry was
\texttt{r}ifelse(as.data.frame(count(adn\_AUC\_run\_summary, best\_mtry)
\%\textgreater{}\% slice(which.max(n))){[}, ``best\_mtry''{]} ==
as.data.frame(count(srn\_AUC\_run\_summary, best\_mtry)
\%\textgreater{}\% slice(which.max(n))){[}, ``best\_mtry''{]} \&
as.data.frame(count(adn\_AUC\_run\_summary, best\_mtry)
\%\textgreater{}\% slice(which.max(n))){[}, ``best\_mtry''{]} ==
as.data.frame(count(crc\_AUC\_run\_summary, best\_mtry)
\%\textgreater{}\% slice(which.max(n))){[}, ``best\_mtry''{]},
as.data.frame(count(adn\_AUC\_run\_summary, best\_mtry)
\%\textgreater{}\% slice(which.max(n))){[}, ``best\_mtry''{]}, ``not the
same'')`

The default cutoff of 0.5 was used as the threshold to classify
individuals as positive or negative for lesion. The hyper-parameter,
mtry, defines the number of variables to investigate at each split
before a new division of the data was created with the Random Forest
model.

\textbf{\emph{Statistical Analysis:}} The R software package (v3.3.2)
was used for all statistical analysis. Comparisons between bacterial
community structure utilized PERMANOVA {[}23{]} in the vegan package
(v2.4.1). Comparisons between probabilities as well as overall OTU
differences between pre- and post-treatment samples utilized a paired
Wilcoxson ranked sum test. Where multiple comparison testing was
appropriate, a Benjamini-Hochberg (BH) correction was applied {[}24{]}
and a corrected P-value of less than 0.05 was considered significant.
Unless otherwise stated the P-values reported are those that were BH
corrected.

\textbf{\emph{Analysis Overview:}} We first tested whether there were
any differences between pre- and post-treatment samples in alpha and
beta diversity based on adenoma, SRN, or carcinoma. We then tested all
OTUs for differences between pre- and post-treatment samples. We next
used our specific models for adenoma, SRN, and carcinoma to test
classification accuracy, response towards a normal microbiome, and
common OTUs used across models. Finally, for the adenoma group
differences between those that received surgery or not was tested while
for the carcinoma group differences between those receiving chemotherapy
and radiation was tested.

\textbf{\emph{Reproducible Methods:}} A detailed and reproducible
description of how the data were processed and analyzed can be found at
\url{https://github.com/SchlossLab/Sze_followUps_2017}. Raw sequences
have been deposited into the NCBI Sequence Read Archive (SRP062005 and
SRP096978) and the necessary metadata can be found at
\url{https://www.ncbi.nlm.nih.gov/Traces/study/} and searching the
respective SRA study accession.

\newpage

\textbf{Figure 1: General Differences between Adenoma, Advanced Adenoma,
and Carcinoma Groups After Treatment.} A) A significant difference was
found between the adenoma and carcinoma group for thetayc (P-value =
NULL). B) NMDS of the pre- and post-treatment samples for the adenoma
group. C) NMDS of the pre- and post-treatment samples for the advanced
adenoma group. D) NMDS of the pre- and post-treatment samples for the
carcinoma group.

\textbf{Figure 2: Treatment Response Based on Models Built for Adenoma,
SRN, or Carcinoma.} A) Positive probability change from initial to
follow up sample in those with adenoma. B) Positive probability change
from initial to follow up sample in those with SRN. C) Positive
probability change from initial to follow up sample in those with
carcinoma..

\newpage

\textbf{Table 1: Demographic Data of Training Cohort}

\textbf{Table 2: Demographic Data of Pre and Post Treatment Cohort}

\newpage

\textbf{Figure S1: Distribution of P-values from Paired Wilcoxson
Analysis of All OTUs Before and After Treatment}

\textbf{Figure S2: ROC Curves of the Adenoma, Advanced Adenoma, and
Carcinoma Models.} A) Adenoma ROC curve: The light green shaded areas
represent the range of values of a 100 different 80/20 splits of the
test set data and the dark green line represents the model using 100\%
of the data set and what was used for subsequent classification. B)
Advanced Adenoma ROC curve: The light yellow shaded areas represent the
range of values of a 100 different 80/20 splits of the test set data and
the dark yellow line represents the model using 100\% of the data set
and what was used for subsequent classification. C) Carcinoma ROC curve:
The light red shaded areas represent the range of values of a 100
different 80/20 splits of the test set data and the dark red line
represents the model using 100\% of the data set and what was used for
subsequent classification.

\textbf{Figure S3: Summary of Important Variables for the Adenoma,
Advanced Adenoma, and Carcinoma Models.} A) MDA of the most important
variables in the adenoma model. The dark green point represents the mean
and the lighter green points are the value of each of the 100 different
runs. B) Summary of Important Variables in the advanced adenoma model.
MDA of the most important variables in the SRN model. The dark yellow
point represents the mean and the lighter yellow points are the value of
each of the 100 different runs. C) MDA of the most important variables
in the carcinoma model. The dark red point represents the mean and the
lighter red points are the value of each of the 100 different runs.

\newpage

\subsection{Declarations}\label{declarations}

\subsubsection{Ethics approval and consent to
participate}\label{ethics-approval-and-consent-to-participate}

The University of Michigan Institutional Review Board approved this
study, and all subjects provided informed consent. This study conformed
to the guidelines of the Helsinki Declaration.

\subsubsection{Consent for publication}\label{consent-for-publication}

Not applicable.

\subsubsection{Availability of data and
material}\label{availability-of-data-and-material}

A detailed and reproducible description of how the data were processed
and analyzed can be found at
\url{https://github.com/SchlossLab/Sze_followUps_2017}. Raw sequences
have been deposited into the NCBI Sequence Read Archive (SRP062005 and
SRP096978) and the necessary metadata can be found at
\url{https://www.ncbi.nlm.nih.gov/Traces/study/} and searching the
respective SRA study accession.

\subsubsection{Competing Interests}\label{competing-interests}

All authors declare that they do not have any relevant competing
interests to report.

\subsubsection{Funding}\label{funding}

This study was supported by funding from the National Institutes of
Health to P. Schloss (R01GM099514, P30DK034933) and to the Early
Detection Research Network (U01CA86400).

\subsubsection{Authors' contributions}\label{authors-contributions}

All authors were involved in the conception and design of the study. MAS
analyzed the data. NTB processed samples and analyzed the data. All
authors interpreted the data. MAS and PDS wrote the manuscript. All
authors reviewed and revised the manuscript. All authors read and
approved the final manuscript.

\subsubsection{Acknowledgements}\label{acknowledgements}

The authors thank the Great Lakes-New England Early Detection Research
Network for providing the fecal samples that were used in this study. We
would also like to thank Amanda Elmore for reviewing and correcting code
error and providing feedback on manuscript drafts. We would also like to
thank Nicholas Lesniak for providing feedback on manuscript drafts.

\newpage

\subsection*{References}\label{references}
\addcontentsline{toc}{subsection}{References}

\hypertarget{refs}{}
\hypertarget{ref-jemal_cancer_2010}{}
1. Jemal A, Siegel R, Xu J, Ward E. Cancer statistics, 2010. CA: a
cancer journal for clinicians. 2010;60:277--300.

\hypertarget{ref-haggar_colorectal_2009}{}
2. Haggar FA, Boushey RP. Colorectal cancer epidemiology: Incidence,
mortality, survival, and risk factors. Clinics in Colon and Rectal
Surgery. 2009;22:191--7.

\hypertarget{ref-hellinger_reoperation_2006}{}
3. Hellinger MD, Santiago CA. Reoperation for recurrent colorectal
cancer. Clinics in Colon and Rectal Surgery. 2006;19:228--36.

\hypertarget{ref-ryuk_predictive_2014}{}
4. Ryuk JP, Choi G-S, Park JS, Kim HJ, Park SY, Yoon GS, et al.
Predictive factors and the prognosis of recurrence of colorectal cancer
within 2 years after curative resection. Annals of Surgical Treatment
and Research. 2014;86:143--51.

\hypertarget{ref-zackular_manipulation_2016}{}
5. Zackular JP, Baxter NT, Chen GY, Schloss PD. Manipulation of the Gut
Microbiota Reveals Role in Colon Tumorigenesis. mSphere. 2016;1.

\hypertarget{ref-arthur_microbial_2014}{}
6. Arthur JC, Gharaibeh RZ, Mühlbauer M, Perez-Chanona E, Uronis JM,
McCafferty J, et al. Microbial genomic analysis reveals the essential
role of inflammation in bacteria-induced colorectal cancer. Nature
Communications. 2014;5:4724.

\hypertarget{ref-zackular_gut_2013}{}
7. Zackular JP, Baxter NT, Iverson KD, Sadler WD, Petrosino JF, Chen GY,
et al. The gut microbiome modulates colon tumorigenesis. mBio.
2013;4:e00692--00613.

\hypertarget{ref-dejea_microbiota_2014}{}
8. Dejea CM, Wick EC, Hechenbleikner EM, White JR, Mark Welch JL,
Rossetti BJ, et al. Microbiota organization is a distinct feature of
proximal colorectal cancers. Proceedings of the National Academy of
Sciences of the United States of America. 2014;111:18321--6.

\hypertarget{ref-baxter_microbiota-based_2016}{}
9. Baxter NT, Ruffin MT, Rogers MAM, Schloss PD. Microbiota-based model
improves the sensitivity of fecal immunochemical test for detecting
colonic lesions. Genome Medicine. 2016;8:37.

\hypertarget{ref-zeller_potential_2014}{}
10. Zeller G, Tap J, Voigt AY, Sunagawa S, Kultima JR, Costea PI, et al.
Potential of fecal microbiota for early-stage detection of colorectal
cancer. Molecular Systems Biology. 2014;10:766.

\hypertarget{ref-flynn_metabolic_2016}{}
11. Flynn KJ, Baxter NT, Schloss PD. Metabolic and Community Synergy of
Oral Bacteria in Colorectal Cancer. mSphere. 2016;1.

\hypertarget{ref-hassan_efficacy_2016}{}
12. Hassan C, Repici A, Sharma P, Correale L, Zullo A, Bretthauer M, et
al. Efficacy and safety of endoscopic resection of large colorectal
polyps: A systematic review and meta-analysis. Gut. 2016;65:806--20.

\hypertarget{ref-yu_metagenomic_2017}{}
13. Yu J, Feng Q, Wong SH, Zhang D, Liang QY, Qin Y, et al. Metagenomic
analysis of faecal microbiome as a tool towards targeted non-invasive
biomarkers for colorectal cancer. Gut. 2017;66:70--8.

\hypertarget{ref-zackular_human_2014}{}
14. Zackular JP, Rogers MAM, Ruffin MT, Schloss PD. The human gut
microbiome as a screening tool for colorectal cancer. Cancer Prevention
Research (Philadelphia, Pa.). 2014;7:1112--21.

\hypertarget{ref-warren_co-occurrence_2013}{}
15. Warren RL, Freeman DJ, Pleasance S, Watson P, Moore RA, Cochrane K,
et al. Co-occurrence of anaerobic bacteria in colorectal carcinomas.
Microbiome. 2013;1:16.

\hypertarget{ref-louis_gut_2014}{}
16. Louis P, Hold GL, Flint HJ. The gut microbiota, bacterial
metabolites and colorectal cancer. Nature Reviews Microbiology
{[}Internet{]}. 2014 {[}cited 2017 Feb 14{]};12:661--72. Available from:
\url{http://www.nature.com/doifinder/10.1038/nrmicro3344}

\hypertarget{ref-zhu_intake_2016}{}
17. Zhu Y, Lin X, Li H, Li Y, Shi X, Zhao F, et al. Intake of Meat
Proteins Substantially Increased the Relative Abundance of Genus
Lactobacillus in Rat Feces. PloS One. 2016;11:e0152678.

\hypertarget{ref-mu_colonic_2016}{}
18. Mu C, Yang Y, Luo Z, Guan L, Zhu W. The Colonic Microbiome and
Epithelial Transcriptome Are Altered in Rats Fed a High-Protein Diet
Compared with a Normal-Protein Diet. The Journal of Nutrition.
2016;146:474--83.

\hypertarget{ref-ozdal_reciprocal_2016}{}
19. Ozdal T, Sela DA, Xiao J, Boyacioglu D, Chen F, Capanoglu E. The
Reciprocal Interactions between Polyphenols and Gut Microbiota and
Effects on Bioaccessibility. Nutrients {[}Internet{]}. 2016 {[}cited
2017 Feb 14{]};8:78. Available from:
\url{http://www.mdpi.com/2072-6643/8/2/78}

\hypertarget{ref-obrien_impact_2013}{}
20. O'Brien CL, Allison GE, Grimpen F, Pavli P. Impact of colonoscopy
bowel preparation on intestinal microbiota. PloS One. 2013;8:e62815.

\hypertarget{ref-kozich_development_2013}{}
21. Kozich JJ, Westcott SL, Baxter NT, Highlander SK, Schloss PD.
Development of a dual-index sequencing strategy and curation pipeline
for analyzing amplicon sequence data on the MiSeq Illumina sequencing
platform. Applied and Environmental Microbiology. 2013;79:5112--20.

\hypertarget{ref-breiman_random_2001}{}
22. Breiman L. Random Forests. Machine Learning {[}Internet{]}. 2001
{[}cited 2013 Feb 7{]};45:5--32. Available from:
\href{http://link.springer.com/article/10.1023/A\%3A1010933404324\%20http://link.springer.com/article/10.1023\%2FA\%3A1010933404324?LI=true}{http://link.springer.com/article/10.1023/A\%3A1010933404324 http://link.springer.com/article/10.1023\%2FA\%3A1010933404324?LI=true}

\hypertarget{ref-anderson_permanova_2013}{}
23. Anderson MJ, Walsh DCI. PERMANOVA, ANOSIM, and the Mantel test in
the face of heterogeneous dispersions: What null hypothesis are you
testing? Ecological Monographs {[}Internet{]}. 2013 {[}cited 2017 Jan
5{]};83:557--74. Available from:
\url{http://doi.wiley.com/10.1890/12-2010.1}

\hypertarget{ref-benjamini_controlling_1995}{}
24. Benjamini Y, Hochberg Y. Controlling the false discovery rate: A
practical and powerful approach to multiple testing. Journal of the
Royal Statistical Society. Series B (Methodological). 1995;57:289--300.


\end{document}
